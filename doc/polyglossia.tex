% !TeX spellcheck = en_US
% !TeX TS-program = xelatex
\documentclass[11pt,nohyperref]{ltxdoc}
\usepackage{xcolor}
\usepackage{xspace,fancyvrb,longtable,booktabs}
\usepackage[neverdecrease]{paralist}
\usepackage[format=hang,labelfont=bf,labelsep=period]{caption}
\definecolor{xpgblue}{rgb}{0.02,0.04,0.48}
\definecolor{lightblue}{rgb}{0.61,.8,.8}
\definecolor{xpgred}{rgb}{0.65,0.04,0.07}
\usepackage[
    unicode=true,
    bookmarks=true,
    colorlinks=true,
    linkcolor=xpgblue,
    urlcolor=xpgblue,
    citecolor=xpgblue,
    hyperindex=false,
    hyperfootnotes=false,
    pdftitle={Polyglossia: Modern multilingual typesetting with XeLaTeX and LuaLaTeX},
    pdfauthor={F Charette, A Reutenauer, B Roucariès, J Spitzmüller},
    pdfkeywords={xetex, xelatex, luatex, lualatex, multilingual, babel, hyphenation}
    ]{hyperref}
\usepackage{array}
\usepackage{metalogo}
\let\XeTeX\undefined
\let\XeLaTeX\undefined
\usepackage[babelshorthands]{polyglossia}
\usepackage{farsical}
%
% Languages we need in the documentation. See below for fonts
\setmainlanguage[variant=british,ordinalmonthday=false]{english}
\setotherlanguages{arabic,armenian,hebrew,syriac,greek,russian,serbian,catalan,georgian,chinese}
\usepackage[protrusion,nopatch=footnote]{microtype}
\renewcommand\MacroFont{\footnotesize\color{xpgblue}\ttfamily}
% Customizable version of \DescribeMacro (e.g. if macro is too long)
\newcommand*\DescribeMacroCust[2]{%
  \let\origMacroFont\MacroFont%
  \renewcommand\MacroFont{\footnotesize\color{xpgblue}\ttfamily#1}%
  \DescribeMacro{#2}%
  \let\MacroFont\origMacroFont%
}
\newcommand*\Cmd[1]{\cmd{#1}\DescribeMacro{#1}\xspace}
\newcommand*\pkg[1]{\textsf{\color{xpgblue}#1}}
\newcommand*\file[1]{\texttt{\color{xpgblue}#1}}
\newcommand*\TR[1]{\textcolor{xpgred}{#1}}
\newcommand*\TX[1]{\hyperref[#1]{\textcolor{xpgred}{#1}}}
\newcommand*\TA[1]{\textsc{\color{xpgblue}#1}}
\newcommand*\link[1]{\href{#1}{#1}}
\renewcommand*\meta[1]{\texttt{⟨#1⟩}}
\newcommand*\TXI[1]{\href{https://github.com/reutenauer/polyglossia/issues/#1}{\textcolor{xpgred}{\##1}}}
\newcommand*\TXP[1]{\href{https://github.com/reutenauer/polyglossia/pull/#1}{\textcolor{xpgred}{\##1}}}
\def\eg{\textit{e.g.,}\xspace}
\def\ie{\textit{i.e.,}\xspace}
\def\ca{\textit{ca.}\@\xspace}
\def\Eg{\textit{E.g.,}\xspace}
\def\Ie{\textit{I.e.,}\xspace}
\def\etc{\@ifnextchar.{\textit{etc}}{\textit{etc.}\@\xspace}}

%
%% Commands for documenting options
%
\newcommand*\xpgoption[1]{\textcolor{xpgblue}{\ttfamily\bfseries #1}}
\newcommand*\xpgvalue[1]{\texttt{#1}}
\newcommand*\xpgpresetvalue[1]{\texttt{\textit{#1}}}
\newcommand*\xpgdefaultvalue[1]{\texttt{*#1}}

% arguments: #1 version number, #2 key name, #3 footnote, #4 possible values
\NewDocumentCommand\xpgchoicekey{omom}{%
	\xpgoption{#2}\IfValueT{#3}{\footnote{#3}}%
    \IfValueT{#1}{\new{#1}} \xpgvalue{=} #4\par%
}

% arguments: #1 version number, #2 key name, #3 footnote
\NewDocumentCommand\xpgboolkey{omo}{\xpgchoicekey[#1]{#2}[#3]{\xpgdefaultvalue{true} or \xpgvalue{false}}}
\NewDocumentCommand\xpgboolkeytrue{omo}{\xpgchoicekey[#1]{#2}[#3]{\xpgdefaultvalue{\xpgpresetvalue{true}} or \xpgvalue{false}}}
\NewDocumentCommand\xpgboolkeyfalse{omo}{\xpgchoicekey[#1]{#2}[#3]{\xpgdefaultvalue{true} or \xpgpresetvalue{false}}}

% arguments: #1 version number, #2 key name, #3 default code
\NewDocumentCommand\xpgcodekey{omv}{%
	\xpgoption{#2}\IfValueT{#1}{\new{#1}}
     \xpgvalue{=} \meta{code} (default value: \texttt{#3})\par%
}

% arguments: #1 version number, #2 key name, #3 value type, #4 default value
\NewDocumentCommand\xpgoptkey{ommo}{%
	\xpgoption{#2}\IfValueT{#1}{\new{#1}}
	\xpgvalue{=} \meta{#3} (default value: \texttt{#4})\par%
}

%% Sidenotes  << copied from fontspec.dtx
\newcommand\new[1]{%
  \edef\thisversion{v#1}%
  \ifhmode\unskip~\fi{\ifx\thisversion\fileversion\color{blue}\else\color[gray]{0.5}\fi
  $\leftarrow$}%
  \marginpar{\centering
    \small\ifx\thisversion\fileversion\color{blue}\else\color[gray]{0.5}\fi
    \textsf{v#1}}}
\newcommand\displaycmd[2]{%
  \\\DescribeMacro{#2}\centerline{\cmd{#1}}}
\renewenvironment{itemize}{\begin{compactitem}[\char"2023]}%[{\fontspec{DejaVu Sans}\char"25BB}]}%
		{\end{compactitem}}
\renewenvironment{enumerate}{\begin{compactenum}}{\end{compactenum}}
\newenvironment{shorthands}{%
	 \begin{list}{}%
	 	{\settowidth{\labelwidth}{MM}%
	 	 \setlength{\leftmargin}{\labelwidth}%
	 	 \addtolength{\leftmargin}{\labelsep}%
	     \renewcommand{\makelabel}[1]{##1\hfil}}}%
	{\end{list}}

% This is to prevent page breaks too short after subsections
\def\condbreak#1{%
	\vskip 0pt plus #1\pagebreak[3]\vskip 0pt plus -#1\relax}
\pretocmd{\subsection}{\condbreak{2\baselineskip}}{}{}

%
% Fonts
% We try to use fonts that are shipped with TeXLive
% (see #478)
%
%% fontspec declarations:
% Main font: Package libertine (Linux Libertine and Biolinum Mono)
\usepackage[osf]{libertine}
% Monospace: Package dejavu-otf
\setmonofont[Scale=MatchLowercase]{DejaVuSansMono}
% Libertine includes Hebrew, but we load it separately since
% osf is not supported with this script and we want to avoid the
% respective warnings
\newfontfamily\hebrewfont[Script=Hebrew]{Linux Libertine O}
% Package dejavu-otf
\newfontfamily\armenianfont[Script=Armenian,Scale=MatchLowercase]{DejaVuSans}
\newfontfamily\georgianfont[Scale=MatchLowercase]{DejaVuSerif}
% Package amiri
\newfontfamily\arabicfont[Script=Arabic]{Amiri-Regular}
% No Syriac otf/ttf font seems available on CTAN
% This one, from the Syriac Institute, is included on Linux
% in the x11-fonts set and can be downloaded from
% https://bethmardutho.org/meltho/ (Meltho Fonts Package)
\newfontfamily\syriacfont[Script=Syriac]{Serto Jerusalem}
% Our decorative title: package gfscomplutum
\newfontfamily\xpgtitlefont[Scale=1.4, LetterSpace=30]{GFSPolyglot.otf}
% Package noto, notoCJKsc
\newfontfamily\chinesefont{Noto Serif CJK SC}[Script=CJK]

\linespread{1.05}
\frenchspacing
\EnableCrossrefs
\CodelineIndex
\RecordChanges
% COMMENT THE NEXT LINE TO INCLUDE THE CODE
\AtBeginDocument{\OnlyDescription}


\begin{document}
\hyphenation{Kha-li-ghi Reu-ten-auer new-zea-land}
\GetFileInfo{polyglossia.sty}

\title{{\textgreek{\xpgtitlefont\color{lightblue}\MakeUppercase[lang=el]{Πολυγλωσσία}}}
\\[16pt]
\color{xpgblue}Polyglossia: Modern multilingual typesetting with \XeLaTeX\ and \LuaLaTeX}
\author{\TA{François Charette} \and \TA{Arthur Reutenauer}\thanks{Current maintainer}
	    \and \TA{Bastien Roucariès} \and \TA{Jürgen Spitzmüller}}
\date{\filedate \qquad \fileversion\thanks{Please report bugs to \url{http://github.com/reutenauer/polyglossia/issues}}\\
\footnotesize (\textsc{pdf} file generated on \today)}

\maketitle
\tableofcontents

\condbreak{4\baselineskip}


\DeleteShortVerb{\|}
\MakeShortVerb{\¦}

%\begin{abstract}
%Blablabla
%\end{abstract}


\section{Introduction}

\pkg{Polyglossia} is a package for facilitating multilingual typesetting with
\XeLaTeX\ and \LuaLaTeX. Basically, it
can be used as an alternative to \pkg{babel} for performing the following
tasks automatically:

\begin{enumerate}
\item Loading the appropriate hyphenation patterns.
\item Setting the script and language tags of the current font (if possible and
      available), via the package \pkg{fontspec}.
\item Switching to a font assigned by the user to a particular script or language.
\item Adjusting some typographical conventions according to the current language
      (such as afterindent, frenchindent, spaces before or after punctuation marks,
      etc.).
\item Redefining all document strings (like “chapter”, “figure”, “bibliography”).
\item Adapting the formatting of dates (for non-Gregorian calendars via external
      packages bundled with polyglossia: currently the Hebrew, Islamic and Farsi
      calendars are supported).
\item For languages that have their own numbering system, modifying the formatting
      of numbers appropriately (this also includes redefining the alphabetic sequence
      for non-Latin alphabets).\footnote{%
        This is done by bundled sub-packages such as \pkg{arabicnumbers}.}
\item Ensuring proper directionality if the document contains languages
      that are written from right to left (via the package \pkg{bidi},
      available separately).
\end{enumerate}
%
Several features of \pkg{babel} that do not make sense in the \XeTeX\ world (like font
encodings, shorthands, etc.) are not supported.
Generally speaking, \pkg{polyglossia} aims to remain as compatible as possible
with the fundamental features of \pkg{babel} while being cleaner, light-weight,
and modern. The package \pkg{antomega} has been very beneficial in our attempt to
reach this objective.

\paragraph{Requirements} The current version of \pkg{polyglossia} makes use of some convenient
macros defined in the \pkg{etoolbox} package by \TA{Philipp Lehmann} and \TA{Joseph Wright}.
Being designed for \XeLaTeX\ and \LuaLaTeX, it obviously also relies on \pkg{fontspec} by
\TA{Will Robertson}. For languages written from right to left, it needs the package \pkg{bidi}
(for \XeTeX) or \pkg{luabidi} (for \LuaTeX) by \TA{Vafa Khalighi} (\textarabic{وفا خليقي}) and
the \pkg{bidi-tex GitHub Organisation}.
Polyglossia also bundles three packages for calendaric computations (\pkg{hebrewcal},
\pkg{hijrical}, and \pkg{farsical}).


\section{Setting up multilingual documents}

\subsection{Activating languages}

The default language of a document is specified by means of the command
	\displaycmd{\setdefaultlanguage\oarg{options}\marg{lang}}{\setdefaultlanguage}
(or, equivalently, \Cmd\setmainlanguage).
Secondary languages are specified with
	\displaycmd{\setotherlanguage\oarg{options}\marg{lang}.}{\setotherlanguage}
All these commands allow you to set language-specific options.\footnote{%
	Section~\ref{specific} documents these options for the respective languages.}
It is also possible to load a series of secondary languages at once (but without options)
using
	\displaycmd{\setotherlanguages\marg{lang1⟩,⟨lang2⟩,⟨lang3⟩,⟨…}.}{\setotherlanguages}
%
All language-specific options can be modified locally by means of the
language-switching commands described in section \ref{languageswitching}.

\paragraph{Note} In general, it is advisable to activate the languages \emph{after} all
packages have been loaded. This is particularly important if you use right-to-left scripts
or languages with babel shorthands.


\subsection{Supported languages}

Table~\ref{tab:lang} lists all languages currently supported.
Those in \TR{red} have specific options and/or commands
that are explained in section \ref{specific} below.

\begin{table}[ht]\centering
\caption{\label{tab:lang}Languages currently supported in \pkg{polyglossia}}
% Produced with tools/insert-language-list.rb -- JS, 2022-11-23
% Edited by hand -- JS, 2022-11-23
\begin{tabular}{lllll}
	\toprule
	\TX{afrikaans}	& danish		& icelandic		& nko			& \TX{spanish}	\\
	albanian		& divehi		& interlingua	& \TX{norwegian}& swedish		\\
	amharic			& \TX{dutch}	& \TX{italian}	& occitan		& \TX{syriac}	\\
	\TX{arabic}		& \TX{english}	& japanese		& \TX{persian}	& tamil			\\
	\TX{armenian}	& \TX{esperanto}& kannada		& piedmontese	& telugu		\\
	asturian		& estonian		& khmer			& \TX{polish}	& \TX{thai}		\\
	basque			& \TX{finnish}	& \TX{korean}	&\TX{portuguese}& \TX{tibetan}	\\
	\TX{belarusian}	& \TX{french}	& \TX{kurdish}	& \TX{punjabi}	& turkish		\\
	\TX{bengali}	& friulian		& \TX{lao}		& romanian		& turkmen		\\
	bosnian			& \TX{gaelic}	& \TX{latin}	& romansh		& \TX{ukrainian}\\
	breton			& galician		& latvian		& \TX{russian}	& \TX{urdu}		\\
	bulgarian		& \TX{georgian}	& lithuanian	& \TX{sami}		& \TX{uyghur}	\\
	\TX{catalan}	& \TX{german}	& macedonian	& \TX{sanskrit}	& vietnamese	\\
	\TX{chinese}	& \TX{greek}	& \TX{malay}	& \TX{serbian}	& \TX{welsh}	\\
	coptic			& \TX{hebrew}	& malayalam		& \TX{slovak}	& 				\\
	\TX{croatian}	& \TX{hindi}	& \TX{marathi}	& \TX{slovenian}& 				\\
	\TX{czech}		& \TX{hungarian}& \TX{mongolian}& \TX{sorbian}	& 				\\
	\bottomrule
\end{tabular}
\end{table}

\paragraph{Version Notes} The support for Amharic\new{1.0.1} should be considered an experimental attempt to
port the package \pkg{ethiop}; feedback is welcome.
Version 1.1.1\new{1.1.1} added support for Asturian, %\footnote{ Provided by Kevin Godby and Xuacu Saturio.},
Lithuanian, %\footnote{ Provided by Kevin Godby and Paulius Sladkevičius.},
and Urdu. %\footnote{ Provided by Kamal Abdali.}
%
Version 1.2\new{1.2.0} introduced Armenian, Occitan, Bengali,
Lao, Malayalam, Marathi, Tamil, Telugu, and Turkmen.\footnote{%
  See acknowledgements at the end for due credit to the various contributors.}
Version 1.43\new{1.43} brought basic support for Japanese (this
is considered experimental, feedback is appreciated).
In version 1.45\new{1.45}, support for Kurdish and Mongolian as well as some new
variants (Canadian French and English) have been added. Furthermore, for consistency reasons, some language have
been renamed (\emph{farsi}\textrightarrow\emph{persian}, \emph{friulan}\textrightarrow\emph{friulian},
\emph{magyar}\textrightarrow\emph{hungarian}, \emph{portuges}\textrightarrow\emph{portuguese},
\emph{samin}\textrightarrow\emph{sami}) or merged (\emph{bahasai}\slash\emph{bahasam}\textrightarrow\emph{malay},
\emph{brazil}\slash\emph{portuges}\textrightarrow\emph{portuguese},
\emph{lsorbian}\slash\emph{usorbian}\textrightarrow\emph{sorbian},
\emph{irish}\slash\emph{scottish}\textrightarrow\emph{gaelic},
\emph{norsk}\slash\emph{nynorsk}\textrightarrow\emph{norwegian}). The old names are still supported for backwards
compatibility reasons.
Version 1.46\new{1.46} introduces support for Afrikaans, Belarusian, Bosnian and Georgian.
Version 1.52\new{1.52} introduces support for Uyghur.


\subsection{Relation to and use of Babel language names}\label{sec:babelnames}

If you are familiar with the \pkg{babel} package, you will note that \pkg{polyglossia}'s language naming
slightly differs. Whereas \pkg{babel} has a unique name for each language variety (\eg\emph{american} and \emph{british}),
\pkg{polyglossia} differentiates language varieties via language options (\eg \emph{english}, ¦variant=american¦).

Furthermore, \pkg{babel} sometimes uses abbreviated language names (\eg\emph{bahasam} for Bahasa Malayu) as well
as endonyms, \ie language names coming from the designated languages (such as \emph{bahasa}, \emph{canadien} or \emph{magyar}).
As opposed to this, \pkg{polyglossia} always uses spelled-out (lower-cased) English language names.
Please refer to table~\ref{tab:bbllang} for the differing language names in both packages.

\begin{table}
\caption{\label{tab:bbllang}Babel-polyglossia language name matching}

\begin{minipage}[t]{1\columnwidth}
\small\centering
\begin{tabular}{lll}
\toprule
\textbf{Babel name} & \textbf{Polyglossia name} & \textbf{Polyglossia options}\tabularnewline
\midrule
acadien            & french     & variant=acadian                     \\
american           & english    & variant=american [\emph{default}]   \\
australian         & english    & variant=australian                  \\
austrian           & german     & variant=austrian, spelling=old      \\
bahasa             & malay      & variant=indonesian [\emph{default}] \\
bahasam            & malay      & variant=malaysian                   \\
brazil             & portuguese & variant=brazilian                   \\
british            & english    & variant=british                     \\
canadian           & english    & variant=canadian                    \\
canadien           & french     & variant=canadian                    \\
classiclatin       & latin      & variant=classic                     \\
farsi              & persian    &                                     \\
ecclesiasticlatin  & latin      & variant=ecclesiastic                \\
friulan            & friulian   &                                     \\
german\footnote{Due to the name conflict only available in \pkg{polyglossia} as \emph{germanb} (which is a \pkg{babel} synonym).}
                   & german     & spelling=old                        \\
irish              & gaelic     & variant=irish [\emph{default}]      \\
kurmanji           & kurdish    & variant=kurmanji                    \\
lowersorbian       & sorbian    & variant=lower                       \\
magyar             & hungarian  &                                     \\
medievallatin      & latin      & variant=medieval                    \\
naustrian          & german     & variant=austrian                    \\
newzealand         & english    & variant=newzealand                  \\
ngerman            & german     & variant=german [\emph{default}]     \\
norsk              & norwegian  & variant=bokmal                      \\
nswissgerman       & german     & variant=swiss                       \\
nynorsk            & norwegian  & variant=nynorsk [\emph{default}]    \\
polutonikogreek    & greek      & variant=polytonic                   \\
portuges           & portuguese & variant=portuguese [\emph{default}] \\
samin              & sami       &                                     \\
scottish           & gaelic     & variant=scottish                    \\
serbianc           & serbian    & script=Cyrillic                     \\
slovene            & slovenian  &                                     \\
spanishmx          & spanish    & variant=mexican                     \\
swissgerman        & german     & variant=swiss, spelling=old         \\
uppersorbian       & sorbian    & variant=upper [\emph{default}]      \\
\bottomrule
\end{tabular}
\end{minipage}

\end{table}

For convenience reasons, \pkg{polyglossia} also supports the use of babel names\new{1.46} (for the few justified
exceptions, please refer to the notes in table~\ref{tab:bbllang}).
The babel names listed in table~\ref{tab:bbllang} can be used instead of the corresponding polyglossia
name\slash options in \cmd\setdefaultlanguage\ and \cmd\setotherlanguage\ as well as in the \pkg{polyglossia} and
\pkg{babel} language switching commands\slash environments documented in section~\ref{sec:langcmds} and
\ref{sec:babelcmds} (\eg \cmd\textaustrian\ is synonymous to ¦\textgerman[variant=austrian,spelling=old]¦).
However, unless you have special reasons, we strongly encourage you to use the \pkg{polyglossia} names.


\subsection{Using IETF language tags}\label{sec:langtags}

\pkg{Polyglossia}\new{1.47} also supports the use of language tags that conform to the IETF BCP-47
\emph{Best Current Practice}.\footnote{Please refer to \url{https://tools.ietf.org/html/bcp47} and
	\url{https://en.wikipedia.org/wiki/IETF_language_tag} for details.}
Thus, you can use tags such as ¦en-GB¦ (for British English) or ¦de-AT-1901¦ (for Austrian German, old spelling)
in \cmd\setdefaultlanguage\ and \cmd\setotherlanguage\ as well as in the language switching command
\cmd{\textlang\marg{tag}}, the environment \cmd{\begin\{lang\}\marg{tag}} \ldots\ \cmd{\end\{lang\}} and the \pkg{babel}
language switching commands\slash environments documented in section~\ref{sec:babelcmds}.
Table~\ref{tab:BCP47-polyglossia} lists the currently supported tags.

\bgroup\small\addfontfeatures{Numbers={Monospaced,Lining}}
\begin{longtable}[c]{ll>{\raggedright}p{6cm}}
	\caption{\label{tab:BCP47-polyglossia}BCP47-polyglossia language name matching}\tabularnewline
	\toprule
	\textbf{BCP-47 tag} & \textbf{Polyglossia name} & \textbf{Polyglossia options}\tabularnewline
	\midrule
	\endfirsthead
	\caption[]{BCP47-polyglossia language name matching (\emph{continued})}\tabularnewline
	\toprule
	\textbf{BCP-47 tag} & \textbf{Polyglossia name} & \textbf{Polyglossia options}\tabularnewline
	\midrule
	\endhead
	aeb              & arabic      & locale=tunisia                                     \tabularnewline
	af               & afrikaans   &                                                    \tabularnewline
	afb              & arabic      & locale=default                                     \tabularnewline
	am               & amharic     &                                                    \tabularnewline
	apd              & arabic      & locale=default                                     \tabularnewline
	ar               & arabic      &                                                    \tabularnewline
	ar-IQ            & arabic      & locale=mashriq                                     \tabularnewline
	ar-JO            & arabic      & locale=mashriq                                     \tabularnewline
	ar-LB            & arabic      & locale=mashriq                                     \tabularnewline
	ar-MR            & arabic      & locale=mauritania                                  \tabularnewline
	ar-PS            & arabic      & locale=mashriq                                     \tabularnewline
	ar-SY            & arabic      & locale=mashriq                                     \tabularnewline
	ar-YE            & arabic      & locale=default                                     \tabularnewline
	arq              & arabic      & locale=algeria                                     \tabularnewline
	ary              & arabic      & locale=morocco                                     \tabularnewline
	arz              & arabic      & locale=default                                     \tabularnewline
	ast              & asturian    &                                                    \tabularnewline
	ayl              & arabic      & locale=libya                                       \tabularnewline
	be               & belarusian  &                                                    \tabularnewline
	be-tarask        & belarusian  & spelling=classic                                   \tabularnewline
	bg               & bulgarian   &                                                    \tabularnewline
	bn               & bengali     &                                                    \tabularnewline
	bo               & tibetan     &                                                    \tabularnewline
	br               & breton      &                                                    \tabularnewline
	bs               & bosnian     &                                                    \tabularnewline
	ca               & catalan     &                                                    \tabularnewline
	ckb              & kurdish     & variant=sorani [\emph{default}]                    \tabularnewline
	ckb-Arab         & kurdish     & variant=sorani, script=Arabic [\emph{default}]     \tabularnewline
	ckb-Latn         & kurdish     & variant=sorani, script=Latin                       \tabularnewline
	cop              & coptic      &                                                    \tabularnewline
	cy               & welsh       &                                                    \tabularnewline
	cz               & czech       &                                                    \tabularnewline
	da               & danish      &                                                    \tabularnewline
	de               & german      &                                                    \tabularnewline
	de-AT            & german      & variant=austrian, spelling=new                     \tabularnewline
	de-AT-1901       & german      & variant=austrian, spelling=old                     \tabularnewline
	de-AT-1996       & german      & variant=austrian, spelling=new                     \tabularnewline
	de-CH            & german      & variant=swiss, spelling=new                        \tabularnewline
	de-CH-1901       & german      & variant=swiss, spelling=old                        \tabularnewline
	de-CH-1996       & german      & variant=swiss, spelling=new                        \tabularnewline
	de-DE            & german      & variant=german, spelling=new                       \tabularnewline
	de-DE-1901       & german      & variant=german, spelling=old                       \tabularnewline
	de-DE-1996       & german      & variant=german, spelling=new  [\emph{default}]     \tabularnewline
	de-Latf          & german      & script=blackletter                                 \tabularnewline
	de-Latf-AT       & german      & variant=austrian, spelling=new, script=blackletter \tabularnewline
	de-Latf-AT-1901  & german      & variant=austrian, spelling=old, script=blackletter \tabularnewline
	de-Latf-AT-1996  & german      & variant=austrian, spelling=new, script=blackletter \tabularnewline
	de-Latf-CH       & german      & variant=swiss, spelling=new, script=blackletter    \tabularnewline
	de-Latf-CH-1901  & german      & variant=swiss, spelling=old, script=blackletter    \tabularnewline
	de-Latf-CH-1996  & german      & variant=swiss, spelling=new, script=blackletter    \tabularnewline
	de-Latf-DE       & german      & variant=german, spelling=new, script=blackletter   \tabularnewline
	de-Latf-DE-1901  & german      & variant=german, spelling=old, script=blackletter   \tabularnewline
	de-Latf-DE-1996  & german      & variant=german, spelling=new, script=blackletter   \tabularnewline
	dsb              & sorbian     & variant=lower                                      \tabularnewline
	dv               & divehi      &                                                    \tabularnewline
	el               & greek       &                                                    \tabularnewline
	el-monoton       & greek       & variant=monotonic  [\emph{default}]                \tabularnewline
	el-polyton       & greek       & varant=polytonic                                   \tabularnewline
	en               & english     &                                                    \tabularnewline
	en-AU            & english     & variant=australian                                 \tabularnewline
	en-CA            & english     & variant=canadian                                   \tabularnewline
	en-GB            & english     & variant=british                                    \tabularnewline
	en-NZ            & english     & variant=newzealand                                 \tabularnewline
	en-US            & english     & variant=us [\emph{default}]                        \tabularnewline
	eo               & esperanto   &                                                    \tabularnewline
	es               & spanish     &                                                    \tabularnewline
	es-ES            & spanish     & variant=spanish [\emph{default}]                   \tabularnewline
	es-MX            & spanish     & variant=mexican                                    \tabularnewline
	et               & estonian    &                                                    \tabularnewline
	eu               & basque      &                                                    \tabularnewline
	fa               & persian     &                                                    \tabularnewline
	fi               & finnish     &                                                    \tabularnewline
	fr               & french      &                                                    \tabularnewline
	fr-CA            & french      & variant=canadian                                   \tabularnewline
	fr-CA-u-sd-canb  & french      & variant=acadian                                    \tabularnewline
	fr-CH            & french      & variant=swiss                                      \tabularnewline
	fr-FR            & french      & variant=french [\emph{default}]                    \tabularnewline
	fur              & friulian    &                                                    \tabularnewline
	ga               & gaelic      & variant=irish [\emph{default}]                     \tabularnewline
	gd               & gaelic      & variant=scottish                                   \tabularnewline
	gl               & galician    &                                                    \tabularnewline
	grc              & greek       & variant=ancient                                    \tabularnewline
	he               & hebrew      &                                                    \tabularnewline
	hi               & hindi       &                                                    \tabularnewline
	hr               & croatian    &                                                    \tabularnewline
	hsb              & sorbian     & variant=upper [\emph{default}]                     \tabularnewline
	hu               & hungarian   &                                                    \tabularnewline
	hy               & armenian    &                                                    \tabularnewline
	ia               & interlingua &                                                    \tabularnewline
	id               & malay       & variant=indonesian                                 \tabularnewline
	is               & icelandic   &                                                    \tabularnewline
	it               & italian     &                                                    \tabularnewline
	ja               & japanese    &                                                    \tabularnewline
	ka               & georgian    &                                                    \tabularnewline
	km               & khmer       &                                                    \tabularnewline
	kmr              & kurdish     & variant=kurmanji                                   \tabularnewline
	kmr-Arab         & kurdish     & variant=kurmanji, script=Arabic                    \tabularnewline
	kmr-Latn         & kurdish     & variant=kurmanji, script=Latin                     \tabularnewline
	kn               & kannada     &                                                    \tabularnewline
	ko               & korean      &                                                    \tabularnewline
	ku               & kurdish     &                                                    \tabularnewline
	ku-Arab          & kurdish     & script=Arabic                                      \tabularnewline
	ku-Latn          & kurdish     & script=Latin                                       \tabularnewline
	la               & latin       &                                                    \tabularnewline
	la-x-classic     & latin       & variant=classic                                    \tabularnewline
	la-x-ecclesia    & latin       & variant=ecclesiastic                               \tabularnewline
	la-x-medieval    & latin       & variant=medieval                                   \tabularnewline
	lo               & lao         &                                                    \tabularnewline
	lt               & lithuanian  &                                                    \tabularnewline
	lv               & latvian     &                                                    \tabularnewline
	mk               & macedonian  &                                                    \tabularnewline
	ml               & malayalam   &                                                    \tabularnewline
	mn               & mongolian   &                                                    \tabularnewline
	mr               & marathi     &                                                    \tabularnewline
	nb               & norwegian   & variant=bokmal                                     \tabularnewline
	nko              & nko         &                                                    \tabularnewline
	nl               & dutch       &                                                    \tabularnewline
	nn               & norwegian   & variant=nynorsk [\emph{default}]                   \tabularnewline
	oc               & occitan     &                                                    \tabularnewline
	pa               & punjabi     & numerals=gurmukhi                                  \tabularnewline
	pl               & polish      &                                                    \tabularnewline
	pms              & piedmontese &                                                    \tabularnewline
	pt               & portuguese  &                                                    \tabularnewline
	pt-BR            & portuguese  & variant=brazilian                                  \tabularnewline
	pt-PT            & portuguese  & variant=portuguese [\emph{default}]                \tabularnewline
	rm               & romansh     &                                                    \tabularnewline
	ro               & romanian    &                                                    \tabularnewline
	ru               & russian     &                                                    \tabularnewline
	ru-luna1918      & russian     & spelling=modern [\emph{default}]                   \tabularnewline
	ru-petr1708      & russian     & spelling=old                                       \tabularnewline
	sa               & sanskrit    &                                                    \tabularnewline
	sa-Beng          & sanskrit    & script=Bengali                                     \tabularnewline
	sa-Deva          & sanskrit    & script=Devanagari [\emph{default}]                 \tabularnewline
	sa-Gujr          & sanskrit    & script=Gujarati                                    \tabularnewline
	sa-Knda          & sanskrit    & script=Kannada                                     \tabularnewline
	sa-Mlym          & sanskrit    & script=Malayalam                                   \tabularnewline
	sa-Telu          & sanskrit    & script=Telugu                                      \tabularnewline
	se               & sami        &                                                    \tabularnewline
	sk               & slovak      &                                                    \tabularnewline
	sl               & slovenian   &                                                    \tabularnewline
	sq               & albanian    &                                                    \tabularnewline
	sr               & serbian     &                                                    \tabularnewline
	sr-Cyrl          & serbian     & script=Cyrillic                                    \tabularnewline
	sr-Latn          & serbian     & script=Latin [\emph{default}]                      \tabularnewline
	sv               & swedish     &                                                    \tabularnewline
	syr              & syriac      &                                                    \tabularnewline
	ta               & tamil       &                                                    \tabularnewline
	te               & telugu      &                                                    \tabularnewline
	th               & thai        &                                                    \tabularnewline
	tk               & turkmen     &                                                    \tabularnewline
	tr               & turkish     &                                                    \tabularnewline
	ug               & uyghur      &                                                    \tabularnewline
	uk               & ukrainian   &                                                    \tabularnewline
	ur               & urdu        &                                                    \tabularnewline
	vi               & vietnamese  &                                                    \tabularnewline
	zh               & chinese     &                                                    \tabularnewline
	zh-CN            & chinese     & variant=simplified [\emph{default}]                \tabularnewline
	zh-TW            & chinese     & variant=traditional                                \tabularnewline
	zsm              & malay       & variant=malaysian [\emph{default}]                 \tabularnewline
	\bottomrule
\end{longtable}
\egroup

\subsection{Global options}

\pkg{Polyglossia} can be loaded with the following global package options:

\begin{itemize}
	\item \xpgboolkeyfalse[1.1.1]{babelshorthands}
		Globally activates \pkg{babel} shorthands whenever available. Such shorthands are
		implemented currently for Afrikaans, Belarusian, Catalan, Croatian,
		Czech, Dutch, Finnish, Georgian, German, Italian, Latin, Mongolian,
		Russian, Slovak, and Ukrainian. Please refer to the respective language descriptions
		(sec.~\ref{specific}) for details.

	\item \xpgboolkeyfalse{localmarks} redefines the internal \LaTeX\ macros \cmd\markboth\ and
		\cmd\markright\ to the effect that the header text is explicitly set in the currently
		active language (\ie wrapped into \cmd\foreignlanguage\{\meta{lang}\}\{\meta{\ldots}\}).
		
		In earlier versions of \pkg{polyglossia},\new{1.2.0} this
		option was enabled by default, but we now realize that it causes more problems
		than it helps (since it breaks if a package or class redefines \cmd\markboth\ or
		\cmd\markright), so it is now disabled by default. For backwards compatibility, the
		option \xpgoption{nolocalmarks} which used to switch off the previous default, and
		now equals the default, is still available.

    \item \xpgoptkey[1.50]{luatexrenderer}{renderer}[Harfbuzz] determines which font renderer is used
        with \LuaTeX\ output. The correct font renderer is essential particularly for non-Latin scripts.
        By default, \pkg{polyglossia} uses the \xpgvalue{Harfbuzz} renderer that has been introduced to
        \LuaTeX\ in 2019 (\TeX Live 2020), as this gives the best results generally. If you want to use
        a different renderer, you can specify this here (or individually for specific fonts via the optional
        argument of the font selection commands). Please refer to the \pkg{fontspec} manual for supported
        values and for details on how to change the renderer for individual fonts.\\
        \xpgoption{luatexrenderer=none} disables \pkg{polyglossia}'s automatic renderer setting.

	\item \xpgboolkeytrue{verbose} determines whether info messages and (some of the) warnings issued
		by \LaTeX, \pkg{fontspec} and \pkg{polyglossia} are output.
\end{itemize}

\section{Language-switching commands}\label{languageswitching}

\subsection{Recommended commands}\label{sec:langcmds}
For each activated language the command
\cmd{\text\meta{lang}\oarg{options}\marg{…}} \DescribeMacro{\text\meta{lang}}
(as well as the synonymous \DescribeMacro{\textlang}%
\cmd{\textlang\oarg{options}\marg{lang}\marg{…}}\new{1.46})
becomes available for short insertions of text in that language.

For example ¦\textrussian{\today}¦ and ¦\textlang{russian}{\today}¦ yield \textrussian{\today}
The commands switch to the correct hyphenation patterns, they activate
some extra features for the selected language (such as extra spacing before
punctuation in French), and they translate the date when using \cmd\today.
They do not, however, translate so-called \textit{caption strings}, \ie
``chapter'', ``figure'' etc., to the local language (these remain in the currently active `outer'
language).

The\DescribeEnv{\meta{lang}}\ environment \meta{lang}, which is also available for any activated language
(as well as the equivalent \DescribeMacro{lang}%
\cmd{\begin\{lang\}\oarg{options}\marg{lang}} \dots{} \cmd{\end\{lang\}}\new{1.47}),
is meant for longer passages of text. It behaves slightly different than the \cmd{\text\meta{lang}} and
\cmd\textlang\ commands: It does everything the latter do, but additionally, the caption strings
are translated as well, and the language is also passed to auxiliary files, the table of contents
and the lists of figures/tables.
Like the commands, the environment provides the possibility of setting language options locally.
For instance the following allows us to quote the beginning
of Homer’s \textit{Iliad}:

\begin{Verbatim}[formatcom=\color{xpgblue}]
  \begin{quote}
   \begin{greek}[variant=ancient]
     μῆνιν ἄειδε θεὰ Πηληϊάδεω Ἀχιλῆος οὐλομένην, ἣ μυρί' Ἀχαιοῖς
     ἄλγε' ἔθηκε, πολλὰς δ' ἰφθίμους ψυχὰς Ἄϊδι προί̈αψεν ἡρώων,
     αὐτοὺς δὲ ἑλώρια τεῦχε κύνεσσιν οἰωνοῖσί τε πᾶσι, Διὸς δ'
     ἐτελείετο βουλή, ἐξ οὗ δὴ τὰ πρῶτα διαστήτην ἐρίσαντε Ἀτρεί̈δης
     τε ἄναξ ἀνδρῶν καὶ δῖος Ἀχιλλεύς.
   \end{greek}
  \end{quote}
\end{Verbatim}

\vspace{-.5\baselineskip}

\begin{quote}
\begin{greek}[variant=ancient]
μῆνιν ἄειδε θεὰ Πηληϊάδεω Ἀχιλῆος οὐλομένην, ἣ μυρί' Ἀχαιοῖς ἄλγε' ἔθηκε,
πολλὰς δ' ἰφθίμους ψυχὰς Ἄϊδι προί̈αψεν ἡρώων, αὐτοὺς δὲ ἑλώρια τεῦχε κύνεσσιν
οἰωνοῖσί τε πᾶσι, Διὸς δ' ἐτελείετο βουλή, ἐξ οὗ δὴ τὰ πρῶτα διαστήτην ἐρίσαντε
Ἀτρεί̈δης τε ἄναξ ἀνδρῶν καὶ δῖος Ἀχιλλεύς.
\end{greek}
\end{quote}

\noindent\DescribeEnv{Arabic} Note that for Arabic one cannot use the environment ¦arabic¦,
as \cmd\arabic\ is defined internally by \LaTeX. In this case
we need to use the environment ¦Arabic¦ instead.

\subsection{Babel commands}\label{sec:babelcmds}
Some macros defined in \pkg{babel}’s \file{hyphen.cfg} (and thus usually
compiled into the \XeLaTeX\ and \LuaLaTeX\ format) are redefined, but keep a
similar behaviour.
\begin{itemize}
\item \DescribeMacro{\selectlanguage}\cmd{\selectlanguage\oarg{options}\marg{lang}}
\item \DescribeMacro{\foreignlanguage}\cmd{\foreignlanguage\oarg{options}\marg{lang}\marg{…}}
\item \DescribeEnv{otherlanguage}\cmd{\begin\{otherlanguage\}\oarg{options}\marg{lang}} \dots{} \cmd{\end\{otherlanguage\}}
\item \DescribeEnv{otherlanguage*}\cmd{\begin\{otherlanguage*\}\oarg{options}\marg{lang}} \dots{} \cmd{\end\{otherlanguage*\}}
\item \DescribeEnv{hyphenrules}\cmd{\begin\{hyphenrules\}\oarg{options}\marg{lang}} \dots{} \cmd{\end\{hyphenrules\}}\new{1.50}
\end{itemize}
%
\cmd\selectlanguage\marg{lang} and the ¦otherlanguage¦ environment are identical to the
\meta{lang} environment, except that \cmd\selectlanguage\marg{lang}
does not need to be explicitly closed. The command \cmd\foreinlanguage\marg{lang}\marg{…} and the ¦otherlanguage*¦
environment are identical with the use of the \cmd\text\meta{lang} or \cmd\textlang\ command, with the one
notable exception that they do not translate the date with \cmd\today.

The \meta{hyphenrules} environment only switches the hyphenation patterns to the one associated with the language \meta{lang}
(or the language variety as specified via \meta{options}). It does no further language-specific change.

Since the \XeLaTeX\ and \LuaLaTeX\ format incorporate \pkg{babel}’s \file{hyphen.cfg},
the low-level commands for hyphenation and language switching defined there are in principal also accessible.
Note, however, that the availability of such low-level commands is not guaranteed, as \file{hyphen.cfg}, which is
out of \pkg{polyglossia}'s control, is (or at least has been) subject to regular change.

\subsection{Other commands}
The following commands are probably of lesser interest to the end user, but
ought to be mentioned here.
\begin{itemize}
\item \DescribeMacro{\selectbackgroundlanguage}\cmd{\selectbackgroundlanguage\marg{lang}}:
    this selects the global font setup and the numbering definitions for the default language.

\item \DescribeMacro{\resetdefaultlanguage}\cmd{\resetdefaultlanguage\oarg{options}\marg{lang}} (experimental):
	Switches the default language
	to another one in the middle of a document: \textit{this may have adverse effects}!

\item \Cmd\normalfontlatin: in an environment where \cmd\normalfont\ has been redefined
	to a non-latin script, this will reset to the font defined with \cmd\setmainfont\ etc.
	In a similar vein, it is possible to use \Cmd\rmfamilylatin, \Cmd\sffamilylatin,
	and \Cmd\ttfamilylatin.

\item \Cmd\latinalph: Representation of counter as a lower-case letter:  1 = a, 2 = b, etc.

\item \Cmd\latinAlph: Representation of counter as a upper-case letter:  1 = A, 2 = B, etc.
\end{itemize}


\subsection{Setting up alias commands}

By means of the macro\new{1.46}
\displaycmd{\setlanguagealias\oarg{options}\marg{language}\marg{alias}}{\setlanguagealias}
you can define alias commands for specific language (variants). \Eg

\begin{quote}
\begin{verbatim}
\setlanguagealias[variant=austrian]{german}{AT}
\end{verbatim}
\end{quote}
%
will define a command \cmd{\textAT} as well as an environment ¦{AT}¦ which will link towards
the command ¦\textgerman[variant=austrian]¦ and the environment ¦{german}[variant=austrian]¦,
respectively. The aliases can also be used in the language switching commands described in
section~\ref{sec:langcmds} and \ref{sec:babelcmds}.
Note, though, that the usual restrictions for command names apply, so something such as
¦de-AT¦ or ¦de_AT¦ will not work since ¦-¦ and ¦_¦ are not allowed in command names (the same
holds true for any non-ASCII character and for digits).

For the latter case, and for the case where an alias would clash with an existing command
(\eg ¦\fi¦) or a \cmd{\text\meta{\ldots}} command (\eg \cmd\textit), a starred version \Cmd{\setlanguagealias*}
is provided which does neither define a \cmd{\text\meta{alias}} command nor an \meta{alias} environment,
but which will set up the alias for everything else, including \cmd{\textlang\marg{alias}}
and \cmd{\begin\{lang\}\marg{alias}}.

\pkg{Polyglossia} comes with some aliases predefined, namely aliases for \pkg{babel} language names
(see sec.~\ref{sec:babelnames}) and for IETF BCP-47 language tags (the latter via \cmd{\setlanguagealias*};
see sec.~\ref{sec:langtags}).

\section{Font setup}

With polyglossia it is possible to associate a specific font with any script or language
that occurs in the document. That font should always be defined as
\cmd{\⟨script⟩font}\ or \cmd{\⟨language⟩font}.
For instance, if the default font defined by \cmd\setmainfont\
does not support Greek, then one can define the font used to display Greek with:\\
\centerline{ \cmd\newfontfamily\cmd{\greekfont[Script=Greek,\meta{…}]\marg{font}}. }
Note that polyglossia will use the font defined as is, so assure to do all necessary settings
(please refer to the \pkg{fontspec} documentation for details).
For instance, if \cmd\arabicfont\ is explicitly defined, then the option ¦Script=Arabic¦ should
be included in that definition.

If a specific sans serif or monospace (`teletype') font is needed for a particular script or language,
it can be defined by means of \new{1.2.0}
\cmd{\⟨script⟩fontsf} or \cmd{\⟨language⟩fontsf} and \cmd{\⟨script⟩fonttt} or \cmd{\⟨language⟩fonttt}, respectively.

Whenever a new language is activated, \pkg{polyglossia} will first check whether
a font has been defined for that language or – for languages in non-Latin scripts –
for the script it uses. If it is not defined, it will use the currently active font
and – in the case of OpenType fonts – will attempt to turn on the appropriate
OpenType tags for the script and language used, in case these are available in
the font, by means of \pkg{fontspec}’s \cmd\addfontfeature. If the current font
does not appear to support the script of that language, an error message is
displayed.

\section{Adapting hyphenation}

\subsection{Hyphenation exceptions}

\TeX\ provides the command \cmd\hyphenation\marg{exceptions} to globally define hyphenation exceptions
which override the hyphenation patterns for specified words. The command takes as argument a space-separated
list of words where hyphenation points are marked by a dash (if no dash is used, the respective word is not
hyphenated at all):
\begin{quote}
\begin{minipage}{.7\textwidth}
\begin{verbatim}
\hyphenation{%
  po-ly-glos-sia
  LaTeX
}
\end{verbatim}
\end{minipage}
\end{quote}
%
These exceptions, however, apply to all languages. In addition to this, \pkg{polyglossia} provides
the command\new{1.45}
\displaycmd{\pghyphenation\oarg{options}\marg{lang}\marg{exceptions}}{\pghyphenation}
which can be used to define exceptions that only apply to a specific language or language variant,
respectively.

\subsection{Hyphenation thresholds}

\pkg{Polyglossia} sets reasonable defaults for the hyphenation thresholds of each language,
\ie the number of characters that must  at least be there at the beginning or end of a
word before it is hyphenated (\cmd\lefthyphenmin\ and \cmd\righthyphenmin\ in \TeX).
For instance, with English, this threshold is 2 at the beginning (`left') and 3 at the end (`right'),
so a word will not be hyphenated within the first two characters at the beginning and the last three
characters at the end.

To change this value, \pkg{polyglossia} provides the command\new{1.50}
\displaycmd{\setlanghyphenmins\oarg{options}\marg{lang}\marg{l}\marg{r}}{\setlanghyphenmins}
%
where \meta{lang} is to be replaced with the respective language name or alias, \meta{options}
can be used to delimit the modification to a particular variety (\eg via \texttt{variant} or \texttt{spelling}),
\meta{l} with the left threshold value (\eg \texttt{3}), and \meta{r} with the right
one (\eg \verb|\setlanghyphenmins[spelling=old]{german}{4}{4}|).
This setting can be changed repeatedly in the preamble and the document body.
It applies to all subsequent text in the respective language (variety), but only after the
next language switch.

\subsection{Hyphenation disabling}

In some very specific contexts (such as music score creation), \TeX\ hyphenation
is something to avoid completely as it may cause troubles.
\pkg{Polyglossia} provides two functions: \Cmd\disablehyphenation\ and \Cmd\enablehyphenation.
Note that if you select a new language while hyphenation is disabled, it will remain disabled.
If you re-enable it, the hyphenation patterns of the currently selected language
will be activated.

\section{Language-specific options and commands}\label{specific}

This section gives a list of all languages for which options and end-user
commands are defined. Note the following conventions:
\begin{itemize}
	\item The preset value of each option (\ie the setting that applies by default,
	if the option is not explicitly set) is given in \xpgpresetvalue{italics}.
	\item If an option key may be used without a value, the value that applies
	for value-less keys is marked by a preceding \xpgdefaultvalue{asterisk}.
\end{itemize}
For instance, \texttt{babelshorthands = *true} or \texttt{\textit{false}} means that
\xpgvalue{babelshorthands} is \xpgvalue{false} by default in the respective language,
and that passing \xpgvalue{babelshorthands} (without value) is equivalent to passing
\xpgvalue{babelshorthands=true}.

\subsection{afrikaans}\label{afrikaans}
\paragraph*{Options:}
\begin{itemize}
	\item \xpgboolkeyfalse[1.1.1]{babelshorthands}
	If this is turned on, the following shorthands defined for fine-tuning hyphenation and
	micro-typography of Afrikaans words are activated:
	\begin{shorthands}
		\item[¦"-¦] adds a hyphenation point that does still allow for hyphenation at the points preset
		in the hyphenation patterns (as opposed to \cmd\- in default \TeX).
		\item[\texttt{"\textasciitilde}] for a hyphen sign without a breakpoint. Useful for
		cases where the hyphen should stick at the following syllable.
		\item[¦"|¦] disables a ligature at this position.
		\item[¦""¦] allows for a line break at this position (without hyphenation sign).
		\item[¦"/¦] a slash that allows for a subsequent line break. As opposed to \cmd\slash,
		hyphenation at the breakpoints preset in the hyphenation patterns is still allowed.
	\end{shorthands}
\end{itemize}

\subsection{arabic}\label{arabic}
\paragraph*{Options:}
	\begin{itemize}
	\item \xpgchoicekey{calendar}{\xpgpresetvalue{gregorian} or \xpgvalue{islamic} (= \xpgvalue{hijri})}
	\item \xpgoptkey{hijricorrection}{integer}[0]
	    If \xpgvalue{calendar=hijri} or \xpgvalue{calendar=islamic} is selected, \cmd{\today} will output the date
	    according to the lunar Islamic (Hijra) calendar. This option allows to shift the day of the output with respect
	    to the current date. Both positive and negative integer values are allowed (negative decrease the day value).
	    This corresponds to the optional argument of the \cmd{\Hijritoday} command which can be used to output Hijri
	    dates irrespective of the \xpgvalue{calendar} option (see sec.~\ref{sec:hijrical}).
	\item \xpgchoicekey{locale}{\xpgpresetvalue{default}, \xpgvalue{mashriq}, \xpgvalue{libya},
		\xpgvalue{algeria}, \xpgvalue{tunisia}, \xpgvalue{morocco}, \xpgvalue{mauritania}}
		This setting influences the spelling of the month names for the Gregorian calendar,
		as well as the form of the numerals (unless overridden by the following option). Recommended settings are
		\xpgvalue{default} for Egypt, Sudan, Yemen and the Gulf states, and \xpgvalue{mashriq} for Iraq, Syria,
		Jordan, Lebanon and Palestine.
	\item \xpgchoicekey{numerals}{\xpgpresetvalue{mashriq} (=~\xpgvalue{eastern}\new{1.63}) or \xpgvalue{maghrib}
		(=~\xpgvalue{western}\new{1.63})}
		Use Eastern Arabic (a.k.a.\ Indic-Arabic) numerals (\xpgvalue{mashriq} or \xpgvalue{eastern}) or Western numerals
		(\xpgvalue{maghrib} or \xpgvalue{western}).
		The latter is the default when \xpgvalue{locale=algeria}, \xpgvalue{tunisia}, or \xpgvalue{morocco}, the former
		is the default with all other Arabic locales.
	\item \xpgcodekey[1.63]{sectionsep}¦.¦
		This option allows to customize the separator between chapters, sections, and subsections (a dot by default),
		\eg \xpgoption{sectionsep=-}.
		This might be useful particularly with \xpgoption{numerals=mashriq} or \xpgoption{eastern}, as the dot looks
		too similar to the zero	in many fonts.
	\item \xpgboolkeyfalse[1.50]{abjadalph}
	     Set this to true if you want the alphabetic counters to be output using \cmd\abjadalph\ rather than \cmd\abjad.
	     Note that this limits the counter scope to 28 (see \cmd\abjadalph\ below).
	\item \xpgboolkeyfalse[1.0.3]{abjadjimnotail}
	    Set this to true if you want the \textit{abjad} form of the number three to be \textarabic{ج‍} – as in the manuscript
	    tradition – instead of the modern usage \textarabic{ج}.
	\end{itemize}
\paragraph*{Commands:}
	\begin{itemize}
	\item \Cmd\abjad outputs Arabic \textit{abjad} numbers according to the Mashriq varieties.
	      Example: ¦\abjad{1863}¦ yields \textarabic{\abjad{1863}}.
	\item \Cmd\abjadmaghribi outputs Arabic \textit{abjad} numbers according to the Maghrib varieties.
	       Example: ¦\abjadmaghribi{1863}¦ yields \textarabic{\abjadmaghribi{1863}}.
	\item \Cmd\abjadalph\new{1.50} steps through the Arabic alphabet, thus it can only be used up to 28.
	       Example: ¦\textarabic{\abjadalph{18}}¦ yields \textarabic{\abjadalph{18}}.
    \item \Cmd\aemph to emphasize text with \cmd\overline.\new{1.2.0}
          ¦\textarabic{\aemph{اب}}¦ yields \textarabic{\aemph{اب}}.
           This command is also available for Farsi, Urdu, etc.
	\end{itemize}

\subsection{armenian}\label{armenian}
\paragraph*{Options:}
\begin{itemize}
	\item \xpgchoicekey[1.45]{variant}{\xpgvalue{eastern} or \xpgpresetvalue{western}}
	\item \xpgchoicekey[1.45]{numerals}{\xpgvalue{armenian} or \xpgpresetvalue{arabic}}
	\item \xpgboolkeyfalse[1.63]{capitalyiwn}
		If this is set to true, \cmd{\MakeUppercase} will upcase the ligature of letters ech and yiwn to capital
		ech and yiwn. This conforms to the output outside Armenian language selection.
		By default (and with this option set to \xpgvalue{false}), \cmd{\MakeUppercase} upcases the ech and yiwn
		ligature to capital ech and \emph{vew}, adhering to Armenian reformed spelling regulations.
		Note that this feature requires	\LaTeX\ kernel 2023/06 at least.
\end{itemize}

\subsection[belarusian]{belarusian\new{1.46}}\label{belarusian}

\paragraph*{Options:}
\begin{itemize}
  \item \xpgboolkeyfalse{babelshorthands}
	If this is turned on, the following shorthands are activated:
	\begin{shorthands}
		\item[¦"-¦] adds a hyphenation point that does still allow for hyphenation at the points preset
		    in the hyphenation patterns (as opposed to \cmd\-).
		\item[¦"=¦] adds an explicit hyphen with a breakpoint, allowing for hyphenation at the
		    other points preset in the hyphenation patterns (as opposed to plain ¦-¦).
		\item[\texttt{"\textasciitilde}] for a hyphen sign without a breakpoint. Useful for
		    cases where the hyphen should stick at the following syllable.
		\item[¦"|¦] disables a ligature at this position.
		\item[¦""¦] allows for a line break at this position (without hyphenation sign).
		\item[¦",¦] thinspace for initials with a breakpoint in following surname.
		\item[¦"‘¦] for German left double quotes (looks like ,,).
		\item[¦"’¦] for German right double quotes (looks like “).
		\item[¦"<¦] for French left double quotes (looks like <<).
		\item[¦">¦] for French right double quotes (looks like >>).
	\end{shorthands}

	There are also three shorthands for the Cyrillic dash (\textrussian{тире}), which is shorter than the
	emdash but longer than the endash (namely 0.8\,em).
	Note that, since it is not covered by unicode, this character is faked by telescoping two endashes:
	\begin{shorthands}
		\item[¦"---¦] Cyrillic dash for the use in normal text. This requires preceding space
		       in input (trailing space is optional) and prints with a non-breakable thin space before
	           and after the dash.
		\item[¦"--\textasciitilde¦] Cyrillic dash for the use in compound names (surnames).
		       As opposed to ¦"---¦ this removes any space before and after the dash. 
		\item[¦"--*¦] Cyrillic dash for denoting direct speech. This adds a larger space after
		       the dash. Space before the dash is output as is.
	\end{shorthands}
    \item \xpgchoicekey{numerals}{\xpgpresetvalue{arabic}, \xpgvalue{cyrillic-alph} or \xpgvalue{cyrillic-trad}}
         Uses either Arabic numerals or Cyrillic alphanumerical numbering. The two Cyrillic variants differ as follows:
         \begin{itemize}
	          \item \xpgvalue{cyrillic-alph} steps through the Cyrillic alphabet. Thus it can only be used up to 30.
	          \item \xpgvalue{cyrillic-trad} (= \xpgvalue{cyrillic}) uses a traditional Cyrillic alphanumeric system.%
	                \footnote{See \url{https://en.wikipedia.org/wiki/Cyrillic_numerals}.}
	                It supports numbers up to 999\,999.
         \end{itemize}
	\item \xpgchoicekey{spelling}{\xpgpresetvalue{modern} or \xpgvalue{classic} (= \xpgvalue{tarask})}
	With ¦spelling=classic¦, captions and dates adhere to the Taraškievica (or
	Belarusian classical) orthography rather than the standard orthography.
\end{itemize}

\condbreak{\baselineskip}
\paragraph*{Commands:}
\begin{itemize}
     \item \Cmd\Asbuk: produces uppercased Cyrillic alphanumerals, for environments such as ¦enumerate¦.
          It steps through the Cyrillic alphabet and thus it can only be used up to 30.
          The command takes a counter as argument, \eg ¦\textbelarusian{\Asbuk{page}}¦ produces \textrussian{\Asbuk{page}}.
     \item \Cmd\asbuk: same as \cmd\Asbuk\ but in lowercase.
     \item \Cmd\AsbukTrad: same as \cmd\Asbuk\ but using the traditional Cyrillic alphanumeric numbering which supports
          numbers up to 999\,999.\\
          \Eg ¦\textbelarusian{\AsbukTrad{page}}¦ produces \textrussian{\AsbukTrad{page}}.
     \item \Cmd\asbukTrad: same as \cmd\AsbukTrad\ but in lowercase.
\end{itemize}

\subsection[bengali]{bengali\new{1.2.0}}\label{bengali}
\paragraph*{Options:}
	\begin{itemize}
	  \item \xpgchoicekey{numerals}{\xpgvalue{western}, \xpgvalue{bengali}, or \xpgpresetvalue{devanagari}}
		\item \xpgboolkeyfalse{changecounternumbering}
		Use specified numerals for headings and page numbers.
	\end{itemize}

\subsection{catalan}\label{catalan}
\paragraph*{Options:}
\begin{itemize}
  \item \xpgboolkeyfalse[1.1.1]{babelshorthands}
    Activates the shorthands \texttt{"l} and \texttt{"L} to type geminated l or L.
\end{itemize}

\paragraph*{Commands:}
\begin{itemize}
  \item \Cmd{\l.l} and \Cmd{\L.L}\new{1.1.1} can be used to type a geminated l, as in \textit{co\l.laborar},
        similar to \pkg{babel} (the glyph U+00B7 MIDDLE DOT is used as a geminating sign).
\end{itemize}

\subsection[chinese]{chinese\new{1.59}}\label{chinese}
\paragraph*{Options:}
	\begin{itemize}
	\item \xpgchoicekey{variant}{\xpgpresetvalue{simplified} (= \xpgvalue{cn}), or \xpgvalue{traditional} (= \xpgvalue{tw})}
	\item \xpgchoicekey{numerals}{\xpgpresetvalue{arabic} or \xpgvalue{chinese}}
		Uses either Arabic numerals or Chinese ideographic numbering.
	\end{itemize}
\paragraph*{Commands:}
	\begin{itemize}
	\item \Cmd\chinesenumeral (see section \ref{abjad}).
	\end{itemize}

\subsection{croatian}\label{croatian}
\paragraph*{Options:}
\begin{itemize}
	\item \xpgboolkeyfalse[1.47]{babelshorthands}
	If this is turned on, the following shorthands for fine-tuning hyphenation and micro-typography
	of Croatian words are activated.
	\begin{shorthands}
		\item[¦"|¦] disables a ligature at this position.
		\item[¦"=¦] for an explicit hyphen with a breakpoint, allowing for hyphenation at the
		            other points preset in the hyphenation patterns (as opposed to plain ¦-¦).
		\item[\texttt{"\textasciitilde}] for a hyphen sign without a breakpoint. Useful for
		            cases where the hyphen should stick at the following syllable.
		\item[¦"-¦] adds a hyphenation point that does still allow for hyphenation at the points preset
		            in the hyphenation patterns (as opposed to \cmd\-).
		\item[¦""¦] allows for a line break at this position (without hyphenation sign).
		\item[¦"/¦] a slash that allows for a subsequent line break. As opposed to \cmd\slash, hyphenation
		            at the breakpoints preset in the hyphenation patterns is still allowed.
	\end{shorthands}
	Furthermore, the following shorthands generate easy access to Croatian digraphs:
	\begin{shorthands}
		\item[¦"dz¦] Generates the digraph \charifavailable{01C6}{dž}\ if the font provides it. If not,
		             the two consecutive glyphs are output mimicking the digraph. Also available for
		             ¦"Dz¦ (\charifavailable{01C5}{Dž}) and ¦"DZ¦ (\charifavailable{01C4}{DŽ}).
		\item[¦"lj¦] Generates the digraph \charifavailable{01C9}{lj}\ if the font provides it. If not,
		             the two consecutive glyphs are output mimicking the digraph. Also available for
		             ¦"Lj¦ (\charifavailable{01C8}{Lj}) and ¦"LJ¦ (\charifavailable{01C7}{LJ}).
		\item[¦"nj¦] Generates the digraph \charifavailable{01CC}{nj}\ if the font provides it. If not,
		             the two consecutive glyphs are output mimicking the digraph. Also available for
		             ¦"Nj¦ (\charifavailable{01CB}{Nj}) and ¦"NJ¦ (\charifavailable{01CA}{NJ}).
	\end{shorthands}
	
	Finally, there are also four shorthands for quotation marks:
	\begin{shorthands}
		\item[¦"`¦] for Croatian left double quotes („).
		\item[¦"'¦] for Croatian right double quotes (”).
		\item[¦">¦] for Croatian left guillemets (»).
		\item[¦"<¦] for Croatian right guillemets («).
	\end{shorthands}
	\item \xpgboolkeyfalse[1.47]{disabledigraphs}[Up to version 1.53, the option was
		called \xpgvalue{disableligatures}. The old option is kept for backwards compatibility,
		but the use is discouraged.] If this is \xpgvalue{true}, all Croatian digraphs
		(such as \charifavailable{01C6}{dž}) will be replaced by the two consecutive letters,
		which is the most common way of typesetting them in Croatian. This can be useful if
		the Unicode digraphs in your font are broken (if the font does not have them,
		they are automatically mimicked by the two consecutive glyphs).
	\item \xpgboolkeytrue[1.53]{localalph}
		If \xpgvalue{true}, alphanumeric counters will use a locally established version which excludes
		the characters \meta{q}, \meta{w}, \meta{x} and \meta{y} from alphabetic counting.
		Obviously this limits the counting range to 22.
    \item \xpgboolkeytrue[1.51]{splithyphens}
	    According to Croatian typesetting conventions, if a word with a hard hyphen (such as
	    \emph{hrvatsko-engleski}) is hyphenated at this hyphen, a second hyphenation character is to be
	    inserted at the beginning of the line that follows the hyphenation (\emph{hrvatsko-/-engleski}).
	    By default, this is done automatically (if you are using \LuaTeX, the \pkg{luavlna} package is
	    loaded to achieve this).
	    Set this option to ¦false¦ to disable the feature.
\end{itemize}

\subsection{czech}\label{czech}

\paragraph*{Options:}
\begin{itemize}
	\item \xpgboolkeyfalse[1.45]{babelshorthands}
	If this is turned on, the following shorthands for Czech are activated:
	\begin{shorthands}
		\item[¦"=¦] for an explicit hyphen sign which is repeated at the beginning
		            of the next line when hyphenated, as common in Czech typesetting
		            (only needed with ¦splithyphens=false¦).
		\item[¦"‘¦] for Czech left double quotes („).
		\item[¦"’¦] for Czech right double quotes (“).
		\item[¦">¦] for Czech left double guillemets (»).
		\item[¦"<¦] for Czech right double guillemets («).
	\end{shorthands}
	\item \xpgboolkeytrue[1.45]{splithyphens}
	      According to Czech typesetting conventions, if a word with a hard hyphen (such as \emph{je-li})
	      is hyphenated at this hyphen, a second hyphenation character is to be inserted at the beginning
	      of the line that follows the hyphenation (\emph{je-/-li}).
          By default, this is done automatically\new{1.46} (if you are using \LuaTeX, the \pkg{luavlna} package is
          loaded to achieve this).
          Set this option to ¦false¦ to disable the feature.
	\item \xpgboolkeytrue[1.45]{vlna}
         According to Czech typesetting conventions, single-letter words (non-syllable prepositions)
         must not occur at line ends.
         \pkg{Polyglossia} takes care of this automatically by default\new{1.46} (if you are using \LuaTeX, the
         \pkg{luavlna} package is loaded to achieve this).
         Set this option to ¦false¦ to disable the feature.
\end{itemize}

\subsection{dutch}\label{dutch}
\paragraph*{Options:}
\begin{itemize}
  \item \xpgboolkeyfalse[1.1.1]{babelshorthands}
		If this is turned on, the following shorthands defined for fine-tuning hyphenation and
		micro-typography of Dutch words are activated:
		\begin{shorthands}
		\item[¦"-¦] adds a hyphenation point that does still allow for hyphenation at the points preset
			        in the hyphenation patterns (as opposed to \cmd\- in default \TeX).
		\item[\texttt{"\textasciitilde}] for a hyphen sign without a breakpoint. Useful for
			        cases where the hyphen should stick at the following syllable.
		\item[¦"|¦] disables a ligature at this position.
		\item[¦""¦] allows for a line break at this position (without hyphenation sign).
		\item[¦"/¦] a slash that allows for a subsequent line break. As opposed to \cmd\slash,
		            hyphenation at the breakpoints preset in the hyphenation patterns is still allowed.
		\end{shorthands}
        In addition, the macro \Cmd\- is redefined to allow hyphens in the rest of the word (equivalent to ¦"-¦).
\end{itemize}

\subsection{english}\label{english}
\paragraph*{Options:}
	\begin{itemize}
	\item \xpgchoicekey{variant}{\xpgpresetvalue{american} (= \xpgvalue{us}), \xpgvalue{usmax} (same as \xpgvalue{american} but with additional hyphenation patterns),
	\xpgvalue{british} (= \xpgvalue{uk}), \xpgvalue{australian}, \xpgvalue{canadian}\new{1.45}, or \xpgvalue{newzealand}}
	\item \xpgboolkeyfalse{ordinalmonthday}
		The default value is true for \xpgvalue{variant=british}.
	\end{itemize}

\subsection{esperanto}\label{esperanto}
\paragraph*{Commands:}
	\begin{itemize}
	\item \Cmd\hodiau\ and \Cmd\hodiaun are special forms of \cmd\today. The former produces the date in Esperanto
	      preceded by the article (\emph{la}), which is the most common date format.
	      The latter produces the same date format in accusative case.
	\end{itemize}

\subsection{finnish}\label{finnish}
\paragraph*{Options:}
\begin{itemize}
  \item \xpgboolkeyfalse[1.45]{babelshorthands}
		If this is turned on, the following shorthands for fine-tuning hyphenation
		and micro-typography of Finnish words are activated:
		\begin{shorthands}
		\item[¦"-¦] adds a hyphenation point that does still allow for hyphenation at the points preset
                    in the hyphenation patterns (as opposed to \cmd\-).
        \item[\texttt{"\textasciitilde}] for a hyphen sign without a breakpoint. Useful for
                    cases where the hyphen should stick at the following syllable.
        \item[¦"|¦] disables a ligature at this position.
        \item[¦""¦] allows for a line break at this position (without hyphenation sign).
        \item[¦"/¦] a slash that allows for a subsequent line break. As opposed to \cmd\slash,
                  hyphenation at the breakpoints preset in the hyphenation patterns is still allowed.
		\end{shorthands}
  \item \xpgboolkeyfalse[1.54]{schoolhyphens}
	    If this is set to true, alternative (experimental) hyphenation patterns are used rather than the
	    default Finnish patterns. These (so-called `School') patterns allow hyphenation at
	    all valid points, including hyphenation points that are considered sub-optimal from an
	    ortho-typographic point of view (e.\,g., hyphenation between two vowels).
	    The default patterns try to prevent such hyphenation, with the drawback that simple words
	    are sometimes not being hyphenated at points where this would be valid.
	    See \url{https://github.com/hyphenation/basic-finnish} for details.
\end{itemize}

\subsection{french}\label{french}
\paragraph*{Options:}
	\begin{itemize}
		\item \xpgchoicekey{variant}{\xpgpresetvalue{french}, \xpgvalue{canadian}, \xpgvalue{acadian}\new{1.45} or \xpgvalue{swiss}\new{1.47}}
			Currently, the only difference between the four variants is that \xpgvalue{swiss}
			uses \xpgvalue{thincolonspace=true} by default since this conforms to the Swiss
			conventions.
		\item \xpgboolkeytrue{autospacing}
			One of the most distinct features of French typography is the addition of
			extra spacing around punctuation and quotation marks (guillemets). By
			default, polyglossia adds these spaces automatically, so you don't need
			to enter them. This options allows you to switch this feature off
			globally.
		\item \xpgboolkeyfalse[1.46]{thincolonspace}
			With \xpgvalue{variant=swiss}, the default value is \xpgvalue{true}. If \xpgvalue{false}, a full
			(non-breaking) interword space is inserted before a colon. If \xpgvalue{true}, a
			thinner space -- as before \texttt{;}, \texttt{!}, and \texttt{?} -- is
			used. Note that this option must be set after the ¦variant¦ option.
		\item \xpgboolkeytrue{autospaceguillemets}[Up to version 1.44, the option was
		called \xpgvalue{automaticspacesaroundguillemets}. For backwards compatibility
		reasons, the more verbose old option is still supported.]
			If you only want to disable the automatic addition of spacing after
			opening and before closing guillemets (and not at punctuation), set this
			to \textit{false}. Note that the more general option \textit{autospacing}
			overrides this.
		\item \xpgboolkeyfalse[1.45]{autospacetypewriter}[Babel's syntax \xpgvalue{OriginalTypewriter}
		is also supported.]
			By default, automatic spacing is disabled in typewriter font. If this is
			enabled, spacing in typewriter context is the same as with roman and sans
			serif font, depending on the \xpgvalue{autospacing} and
			\xpgvalue{autospaceguillemets} settings (note that this was the default up
			to v.~1.44).
		\item \xpgboolkeyfalse{frenchfootnote}
			If \xpgvalue{true}, footnotes start with a non-superscripted number
			followed by a dot, as common in French typography. Note that this might
			interfere with the specific footnote handling of classes or packages.
			Also note that this option is only functional (by design) if French is
			the main language.
		\item \xpgboolkeyfalse[1.46]{frenchitemlabels}
			If \textit{true}, itemize item labels use em-dashes throughout, as common
			in French typography.  Note that this option is only functional (by
			design) if French is the main language. Also, it might interfere with
			list packages such as \pkg{enumitem}.
		\item \xpgboolkeytrue[1.51]{frenchpart}
			By default, \pkg{polyglossia} modifies part headings to match French conventions
			(\emph{Première partie} rather than \emph{Partie I}). Next to the standard classes,
			specifics of \pkg{KOMA-script}, \pkg{memoir} and the \pkg{titlesec} package are
			taken into account. With other classes or packages, redefinition might fail if these
			have particular part settings.
			In such case, or if you don't want the redefinition, you can switch off the feature
			by passing \textit{false} to this option.
		\item \xpgcodekey[1.46]{itemlabels}¦\textemdash¦
			If \emph{frenchitemlabels} is true, you can customize here the used item
			label of all levels.
		\item \xpgcodekey[1.46]{itemlabeli}¦\textemdash¦
			If \emph{frenchitemlabels} is true, you can customize here the used item
			label of the first level.
		\item \xpgcodekey[1.46]{itemlabelii}¦\textemdash¦
			If \emph{frenchitemlabels} is true, you can customize here the used item
			label of the second level.
		\item \xpgcodekey[1.46]{itemlabeliii}¦\textemdash¦
			If \emph{frenchitemlabels} is true, you can customize here the used item
			label of the third level.
		\item \xpgcodekey[1.46]{itemlabeliv}¦\textemdash¦
			If \emph{frenchitemlabels} is true, you can customize here the used item
			label of the fourth level.
	\end{itemize}
\paragraph*{Commands:}
\begin{itemize}
	\item \Cmd\NoAutoSpacing\new{1.45} disables automatic spacing around punctuation and quotation marks in all following text. The command can also be used locally if braces are used for grouping: ¦{\NoAutoSpacing foo:bar}¦
	\item \Cmd\AutoSpacing\new{1.45} enables automatic spacing around punctuation and quotation marks in all following text. The command can also be used locally if braces are used for grouping: ¦{\AutoSpacing regarde!}¦
\end{itemize}

\subsection[gaelic]{gaelic\new{1.45}}\label{gaelic}
\paragraph*{Options:}
\begin{itemize}
	\item \xpgchoicekey{variant}{\xpgpresetvalue{irish} or \xpgvalue{scottish}}
\end{itemize}

\subsection[georgian]{georgian\new{1.46}}\label{georgian}

\paragraph*{Options:}
\begin{itemize}
	\item \xpgboolkeyfalse{babelshorthands}
		If this is turned on, the following shorthands are activated:
		\begin{shorthands}
			\item[¦"-¦] adds a hyphenation point that does still allow for
			    hyphenation at the points preset in the hyphenation patterns (as opposed to \cmd\-).
		    \item[¦"=¦] adds an explicit hyphen with a breakpoint, allowing for hyphenation at the
			    other points preset in the hyphenation patterns (as opposed to plain ¦-¦).
			\item[\texttt{"\textasciitilde}] for a hyphen sign without a breakpoint. Useful for
			cases where the hyphen should stick at the following syllable.
			\item[¦"|¦] disables a ligature at this position.
			\item[¦""¦] allows for a line break at this position (without hyphenation sign).
			\item[¦",¦] thinspace for initials with a breakpoint in following surname.
			\item[¦"‘¦] for German-style left double quotes (looks like ,,).
			\item[¦"’¦] for German-style right double quotes (looks like “).
			\item[¦"<¦] for French-style left double quotes (looks like <<).
			\item[¦">¦] for French-style right double quotes (looks like >>).
	    \end{shorthands}
    
		There are also three shorthands for the Cyrillic dash (\textrussian{тире}), which is shorter than the
		emdash but longer than the endash (namely 0.8\,em).
		Note that, since it is not covered by unicode, this character is faked by telescoping two endashes:
		\begin{shorthands}
			\item[¦"---¦] Cyrillic dash for the use in normal text. This requires preceding space
			in input (trailing space is optional) and prints with a non-breakable thin space before
			and after the dash.
			\item[¦"--\textasciitilde¦] Cyrillic dash for the use in compound names (surnames).
			As opposed to ¦"---¦ this removes any space before and after the dash. 
			\item[¦"--*¦] Cyrillic dash for denoting direct speech. This adds a larger space after
			the dash. Space before the dash is output as is.
		\end{shorthands}
	\item \xpgchoicekey{numerals}{\xpgpresetvalue{arabic} or \xpgvalue{georgian}}
		Uses either Arabic numerals or Georgian alphanumerical numbering.
	\item \xpgboolkeyfalse{oldmonthnames}
		Uses traditional Georgian month names.
\end{itemize}

\subsection{german}\label{german}
\paragraph*{Options:}
	\begin{itemize}
	\item \xpgchoicekey{variant}{\xpgpresetvalue{german}, austrian, or swiss\new{1.33.4}}
		Setting \xpgvalue{variant=austrian} or \xpgvalue{variant=swiss} uses some lexical variants.
		With \xpgvalue{spelling=old}, \xpgvalue{variant=swiss} furthermore loads specific hyphenation
		patterns.
	\item \xpgchoicekey{spelling}{\xpgpresetvalue{new} (= \xpgvalue{1996}) or \xpgvalue{old} (= \xpgvalue{1901})}
		Indicates whether hyphenation patterns for traditional (1901) or reformed
		(1996) orthography should be used. The latter is the default.
	\item \xpgboolkeyfalse[1.63]{capitaleszett}
	     If this is set to true, \cmd{\MakeUppercase} will upcase the eszett character ⟨ß⟩ to the
	     capital counterpart ⟨\charifavailable{1E9E}{SS}⟩ that has been introduced to Unicode in 2008 and
	     standardized in German regulations in 2017, rather than ⟨SS⟩. Note that this feature requires
	     \LaTeX\ kernel 2023/06 at least and a font that contains the respective glyph.
	\item \xpgchoicekey[1.2.0]{script}{\xpgpresetvalue{latin} or \xpgvalue{blackletter}\new{1.46} (= \xpgvalue{fraktur})}
	     Setting ¦script=blackletter¦ adapts the captions for typesetting German in blackletter type, using the `long s' ⟨ſ\kern1.5pt⟩
	     where appropriate.
	\item \xpgboolkeyfalse[1.0.3]{babelshorthands}
		If this is turned on, all shorthands defined in \pkg{babel}
		for fine-tuning hyphenation and micro-typography of German words are activated.
		\begin{shorthands}
		\item[¦"ck¦] for ¦ck¦ to be hyphenated as ¦k-k¦ (1901 spelling).
		\item[¦"ff¦] for ¦ff¦ to be hyphenated as ¦ff-f¦ (1901 spelling); this is also available
		           for the letters l, m, n, p, r and t.
		\item[¦"|¦] disables a ligature at this position (\eg ¦Auf"|lage¦).
		\item[¦"=¦] for an explicit hyphen with a breakpoint, allowing for hyphenation at the
		           other points preset in the hyphenation patterns (as opposed to plain ¦-¦).
		\item[\texttt{"\textasciitilde}] for a hyphen sign without a breakpoint. Useful for
		           cases where the hyphen should stick at the following syllable, \eg ¦bergauf und "~ab¦.
		\item[¦"-¦] adds a hyphenation point that does still allow for hyphenation at the points preset
		           in the hyphenation patterns (as opposed to \cmd\-).
		\item[¦""¦] allows for a line break at this position (without hyphenation sign);
		           \eg ¦(pseudo"~)""wissenschaftlich¦.
		\item[¦"/¦] a slash that allows for a subsequent line break. As opposed to \cmd\slash, hyphenation at the breakpoints
		           preset in the hyphenation patterns is still allowed.
		\end{shorthands}

		There are also four shorthands for quotation signs:
		\begin{shorthands}
		\item[¦"`¦] for German-style left double quotes („)
		\item[¦"'¦] for German-style right double quotes (“)
		\item[¦"<¦] for French-style left double quotes («)
		\item[¦">¦] for French-style right double quotes (»).
		\end{shorthands}
	\end{itemize}

\subsection{greek}\label{greek}
\paragraph*{Options:}
	\begin{itemize}
	\item \xpgchoicekey{variant}{\xpgpresetvalue{monotonic} (= \xpgvalue{mono}), \xpgvalue{polytonic} (= \xpgvalue{poly}), or \xpgvalue{ancient}}
	\item \xpgchoicekey{numerals}{\xpgpresetvalue{greek} or \xpgvalue{arabic}}
	\item \xpgboolkeyfalse[1.63]{capitaliota}
		If this is set to true, \cmd{\MakeUppercase} will upcase the \emph{ypogegrammeni} (subscript muted iota) to capital iota.
		By default (and with this option set to \xpgvalue{false}), \cmd{\MakeUppercase} retains the subscript versions.
		Note that this feature requires	\LaTeX\ kernel 2023/06 at least.
	\end{itemize}
\paragraph*{Commands:}
	\begin{itemize}
	\item \Cmd\Greeknumber and \Cmd\greeknumber \ (see section \ref{abjad}).
	\item The command \Cmd\atticnumeral (= \Cmd\atticnum) (activated with
	  the option ¦attic=true¦), displays numbers using the acrophonic
          numbering system (defined in the Unicode range
	  \textsf{U+10140–U+10174}).\footnote{ %
	  	See the documentation of the \pkg{xgreek} package for more details.}
	\end{itemize}

\subsection{hebrew}\label{hebrew}
\paragraph*{Options:}
	\begin{itemize}
	\item \xpgchoicekey{numerals}{\xpgvalue{hebrew} or \xpgpresetvalue{arabic}}
	\item \xpgchoicekey{calendar}{\xpgvalue{hebrew} or \xpgpresetvalue{gregorian}}
	\item \xpgboolkeyfalse{marcheshvan}
		If \xpgvalue{true}, the second month of the civil year will be output as
		\texthebrew{מרחשון} (Marcheshvan) rather than \texthebrew{חשון} (Heshvan),
		which is the default.
	 \item \xpgboolkeyfalse[1.57]{fullyear} causes years from the current millennium to be printed with the thousands digit
	      (he-tav-shin-samekh-gimel). Without this option, thousands are not printed for the current millennium.
	\item \xpgchoicekey[1.57]{transliteration}{\xpgvalue{academy} or \xpgpresetvalue{alt}} With value \xpgvalue{academy}, transliteration
	follows	the recommendations of the Hebrew Language Academy. The default (\xpgvalue{alt}) uses the received settings of
	\pkg{babel} (\pkg{hebcal}) and	\pkg{polyglossia} (\pkg{hebrewcal}).
	\end{itemize}
\paragraph*{Commands:}
	\begin{itemize}
	\item \Cmd\hebrewnumeral\ (= \Cmd\hebrewalph) (see section \ref{abjad}).
  \item \Cmd\aemph (see section \ref{arabic}).
	\end{itemize}

\subsection[hindi]{hindi\new{1.2.0}}\label{hindi}
\paragraph*{Options:}
	\begin{itemize}
		\item \xpgchoicekey{numerals}{\xpgvalue{western} or \xpgpresetvalue{devanagari}}
	\end{itemize}

\subsection{hungarian}\label{hungarian}
\paragraph*{Options:}
\begin{itemize}
	\item \xpgchoicekey[1.46]{swapstrings}{\xpgdefaultvalue{\xpgpresetvalue{all}},
	\xpgvalue{captions}, \xpgvalue{headings}, \xpgvalue{headers}, \xpgvalue{hheaders}, or \xpgvalue{none}}
	       In Hungarian, some caption strings need to be in a different order than in other languages
	       (\eg \emph{1. fejezet} instead of \emph{Chapter 1}). By default, \pkg{polyglossia} tries hard to
	       provide the correct order for different classes and packages (standard classes, \pkg{KOMA-script},
	       \pkg{memoir}, and \pkg{titlesec} package should work, as well as \pkg{fancyhdr} and \pkg{caption}).
	       However, since the definition of these strings is not standardized, the redefinitions might not work
	       and even interfere badly if you use specific classes or packages that redefine the respective strings
	       themselves. In this case, you can disable some or all changes.
	       The possibilities are:
	       \begin{itemize}
	       	\item ¦all¦: Redefine figure and table captions, part and chapter headings, and running headers (=~default setting)
	       	\item ¦captions¦: Redefine figure and table captions only
	       	\item ¦headings¦: Redefine part and chapter headings only
	       	\item ¦headers¦: Redefine running headers only
	       	\item ¦hheaders¦: Redefine part and chapter headings as well as running headers
	       	\item ¦none¦: Do not redefine anything
	       \end{itemize}
    \item \xpgboolkeytrue[1.58]{forceheadingpunctuation}
        Section numbers always have a trailing punctuation in Hungarian (as in \emph{1.1.} as opposed to \emph{1.1}).
        For compatibility reasons, the default option is \xpgvalue{false}, thus \textsf{polyglossia}
        does not touch heading punctuation, so this will be whatever the class or a package determines.
        Set this option to \xpgvalue{true}, 
        and \textsf{polyglossia} appends a period after the section counters, 
        and adjusts the header punctuation (as in \emph{1. fejezet.} as opposed to \emph{1. fejezet}).
\end{itemize}
\paragraph*{Commands:}
\begin{itemize}
	\item \Cmd\ontoday\ (= \Cmd\ondatehungarian): special form of \cmd\today\ which produces a slightly different
	date format as used in prepositional phrases (such as ‘on February 10th’) in Hungarian.
\end{itemize}

\subsection{italian}\label{italian}
\paragraph*{Options:}
\begin{itemize}
  \item \xpgboolkeyfalse[1.2.0cc]{babelshorthands}% TODO: check version
  Activates the ¦"¦ character as a switch to perform etymological
  hyphenation when followed by a letter. Furthermore, the following shorthands are activated:
  \begin{shorthands}
  	\item[¦""¦] double raised open quotes (the Italian keyboard misses the backtick).
  	\item[¦"<¦] open guillemet (looks like <<).
  	\item[¦">¦] closing guillemet (looks like >>).
  	\item[¦"/¦] a slash that allows for a subsequent line break. As opposed to \cmd\slash,
               hyphenation at the breakpoints preset in the hyphenation patterns is still allowed.
  	\item[¦"-¦] adds a hyphenation point that does still allow for hyphenation at the points preset
               in the hyphenation patterns (as opposed to \cmd\-).
  \end{shorthands}
\end{itemize}

\subsection[korean]{korean\new{1.40.0}}\label{korean}
\paragraph*{Options:}
  \begin{itemize}
	\item \xpgchoicekey{variant}{\xpgpresetvalue{plain}, \xpgvalue{classic}, or \xpgvalue{modern}}
    These variants control spacing before/after CJK punctuations.
      \begin{itemize}
        \item ¦plain¦: Do nothing
        \item ¦classic¦: Suitable for text with no interword spaces.
          This option forces CJK punctuations to half-width, and
          inserts half-width glue around them.
        \item ¦modern¦: Suitable for text with interword spaces.
          This option forces CJK punctuations to half-width, and
          inserts small (half of interword space) glue around them.
      \end{itemize}
	\item \xpgchoicekey{captions}{\xpgpresetvalue{hangul} or \xpgvalue{hanja}}
	\item \xpgchoicekey[1.47]{swapstrings}{\xpgdefaultvalue{\xpgpresetvalue{all}},
	\xpgvalue{headers}, \xpgvalue{headings}, or \xpgvalue{none}}
    With this option, Korean-style part and chapter headings, and
    running headers are available.
    It is similar to Hungarian (see \ref{hungarian}) except that
    figure and table captions are not touched.
      \begin{itemize}
        \item ¦all¦: Redefine part and chapter headings, and running headers
          (=~default setting)
        \item ¦headings¦: Redefine part and chapter headings only
        \item ¦headers¦: Redefine running headers only
        \item ¦none¦: Do not redefine anything
      \end{itemize}
  \end{itemize}

\subsection[kurdish]{kurdish\new{1.45}}\label{kurdish}

\paragraph*{Options:}
\begin{itemize}
	\item \xpgchoicekey{variant}{\xpgvalue{kurmanji} or \xpgpresetvalue{sorani}}
	\item \xpgchoicekey{script}{\xpgvalue{arabic} or \xpgvalue{latin}}
		Defaults are \xpgvalue{arabic} for Sorani and \xpgvalue{latin} for Kurmanji.
	\item \xpgchoicekey{numerals}{\xpgvalue{western} or \xpgvalue{eastern}}
		Defaults are \xpgvalue{western} for Latin and \xpgvalue{eastern} for Arabic script, depending
		on the selection above.
	\item \xpgcodekey[1.63]{sectionsep}¦.¦
		This option allows to customize the separator between chapters, sections, and subsections (a dot by default),
		\eg \xpgoption{sectionsep=-}.
		This might be useful particularly with \xpgoption{numerals=eastern}, as the dot looks too similar to the zero
		in many fonts.
	\item \xpgboolkeyfalse{abjadjimnotail}
		Set this to true if you want the \textit{abjad} form of the number three to
		be \textarabic{ج‍} – as in the manuscript tradition – instead of the
		modern usage \textarabic{ج}.
%	\item \xpgchoicekey{locale}{} (not yet implemented)
%	\item \xpgchoicekey{calendar}{} (not yet implemented)
\end{itemize}
\condbreak{2\baselineskip}
\paragraph*{Commands:}
\begin{itemize}
	\item \Cmd\ontoday: special form of \cmd\today\ which produces a slightly different
	date format as used in prepositional phrases (as in ‘on February 10th’). Only available for Latin script.
	\item \Cmd\abjad (see section \ref{abjad})
	\item \Cmd\aemph (see section \ref{arabic})
\end{itemize}

\subsection[lao]{lao\new{1.2.0}}\label{lao}
\paragraph*{Options:}
	\begin{itemize}
	\item \xpgchoicekey{numerals}{\xpgvalue{lao} or \xpgpresetvalue{arabic}}
	\end{itemize}

\subsection{latin}\label{latin}
\paragraph*{Options:}
\begin{itemize}
	\item \xpgchoicekey{variant}{\xpgvalue{classic}, \xpgvalue{medieval}, \xpgpresetvalue{modern}, or \xpgvalue{ecclesiastic}\new{1.46}}
			These variants refer to different spelling conventions. The \xpgvalue{classic}
			and the \xpgvalue{medieval} variant do not use the letters \emph{U} and
			\emph{v}, but only \emph{V} and \emph{u}. This concerns predefined terms like
			month names as well as the behaviour of the \cmd\MakeUppercase\ and the
			\cmd\MakeLowercase\ command. The ¦medieval¦ and the
			¦ecclesiastic¦ variant use the ligatures \emph{\ae} and \emph{\oe}.
			See table \ref{tab:latin-spelling} for examples.

			Furthermore, the \xpgvalue{ecclesiastic} variant takes care for a punctuation
			spacing similar to French, but with smaller spaces, as provided for
			PDF\TeX\ by the \pkg{ecclesiastic} package.
			\begin{table}
			\centering
			\caption{\label{tab:latin-spelling}Spelling differences between the Latin
				language variants.\newline The capitalization of month names and the use of
				\emph{i/j} may be affected by the \xpgvalue{capitalizemonth} and the
				\xpgvalue{usej} option.}
			\begin{tabular}{llll}
			\toprule
			\textbf{classic} & \textbf{medieval} & \textbf{modern} & \textbf{ecclesiastic} \\
			\midrule
			Ianuarii & Ianuarii & Ianuarii & ianuarii \\
			Nouembris & Nouembris & Novembris & novembris \\
			Praefatio & Præfatio & Praefatio & Præfatio \\
			\addlinespace\multicolumn{4}{@{}l}{\cmd{\MakeUppercase\{Iulius\}} yields:} \\
			IVLIVS & IVLIVS & IULIUS & IULIUS \\
			\bottomrule
			\end{tabular}
			\end{table}
	\item \xpgchoicekey[1.46]{hyphenation}{\xpgvalue{classic}, \xpgvalue{modern}, or \xpgvalue{liturgical}}
			There are three different sets of hyphenation patterns for Latin. Separate
			documention for them is available on the
			Internet.\footnote{Refer to \url{https://github.com/gregorio-project/hyphen-la/blob/master/doc/README.md\#hyphenation-styles}}
			Each of the four variants mentioned above has its default set of hyphenation
			patterns as indicated by table \ref{tab:latin-hyphenation}. Use the
			¦hyphenation¦ option if the default style does not fit your needs.
			Note that the liturgical hyphenation patterns are the default of none of the
			language variants. To use them, you have to say
			\xpgvalue{hyphenation=liturgical} in any case.
			\begin{table}
			\caption{\label{tab:latin-hyphenation}Latin default hyphenation styles}
			\centering
			\begin{tabular}{ll}
			\toprule
			\textbf{Language variant} & \textbf{Default hyphenation style} \\
			\midrule
			classic & classic \\
			medieval & modern \\
			modern & modern \\
			ecclesiastic & modern \\
			\bottomrule
			\end{tabular}
			\end{table}
	\item \xpgboolkeyfalse[1.46]{ecclesiasticfootnotes}
			Use footnotes as provided by the \pkg{ecclesiastic} package, which typesets
			footnotes with ordinary instead of superior numbers and without indentation.
			As many ecclesiastic documents and liturgical books use footnotes that are
			very similar to the ordinary \LaTeX\ ones, we do not use this footnote style
			as default even for the ¦ecclesiastic¦ variant.
		
			Note that this option is only possible if Latin is the main language of your
			document.
	\item \xpgboolkeyfalse[1.46]{usej}
			Use \emph{J/j} in predefined terms. The letter \emph{j} is not of ancient
			origin. In early modern times, it was used to distinguish the consonantic
			\emph{i} from the vocalic~\emph{i}. Nowadays, the use of \emph{j} has
			disappeared from most Latin publications. So ¦false¦ is the default
			value for all four language variants. Use this option if you prefer
			\emph{Januarii} and \emph{Maji} to \emph{Ianuarii} and \emph{Maii}.
	\item \xpgboolkey[1.46]{capitalizemonth}
			Capitalize the month name when printing dates (using the \cmd\today\
			command).  Traditionally, month names are capitalized. However, in recent
			liturgical books they are lowercase. So ¦true¦ is the default value for
			the variants ¦classic¦, ¦medieval¦, and ¦modern¦,
			whereas ¦false¦ is the default value for the ¦ecclesiastic¦
			variant.
	\item \xpgboolkeyfalse{babelshorthands}
			Enable the following shorthands inherited from \pkg{babel-latin} and the
			\pkg{ecclesiastic} package.
			\begin{shorthands}
				\item[¦"<¦] for « (left guillemet)
				\item[¦">¦] for » (right guillemet)
				\item[¦"¦] If no other shorthand applies, ¦"¦ before any letter
					character defines an optional break point allowing further break points
					within the same word (as opposed to the \cmd\- command).
				\item[¦"|¦] the same as ¦"¦, but also possible before non-letter
					characters
				\item[¦'a¦] for á (a with acute), also available for é, í, ó, ú, ý, ǽ, and
					\'œ
				\item[¦'A¦] for Á (A with acute), also available for É, Í, Ó, Ú, \'V, Ý, Ǽ,
					and \'Œ
			\end{shorthands}
			The following shorthands are only available for the ¦medieval¦ and the
			¦ecclesiastic¦ variant.
			\begin{shorthands}
				\item[¦"ae¦] for æ (ae ligature), also available for œ
				\item[¦"Ae¦] for Æ (AE ligature), also available for Œ
				\item[¦"AE¦] for Æ (AE ligature), also available for Œ
				\item[¦'ae¦] for ǽ (ae ligature with acute), also available for \'œ
				\item[¦'Ae¦] for Ǽ (AE ligature with acute), also available for \'Œ
				\item[¦'AE¦] for Ǽ (AE ligature with acute), also available for \'Œ
			\end{shorthands}
	\item \xpgboolkeyfalse[1.46]{prosodicshorthands}
			Enable shorthands for prosodic marks (macrons and breves) very similiar to
			those provided by \pkg{babel-latin} using the ¦withprosodicmarks¦
			modifier.
			Note that the active ¦=¦ character used for macrons will cause problems with
			commands using \texttt{key=value} interfaces, such as the command
			¦\includegraphics[scale=2]{...}¦. Use \cmd{\shorthandoff\{=\}} before
			such commands (and \cmd{\shorthandon\{=\}} thereafter) within every
			environment with prosodic shorthands enabled.
		
			The following shorthands are available.
			\begin{shorthands}
				\item[¦=a¦] for ā (a with macron), also available for ē, ī, ō, ū, and ȳ
				\item[¦=A¦] for Ā (A with macron), also available for Ē, Ī, Ō, Ū, V̄, and Ȳ.
					Note that a macron above the letter V is only displayed if your font
					supports the Unicode character ¦0304¦ (\emph{combining macron}).
				\item[¦=ae¦] for a͞e (ae diphthong with macron, for the ¦classic¦ and
					the ¦modern¦ variant) or ǣ (ae ligature with macron, for the ¦medieval¦
					and the ¦ecclesiastic¦ variant), respectively; also available for a͞u,
					e͞u, and o͞e/œ̄. Note that macrons above diphthongs are only displayed if
					your font supports the Unicode character ¦035E¦ (\emph{combining double
					macron}).
				\item[¦=Ae¦] for A͞e (Ae diphthong with macron, for the ¦classic¦ and
					the ¦modern¦ variant) or Ǣ (AE ligature with macron, for the ¦medieval¦
					and the ¦ecclesiastic¦ variant), respectively; also available for A͞u,
					E͞u, and O͞e/Œ̄.
				\item[¦=AE¦] for A͞E (AE diphthong with macron, for the ¦classic¦ and
					the ¦modern¦ variant) or Ǣ (AE ligature with macron, for the ¦medieval¦
					and the ¦ecclesiastic¦ variant), respectively; also available for A͞U,
					E͞U, and O͞E/Œ̄.
				\item[¦\textasciicircum a¦] for ă (a with breve), also available for ĕ, ĭ,
					ŏ, ŭ, and y̆. Note that a breve above the letter y is only displayed if
					your font supports the Unicode character ¦0306¦ (\emph{combining
					breve}).
				\item[¦\textasciicircum A¦] Ă (A with breve), also available for Ĕ, Ĭ, Ŏ,
					Ŭ, V̆, and Y̆. Note that breves above the letters V and Y are only displayed
					if your font supports the Unicode character ¦0306¦ (\emph{combining
					breve}).
			\end{shorthands}
\end{itemize}

\subsection{malay}\label{malay}
\paragraph*{Options:}
\begin{itemize}
	\item \xpgchoicekey[1.45]{variant}{\xpgvalue{indonesian} or \xpgpresetvalue{malaysian}}
\end{itemize}

\subsection{marathi}\label{marathi}
\paragraph*{Options:}
\begin{itemize}
	\item \xpgchoicekey{numerals}{\xpgpresetvalue{devanagari} or \xpgvalue{western}}
\end{itemize}

\subsection[mongolian]{mongolian\new{1.45}}\label{mongolian}
Currently, only the Khalkha variety in Cyrillic script is supported.

\paragraph*{Options:}
\begin{itemize}
	\item \xpgboolkeyfalse{babelshorthands}
	If this is turned on, the following shorthands are activated:
	\begin{shorthands}
        \item[¦"-¦] adds a hyphenation point that does still allow for hyphenation at the points preset
		           in the hyphenation patterns (as opposed to \cmd\-).
		\item[¦"=¦] adds an explicit hyphen with a breakpoint, allowing for hyphenation at the
		           other points preset in the hyphenation patterns (as opposed to plain ¦-¦).
		\item[\texttt{"\textasciitilde}] for a hyphen sign without a breakpoint. Useful for
		           cases where the hyphen should stick at the following syllable.
		\item[¦"|¦] disables a ligature at this position.
		\item[¦""¦] allows for a line break at this position (without hyphenation sign).
		\item[¦",¦] thinspace for initials with a breakpoint in following surname.
		\item[¦"‘¦] for German-style left double quotes (looks like ,,).
		\item[¦"’¦] for German-style right double quotes (looks like “).
		\item[¦"<¦] for French-style left double quotes (looks like <<).
		\item[¦">¦] for French-style right double quotes (looks like >>).
    \end{shorthands}

	There are also three shorthands for the Cyrillic dash (\textrussian{тире}), which is shorter than the
	emdash but longer than the endash (namely 0.8\,em).
	Note that, since it is not covered by unicode, this character is faked by telescoping two endashes:
	\begin{shorthands}
		\item[¦"---¦] Cyrillic dash for the use in normal text. This requires preceding space
		in input (trailing space is optional) and prints with a non-breakable thin space before
		and after the dash.
		\item[¦"--\textasciitilde¦] Cyrillic dash for the use in compound names (surnames).
		As opposed to ¦"---¦ this removes any space before and after the dash. 
		\item[¦"--*¦] Cyrillic dash for denoting direct speech. This adds a larger space after
		the dash. Space before the dash is output as is.
	\end{shorthands}
	\item \xpgchoicekey{numerals}{\xpgpresetvalue{arabic}, \xpgvalue{cyrillic-alph} or \xpgvalue{cyrillic-trad}}
          Uses either Arabic numerals or Cyrillic alphanumerical numbering. The two Cyrillic variants differ as follows:
          \begin{itemize}
	           \item \xpgvalue{cyrillic-alph} steps through the Cyrillic alphabet. Thus it can only be used up to 30.
	           \item \xpgvalue{cyrillic-trad} (= \xpgvalue{cyrillic}) uses a traditional Cyrillic alphanumeric system.%
	                 \footnote{See \url{https://en.wikipedia.org/wiki/Cyrillic_numerals}.}
	                 It supports numbers up to 999\,999.
          \end{itemize}
\end{itemize}
%
\paragraph*{Commands:}
\begin{itemize}
    \item \Cmd\Asbuk: produces uppercased Cyrillic alphanumerals, for environments such as ¦enumerate¦.
          It steps through the Cyrillic alphabet and thus it can only be used up to 30.
          The command takes a counter as argument, \eg ¦\textmongolian{\Asbuk{section}}¦ produces \textrussian{\Asbuk{section}}.
    \item \Cmd\asbuk: same as \cmd\Asbuk\ but in lowercase.
    \item \Cmd\AsbukTrad: same as \cmd\Asbuk\ but using the traditional Cyrillic alphanumeric numbering which supports
          numbers up to 999\,999.\\
          \Eg ¦\textmongolian{\AsbukTrad{section}}¦ produces \textrussian{\AsbukTrad{section}}.
    \item \Cmd\asbukTrad: same as \cmd\AsbukTrad\ but in lowercase.
\end{itemize}

\subsection{norwegian}\label{norwegian}
\paragraph*{Options:}
\begin{itemize}
	\item \xpgchoicekey[1.45]{variant}{\xpgvalue{bokmal} or \xpgpresetvalue{nynorsk}}
\end{itemize}

\subsection{persian}\label{persian}
\paragraph*{Options:}
\begin{itemize}
	\item \xpgchoicekey{numerals}{\xpgvalue{western} or \xpgpresetvalue{eastern}}
	\item \xpgcodekey[1.63]{sectionsep}¦.¦
		This option allows to customize the separator between chapters, sections, and subsections (a dot by default),
		\eg \xpgoption{sectionsep=-}.
		This might be useful particularly with \xpgoption{numerals=eastern}, as the dot looks too similar to the zero
		in many fonts.
	\item \xpgboolkeyfalse[1.0.3]{abjadjimnotail}
		Set this to \xpgvalue{true} if you want the \textit{abjad} form of the number three to
		be \textarabic{ج‍} – as in the manuscript tradition – instead of the
		modern usage \textarabic{ج}.
%	\item \xpgchoicekey{locale}{} (not yet implemented)
%	\item \xpgchoicekey{calendar}{} (not yet implemented)
\end{itemize}
\paragraph*{Commands:}
\begin{itemize}
	\item \Cmd\abjad (see section \ref{abjad})
	\item \Cmd\aemph (see section \ref{arabic}).
\end{itemize}

\subsection{polish}\label{polish}

\paragraph*{Options:}
\begin{itemize}
	\item \xpgboolkeyfalse[1.55]{babelshorthands}
	If this is turned on, the following shorthands for Polish are activated:
	\begin{shorthands}
		\item[¦"=¦] for an explicit hyphen sign which is repeated at the beginning
		of the next line when hyphenated, as common in Polish typesetting
		(also if ¦splithyphens=false¦). Unlike  plain ¦-¦, this also allows
		for hyphenation at the other points preset in the hyphenation patterns.
		\item[¦"|¦] disables a ligature at this position.
		\item[\texttt{"\textasciitilde}] for a hyphen sign without a breakpoint. Useful for
		cases where the hyphen should stick at the following syllable.
		\item[¦"-¦] adds a hyphenation point that does still allow for hyphenation at the points preset
		in the hyphenation patterns (as opposed to \cmd\-).
		\item[¦""¦] allows for a line break at this position (without hyphenation sign).
		\item[¦"/¦] a slash that allows for a subsequent line break. As opposed to \cmd\slash, hyphenation at the breakpoints
		preset in the hyphenation patterns is still allowed.
		\item[¦"`¦] for Polish left double quotes (looks like ,,).
		\item[¦"'¦] for Polish right double quotes (looks like '').
		\item[¦"<¦] for French left double guillemets (looks like << -- used in Polish as second level quotes). 
		\item[¦">¦] for French right double guillemets (looks like >>).
	\end{shorthands}
	\item \xpgboolkeytrue[1.55]{splithyphens}
	According to Polish typesetting conventions, if a word with a hard hyphen (such as \emph{czerwono-złote})
	is hyphenated at this hyphen, a second hyphenation character is to be inserted at the beginning
	of the line that follows the hyphenation (\emph{czerwono-/-złote}).
	By default, this is done automatically (if you are using \LuaTeX, the \pkg{luavlna} package is
	loaded to achieve this).
	Set this option to ¦false¦ to disable the feature.
	\item \xpgboolkeytrue[1.55]{vlna}
	According to Polish typesetting conventions, single-letter words (non-syllable prepositions)
	must not occur at line ends.
	\pkg{Polyglossia} takes care of this automatically by default (if you are using \LuaTeX, the
	\pkg{luavlna} package is loaded to achieve this).
	Set this option to ¦false¦ to disable the feature.
\end{itemize}

\subsection{portuguese}\label{portuguese}
\paragraph*{Options:}
\begin{itemize}
	\item \xpgchoicekey[1.45]{variant}{\xpgvalue{brazilian} or \xpgpresetvalue{portuguese}}
	\item \xpgboolkeyfalse[1.54]{babelshorthands}
		If this is turned on, the following shorthands for fine-tuning hyphenation and micro-typography
		of Portuguese words are activated.
		\begin{shorthands}
			\item[¦"|¦] disables a ligature at this position.
			\item[¦"=¦] for an explicit hyphen sign which is repeated at the beginning
			of the next line when hyphenated, as common in Portuguese typesetting
			(also if ¦splithyphens=false¦). Unlike  plain ¦-¦, this also allows
			for hyphenation at the other points preset in the hyphenation patterns.
			\item[\texttt{"\textasciitilde}] for a hyphen sign without a breakpoint. Useful for
			cases where the hyphen should stick at the following syllable.
			\item[¦"-¦] adds a hyphenation point that does still allow for hyphenation at the points preset
			in the hyphenation patterns (as opposed to \cmd\-).
			\item[¦""¦] allows for a line break at this position (without hyphenation sign).
			\item[¦"/¦] a slash that allows for a subsequent line break. As opposed to \cmd\slash, hyphenation
			at the breakpoints preset in the hyphenation patterns is still allowed.
			\item[¦"<¦] for French left guillemets («).
			\item[¦">¦] for French right guillemets (»).
		\end{shorthands}
    \item \xpgboolkeytrue[1.54]{splithyphens}
		According to Portuguese typesetting conventions, if a word with a hard hyphen (such as
		\emph{pára-brisas}) is hyphenated at this hyphen, a second hyphenation character is to be
		inserted at the beginning of the line that follows the hyphenation (\emph{pára-/-brisas}).
		By default, this is done automatically (if you are using \LuaTeX, the \pkg{luavlna} package is
		loaded to achieve this).
		Set this option to ¦false¦ to disable the feature.
\end{itemize}

\subsection[punjabi]{punjabi\new{1.59}}\label{punjabi}
\paragraph*{Options:}
\begin{itemize}
	\item \xpgchoicekey{numerals}{\xpgvalue{western} or \xpgpresetvalue{gurmukhi}}
\end{itemize}

\subsection{russian}\label{russian}
\paragraph*{Options:}
	\begin{itemize}
	\item \xpgboolkeyfalse{babelshorthands}
	If this is turned on, the following shorthands are activated:
	\begin{shorthands}
		\item[¦"-¦] adds a hyphenation point that does still allow for hyphenation at the points preset
		           in the hyphenation patterns (as opposed to \cmd\-).
		\item[¦"=¦] adds an explicit hyphen with a breakpoint, allowing for hyphenation at the
		           other points preset in the hyphenation patterns (as opposed to plain ¦-¦).
		\item[\texttt{"\textasciitilde}] adds an explicit hyphen without a breakpoint. Useful for
		           cases where the hyphen should stick at the following syllable.
		\item[¦"|¦] disables a ligature at this position.
		\item[¦""¦] allows for a line break at this position (without hyphenation sign).
	 \end{shorthands}

	There are also three shorthands for the Cyrillic dash (\textrussian{тире}), which is shorter than the
	emdash but longer than the endash (namely 0.8\,em).
	Note that, since it is not covered by unicode, this character is faked by telescoping two endashes:
	\begin{shorthands}
		\item[¦"---¦] Cyrillic dash for the use in normal text. This requires preceding space
		in input (trailing space is optional) and prints with a non-breakable thin space before
		and after the dash.
		\item[¦"--\textasciitilde¦] Cyrillic dash for the use in compound names (surnames).
		As opposed to ¦"---¦ this removes any space before and after the dash. 
		\item[¦"--*¦] Cyrillic dash for denoting direct speech. This adds a larger space after
		the dash. Space before the dash is output as is.
% These are commented out in gloss-russian
%		\item[¦",¦] thinspace for initials with a breakpoint in following surname.
%		\item[¦"‘¦] for German left double quotes (looks like ,,).
%		\item[¦"’¦] for German right double quotes (looks like “).
%		\item[¦"<¦] for French left double quotes (looks like <<).
%		\item[¦">¦] for French right double quotes (looks like >>).
	\end{shorthands}
    \item \xpgboolkeytrue[1.50]{forceheadingpunctuation}
        By default, chapter and section numbers always have a trailing punctuation in Russian
        (as in \emph{1.1.} as opposed to \emph{1.1}). If this option is set to \xpgvalue{false}, \textsf{polyglossia}
        will not touch heading punctuation, so this will be whatever the class or a package determines.
	\item \xpgboolkeytrue[1.46]{indentfirst}
		By default, all paragraphs are indented in Russian, also those after a
		chapter or section heading. If this option is \xpgvalue{false}, the latter paragraphs
		are not indented, as normal in \LaTeX.
    \item \xpgboolkeytrue[1.52]{mathfunctions}
	    By default, some specific math macros are defined for Russian (see below). In order to prevent command
	    clashes (\eg with the \pkg{chemformula} package), you can switch these definitions off by passing \xpgvalue{false}
	    to this option.
	\item \xpgchoicekey{spelling}{\xpgpresetvalue{modern} or \xpgvalue{old}}
		This option is for captions and date only, not for hyphenation.
	\item \xpgchoicekey{numerals}{\xpgpresetvalue{arabic}, \xpgvalue{cyrillic-alph} or \xpgvalue{cyrillic-trad}}
		Uses either Arabic numerals or Cyrillic alphanumerical numbering. The two Cyrillic variants differ as follows:
		\begin{itemize}
			\item \xpgvalue{cyrillic-alph} steps through the Cyrillic alphabet. Thus it can only be used up to 30.
			\item \xpgvalue{cyrillic-trad} (= \xpgvalue{cyrillic}) uses a traditional Cyrillic alphanumeric system.%
			      \footnote{See \url{https://en.wikipedia.org/wiki/Cyrillic_numerals}.}
			      It supports numbers up to 999\,999.
		\end{itemize}
	\end{itemize}
%
\paragraph*{Commands:}
	\begin{itemize}
	\item \Cmd\Asbuk: produces uppercased Cyrillic alphanumerals, for environments such as ¦enumerate¦.
	      It steps through the Cyrillic alphabet and thus it can only be used up to 30.
	      The command takes a counter as argument, \eg ¦\textrussian{\Asbuk{section}}¦ produces \textrussian{\Asbuk{section}}.
	\item \Cmd\asbuk: same as \cmd\Asbuk\ but in lowercase.
	\item \Cmd\AsbukTrad: same as \cmd\Asbuk\ but using the traditional Cyrillic alphanumeric numbering which supports
	      numbers up to 999\,999.\\
	      \Eg ¦\textrussian{\AsbukTrad{page}}¦ produces \textrussian{\AsbukTrad{page}}.
	\item \Cmd\asbukTrad: same as \cmd\AsbukTrad\ but in lowercase.
	\end{itemize}
    If the \xpgoption{mathfunctions} option is \xpgvalue{true}, loading Russian defines a few macros than can be used
    independently of the current language. These are nine macros to be used in math mode to type the names of
    trigonometric functions common for Russian documents: \cmd\sh , \cmd\ch , \cmd\tg , \cmd\ctg , \cmd\arctg ,
    \cmd\arcctg , \cmd\th , \cmd\cth , and \cmd\cosec . Cyrillic letters in math mode can be typed with
    the aid of text commands such as \cmd\textbf , \cmd\textsf , \cmd\textit , \cmd\texttt , etc.
    The macros \cmd\Prob , \cmd\Variance , \cmd\NOD , \cmd\nod , \cmd\NOK , \cmd\nok , \cmd\Proj\ print some rare
    Russian mathematical symbols.

\subsection[sami]{sami\new{1.45}}\label{sami}
Currently support for Sami is limited to Northern Sami.

\subsection{sanskrit}\label{sanskrit}
\paragraph*{Options:}
	\begin{itemize}
	\item \xpgchoicekey[1.0.2]{script}{\xpgpresetvalue{devanagari}, \xpgvalue{gujarati},
	\xpgvalue{malayalam}, \xpgvalue{bengali}, \xpgvalue{kannada}, \xpgvalue{telugu}, \xpgvalue{latin}}
		The value is passed to \pkg{fontspec} should the respective
		\cmd{\⟨script⟩font} not be defined.  This can be useful if you typeset Sanskrit
		texts in scripts other than Devanagari.
	\item \xpgchoicekey[1.45]{numerals}{\xpgpresetvalue{devanagari} or \xpgvalue{western}}
	\end{itemize}

\subsection{serbian}\label{serbian}
\paragraph*{Options:}
	\begin{itemize}
	\item \xpgchoicekey{script}{\xpgpresetvalue{cyrillic} or \xpgvalue{latin}}
	\item \xpgboolkeytrue[1.54]{splithyphens}
			According to Serbian typesetting conventions, if a word with a hard hyphen (such as
			\emph{калцијум-карбонат}) is hyphenated at this hyphen, a second hyphenation character is to be
			inserted at the beginning of the line that follows the hyphenation (\emph{калцијум-/-карбонат}).
			By default, this is done automatically (if you are using \LuaTeX, the \pkg{luavlna} package is
			loaded to achieve this).
			Set this option to ¦false¦ to disable the feature.
	\item \xpgchoicekey{numerals}{\xpgpresetvalue{arabic}, \xpgvalue{cyrillic-alph} or \xpgvalue{cyrillic-trad}}
          Uses either Arabic numerals or Cyrillic alphanumerical numbering. The two Cyrillic variants differ as follows:
          \begin{itemize}
               \item \xpgvalue{cyrillic-alph} steps through the Cyrillic alphabet. Thus it can only be used up to 30.
	           \item \xpgvalue{cyrillic-trad} (= \xpgvalue{cyrillic}) uses a traditional Cyrillic alphanumeric system.%
	                  \footnote{See \url{https://en.wikipedia.org/wiki/Cyrillic_numerals}.}
	                  It supports numbers up to 999\,999.
          \end{itemize}
\end{itemize}
%
\paragraph*{Commands:}
\begin{itemize}
    \item \Cmd\Asbuk: produces uppercased Cyrillic alphanumerals, for environments such as ¦enumerate¦.
          It steps through the Cyrillic alphabet and thus it can only be used up to 30.
          The command takes a counter as argument, \eg ¦\textserbian{\Asbuk{section}}¦ produces \textrussian{\Asbuk{section}}.
    \item \Cmd\asbuk: same as \cmd\Asbuk\ but in lowercase.
    \item \Cmd\AsbukTrad: same as \cmd\Asbuk\ but using the traditional Cyrillic alphanumeric numbering which supports
          numbers up to 999\,999.\\
          \Eg ¦\textserbian{\AsbukTrad{page}}¦ produces \textrussian{\AsbukTrad{page}}.
    \item \Cmd\asbukTrad: same as \cmd\AsbukTrad\ but in lowercase.
\end{itemize}


\subsection{slovak}\label{slovak}

\paragraph*{Options:}
\begin{itemize}
	\item \xpgboolkeyfalse[1.46]{babelshorthands}
		If this is turned on, the following shorthands for Slovak are activated:
		\begin{shorthands}
        \item[¦"=¦] for an explicit hyphen sign which is repeated at the beginning
                    of the next line when hyphenated, as common in Slovak typesetting
		            (also if ¦splithyphens=false¦). Unlike  plain ¦-¦, this also allows
		            for hyphenation at the other points preset in the hyphenation patterns.
        \item[¦"|¦] disables a ligature at this position.
        \item[\texttt{"\textasciitilde}] for a hyphen sign without a breakpoint. Useful for
                    cases where the hyphen should stick at the following syllable.
        \item[¦"-¦] adds a hyphenation point that does still allow for hyphenation at the points preset
                    in the hyphenation patterns (as opposed to \cmd\-).
        \item[¦""¦] allows for a line break at this position (without hyphenation sign).
        \item[¦"/¦] a slash that allows for a subsequent line break. As opposed to \cmd\slash, hyphenation at the breakpoints
                    preset in the hyphenation patterns is still allowed.
        \item[¦"‘¦] for Slovak left double quotes (looks like ,,).
        \item[¦"’¦] for Slovak right double quotes (looks like “).
        \item[¦">¦] for Slovak left double guillemets (looks like >>).
        \item[¦"<¦] for Slovak right double guillemets (looks like <<).
		\end{shorthands}
	\item \xpgboolkeytrue[1.46]{splithyphens}
	      According to Slovak typesetting conventions, if a word with a hard hyphen (such as \emph{je-li})
	      is hyphenated at this hyphen, a second hyphenation character is to be inserted at the beginning
          of the line that follows the hyphenation (\emph{je-/-li}).
	      By default, this is done automatically (if you are using \LuaTeX, the \pkg{luavlna} package is
	      loaded to achieve this).
	      Set this option to ¦false¦ to disable the feature.
	\item \xpgboolkeytrue[1.46]{vlna}
	      According to Slovak typesetting conventions, single-letter words (non-syllable prepositions)
	      must not occur at line ends.
	      \pkg{Polyglossia} takes care of this automatically by default (if you are using \LuaTeX, the
	      \pkg{luavlna} package is loaded to achieve this).
	      Set this option to ¦false¦ to disable the feature.
\end{itemize}

\subsection{slovenian}\label{slovenian}
\paragraph*{Options:}
	\begin{itemize}
	\item \xpgboolkeyfalse{localalph}
	      If \xpgvalue{true}, alphanumeric counters will use a localized version including characters with caron
	      (\emph{a, b, c, č, d, \ldots}) and excluding the characters \meta{q}, \meta{w}, \meta{x} and \meta{y}
	      from alphabetic counting.
	\end{itemize}

\subsection{sorbian}\label{sorbian}
\paragraph*{Options:}
\begin{itemize}
	\item \xpgchoicekey[1.45]{variant}{\xpgvalue{lower} or \xpgpresetvalue{upper}}
	\item \xpgboolkeyfalse[1.45]{olddate}
		If \xpgvalue{true}, \cmd\today\ will use traditional Sorbian month names (\ie it will be
		synonymous to \cmd\oldtoday\ below).
\end{itemize}
\paragraph*{Commands:}
\begin{itemize}
	\item \Cmd\oldtoday: outputs the current date using traditional Sorbian month names, even if
	       \xpgvalue{olddate} is \xpgvalue{false}.
\end{itemize}

\subsection{spanish}\label{spanish}
\paragraph*{Options:}
\begin{itemize}
	\item \xpgchoicekey[1.46]{variant}{\xpgpresetvalue{spanish} or \xpgvalue{mexican}}
	\item \xpgchoicekey[1.46]{spanishoperators}{\xpgdefaultvalue{all}, \xpgvalue{accented},
	\xpgvalue{spaced}, \xpgvalue{none}, or \xpgpresetvalue{false}}
	      Determines of and how math operators are localized to Spanish.
	      \begin{itemize}
	      	\item ¦accented¦ causes some math operators to use accents where usual in Spanish (\emph{lím},
	      	      \emph{lím\,sup}, \emph{lím\,inf}, \emph{máx}, \emph{mín}, \emph{ínf}, \emph{mód}).
	      	\item ¦spaced¦ causes some math operators to use spaces where usual in Spanish (\emph{arc\,cos},
	      	      \emph{arc\,sen}, \emph{arc\,tg}).
	      	\item ¦all¦ activates ¦accented¦ and ¦spaced¦ and furthermore provides Spanish localizations of
	      	      \cmd\sin\ (\emph{sen}), \cmd\tan\ (\emph{tg}), \cmd\sinh\ (\emph{senh}), and \cmd\tanh\ (\emph{tgh}).
	      	\item ¦none¦ does no localization at all (default setting).
	      \end{itemize}
\end{itemize}
\paragraph*{\color{black}Commands:}\new{1.46}
\begin{itemize}
	\item \Cmd\arcsen: alias to \cmd\arcsin\ (\pkg{babel} compatibility)
	\item \Cmd\arctg: alias to \cmd\arctan\ (\pkg{babel} compatibility)
	\item \Cmd\sen: alias to \cmd\sin\ (\pkg{babel} compatibility)
	\item \Cmd\senh: alias to \cmd\sinh\ (\pkg{babel} compatibility)
	\item \Cmd\tg: alias to \cmd\tan\ (\pkg{babel} compatibility)
	\item \Cmd\tgh: alias to \cmd\tanh\ (\pkg{babel} compatibility)
	\item \Cmd\spanishoperator: allows you to define further localized operators. For instance, ¦\spanishoperator{cotg}¦
	      defines a command \cmd\cotg\ that outputs \emph{cotg} in math. The optional argument of the command lets you specify the
	      spelling, if needed, \eg ¦\spanishoperator[arc\,ctg]{arcctg}¦.
\end{itemize}

\subsection{syriac}\label{syriac}
\paragraph*{Options:}
	\begin{itemize}
	\item \xpgchoicekey[1.0.1]{numerals}{\xpgpresetvalue{western} (\ie 1234567890), \xpgvalue{eastern}
		(for which the Oriental Arabic numerals are used: \textarabic{١٢٣٤٥٦٧٨٩٠}), or
		\xpgvalue{abjad}}
	\item \xpgcodekey[1.63]{sectionsep}¦.¦
		This option allows to customize the separator between chapters, sections, and subsections (a dot by default),
		\eg \xpgoption{sectionsep=-}.
		This might be useful particularly with \xpgoption{numerals=eastern}, as the dot looks too similar to the zero
		in many fonts.
	\end{itemize}
\paragraph*{Commands:}
	\begin{itemize}
	\item \Cmd\abjadsyriac (see section \ref{abjad})
  \item \Cmd\aemph (see section \ref{arabic}).
	\end{itemize}

\subsection{thai}\label{thai}
\paragraph*{Options:}
	\begin{itemize}
	\item \xpgchoicekey{numerals}{\xpgvalue{thai} or \xpgpresetvalue{arabic}}
	\end{itemize}
%
To insert word breaks, you need to use an external processor.
See the documentation to \pkg{thai-latex} and the file \file{testthai.tex}
that comes with this package.

\subsection{tibetan}\label{tibetan}
\paragraph*{Options:}
\begin{itemize}
	\item \xpgchoicekey{numerals}{\xpgvalue{tibetan} or \xpgpresetvalue{arabic}}
\end{itemize}

\subsection{ukrainian}\label{ukrainian}

\paragraph*{Options:}
\begin{itemize}
	\item \xpgboolkeyfalse{babelshorthands}
	If this is turned on, the following shorthands are activated:
	\begin{shorthands}
		\item[¦"-¦] adds a hyphenation point that does still allow for hyphenation at the points preset
		     in the hyphenation patterns (as opposed to \cmd\-).
		\item[¦"=¦] adds an explicit hyphen with a breakpoint, allowing for hyphenation at the
		     other points preset in the hyphenation patterns (as opposed to plain ¦-¦).
		\item[\texttt{"\textasciitilde}] for a hyphen sign without a breakpoint. Useful for
		     cases where the hyphen should stick at the following syllable.
		\item[¦"|¦] disables a ligature at this position.
		\item[¦""¦] allows for a line break at this position (without hyphenation sign).
		% These are commented out in gloss-ukrainian
		%		\item[¦",¦] thinspace for initials with a breakpoint in following surname.
		%		\item[¦"‘¦] for German left double quotes (looks like ,,).
		%		\item[¦"’¦] for German right double quotes (looks like “).
		%		\item[¦"<¦] for French left double quotes (looks like <<).
		%		\item[¦">¦] for French right double quotes (looks like >>).
	\end{shorthands}

	There are also three shorthands for the Cyrillic dash (\textrussian{тире}), which is shorter than the
	emdash but longer than the endash (namely 0.8\,em).
	Note that, since it is not covered by unicode, this character is faked by telescoping two endashes:
	\begin{shorthands}
		\item[¦"---¦] Cyrillic dash for the use in normal text. This requires preceding space
		in input (trailing space is optional) and prints with a non-breakable thin space before
		and after the dash.
		\item[¦"--\textasciitilde¦] Cyrillic dash for the use in compound names (surnames).
		As opposed to ¦"---¦ this removes any space before and after the dash. 
		\item[¦"--*¦] Cyrillic dash for denoting direct speech. This adds a larger space after
		the dash. Space before the dash is output as is.
	\end{shorthands}
    \item \xpgboolkeytrue[1.52]{mathfunctions}
		By default, some specific math macros are defined for Ukrainian (see below). In order to prevent command
		clashes (\eg with the \pkg{chemformula} package), you can switch these definitions off by passing \xpgvalue{false}
		to this option.
	\item \xpgchoicekey{numerals}{\xpgpresetvalue{arabic}, \xpgvalue{cyrillic-alph} or \xpgvalue{cyrillic-trad}}
	      Uses either Arabic numerals or Cyrillic alphanumerical numbering. The two Cyrillic variants differ as follows:
	\begin{itemize}
		\item \xpgvalue{cyrillic-alph} steps through the Cyrillic alphabet. Thus it can only be used up to 30.
		\item \xpgvalue{cyrillic-trad} (= \xpgvalue{cyrillic}) uses a traditional Cyrillic alphanumeric system.%
		\footnote{See \url{https://en.wikipedia.org/wiki/Cyrillic_numerals}.}
		It supports numbers up to 999\,999.
	\end{itemize}
\end{itemize}
%
\paragraph*{Commands:}
\begin{itemize}
	\item \Cmd\Asbuk: produces uppercased Cyrillic alphanumerals, for environments such as ¦enumerate¦.
	It steps through the Cyrillic alphabet and thus it can only be used up to 30.
	The command takes a counter as argument, \eg ¦\textukrainian{\Asbuk{section}}¦ produces \textrussian{\Asbuk{section}}.
	\item \Cmd\asbuk: same as \cmd\Asbuk\ but in lowercase.
	\item \Cmd\AsbukTrad: same as \cmd\Asbuk\ but using the traditional Cyrillic alphanumeric numbering which supports
	numbers up to 999\,999.\\
	\Eg ¦\textukrainian{\AsbukTrad{page}}¦ produces \textrussian{\AsbukTrad{page}}.
	\item \Cmd\asbukTrad: same as \cmd\AsbukTrad\ but in lowercase.
\end{itemize}
%
If the \xpgoption{mathfunctions} option is \xpgvalue{true}, loading Ukrainian defines a few macros than can be used
independently of the current language. These are nine macros to be used in math mode to type the names of
trigonometric functions common for Ukrainian documents: \cmd\sh , \cmd\ch , \cmd\tg , \cmd\ctg , \cmd\arctg ,
\cmd\arcctg , \cmd\th , \cmd\cth , and \cmd\cosec . Cyrillic letters in math mode can be typed with
the aid of text commands such as \cmd\textbf , \cmd\textsf , \cmd\textit , \cmd\texttt , etc.
The macros \cmd\Prob , \cmd\Variance , \cmd\NOD , \cmd\nod , \cmd\NOK , \cmd\nok , \cmd\NSD , \cmd\nsd , \cmd\NSK ,
\cmd\nsk , \cmd\Proj\ print some rare
Ukrainian mathematical symbols.

\subsection{urdu}\label{urdu}
\paragraph*{Options:}
\begin{itemize}
	\item \xpgchoicekey{calendar}{\xpgpresetvalue{gregorian} or \xpgvalue{hijri}}
	\item \xpgoptkey{hijricorrection}{integer}[0]
		If \xpgvalue{calendar=hijri} is selected, \cmd{\today} will output the date
		according to the lunar Islamic (Hijra) calendar. This option allows to shift the day of the output with respect
		to the current date. Both positive and negative integer values are allowed (negative decrease the day value).
		This corresponds to the optional argument of the \cmd{\Hijritoday} command which can be used to output Hijri
		dates irrespective of the \xpgvalue{calendar} option (see sec.~\ref{sec:hijrical}).
	\item \xpgchoicekey{numerals}{\xpgvalue{western} or \xpgpresetvalue{eastern}}
	\item \xpgcodekey[1.63]{sectionsep}¦.¦
		This option allows to customize the separator between chapters, sections, and subsections (a dot by default),
		\eg \xpgoption{sectionsep=-}.
		This might be useful particularly with \xpgoption{numerals=eastern}, as the dot looks too similar to the zero
		in many fonts.
	\item \xpgboolkeyfalse{abjadjimnotail}
		Set this to \xpgvalue{true} if you want the \textit{abjad} form of the number three to
		be \textarabic{ج‍} – as in the manuscript tradition – instead of the
		modern usage \textarabic{ج}.
\end{itemize}	

\subsection{uyghur}\label{uyghur}
\paragraph*{Options:}
\begin{itemize}
	\item \xpgchoicekey[1.53]{chapterformat}{\xpgdefaultvalue{\xpgpresetvalue{ordinal}},
		\xpgvalue{roman}, \xpgvalue{arabic}}
	In Uyghur, chapter and part headings usually use ordinals rather than numbers. Alternatively Roman numbers
	are common as well. By default, \pkg{polyglossia} uses ordinals.
	This option allows to change this. Choices are:
	\begin{itemize}
		\item ¦ordinal¦: Use ordinals for chapter and part headings (=~default setting)
		\item ¦roman¦: Use Roman numbering for chapter and part heading.
		\item ¦arabic¦: Use Arabic numbering for chapter and part heading (=~LaTeX default).
	\end{itemize}
    Note that this option only works if \xpgoption{swapstrings} is set to \xpgvalue{all}, \xpgvalue{headings} or
    \xpgvalue{hheaders}. The caveats for that options (see below) apply.
	\item \xpgchoicekey[1.53]{swapstrings}{\xpgdefaultvalue{\xpgpresetvalue{all}},
		\xpgvalue{captions}, \xpgvalue{headings}, \xpgvalue{headers}, \xpgvalue{hheaders}, or \xpgvalue{none}}
	In Uyghur, table and figure caption strings need to be in a different order and format than in other languages.
	By default, \pkg{polyglossia} provides the correct for different classes and packages (standard classes, \pkg{KOMA-script},
	\pkg{memoir}, and \pkg{titlesec} package should work, as well as \pkg{fancyhdr} and \pkg{caption}).
	However, since the definition of these strings is not standardized, the redefinitions might not work
	and even interfere badly if you use specific classes or packages that redefine the respective strings
	themselves. In this case, you can disable some or all changes.
	The possibilities are:
	\begin{itemize}
		\item ¦all¦: Redefine figure and table captions, part and chapter headings, and running headers (=~default setting)
		\item ¦captions¦: Redefine figure and table captions only
		\item ¦headings¦: Redefine part and chapter headings only
		\item ¦headers¦: Redefine running headers only
		\item ¦hheaders¦: Redefine part and chapter headings as well as running headers
		\item ¦none¦: Do not redefine anything
	\end{itemize}
\end{itemize}
%
\paragraph*{Commands:}
\begin{itemize}
	\item \Cmd\uyghurordinal: outputs Uyghur ordinals. Supported input: integer numbers in the range of 1--100.
	\item \Cmd\uyghurord: same as \cmd\uyghurordinal\ but with counter as input (\eg ¦\textuyghur{\uyghurord{page}}¦).
\end{itemize}

\subsection{welsh}\label{welsh}
\paragraph*{Options:}
	\begin{itemize}
	\item \xpgchoicekey{date}{\xpgvalue{long} or \xpgpresetvalue{short}}
	\end{itemize}


\section{Modifying or extending captions, date formats and language settings}

\pkg{Polyglossia} uses the following macros to define language-specific captions
(\ie strings such as ``chapter''), date formats and additional language settings
(\meta{lang} is to be replaces with the respective language name):

\begin{itemize}
	\item \Cmd{\captions\meta{lang}} stores definitions of caption strings
	           (such as, in the case of English, ¦\def\chaptername{Chapter}¦)
	\item \Cmd{\date\meta{lang}} stores definitions of date formats (usually redefinitions
	           of \cmd\today, in some cases also definitions of additional date commands)
	\item \Cmd{\blockextras@\meta{lang}} stores macros that are to be executed when the language
              \meta{lang} is activated via \cmd\selectlanguage\ command or the \meta{lang} environment
	\item \Cmd{\inlineextras@\meta{lang}} stores macros that are to be executed when the language
	          \meta{lang} is activated locally via \cmd\text\meta{lang} command
	\item \Cmd{\noextras@\meta{lang}} stores macros that are to be executed when the language
	          \meta{lang} is closed
\end{itemize}
%
In order to redefine internal macros, we recommend to use the command \cmd\gappto.
For compatibility with \pkg{babel} the command \cmd\addto\ is also available
to the same effect. For instance, to change the \cmd\chaptername\ for language ¦lingua¦,
you can do this:
\begin{verbatim}
\gappto\captionslingua{\def\chaptername{Caput}}
\end{verbatim}
%
Note that this needs to be done after the respective language has been loaded with
\cmd\setmainlanguage\ or \cmd\setotherlanguage.

Specifically for package authors, analogous commands are provided which are only executed
if a specific language \emph{variety} is used. As opposed to the macros above, these refer
to babel names. Other than that, the function is identical:

\begin{itemize}
	\item \Cmd{\captions@bbl@\meta{babelname}}
	\item \Cmd{\date@bbl@\meta{babelname}}
	\item \Cmd{\blockextras@bbl@\meta{babelname}}
	\item \Cmd{\inlineextras@bbl@\meta{babelname}}
	\item \Cmd{\noextras@bbl@\meta{babelname}}
\end{itemize}
%
By default, these macros are undefined. If they are defined (\eg by an external package),
they will be executed after their respective \meta{lang} counterpart and thus can be used to
overwrite definitions of the former. Again, use \cmd\gappto\ to define\slash modify these macros.
For instance, to add a new caption \cmd\footnotename\ to the Swiss variety of German (babel name
¦nswissgerman¦), you can do this:
\begin{verbatim}
\gappto\captions@bbl@nswissgerman{\def\footnotename{Fussnote}}
\end{verbatim}
%
If you do this in a document preamble rather than in a package, you need to embrace the redefinition
by \cmd\makeatletter\ and \cmd\makeatother\ due to the ¦@¦ in the macro names.

Finally, as soon as the language has been switched (either inline or as a block), \pkg{polyglossia}
executes the (by default empty) hook
\begin{itemize}
	\item \cmd{\polyglossia@language@switched}\DescribeMacroCust{\scriptsize}{\polyglossia@language@switched}
\end{itemize}
to which you can append arbitrary code (via \cmd\gappto) that should be executed if (a particular)
language is being activated. This is done before any of the above
macros are issued (so you can still alter them), but at a point where \cmd\languagename, \cmd\babelname\
and \cmd\languageid\ are already set, so you can condition on specific languages in your code.
This hook is particularly provided for package authors.

\section{Script-specific numbering}

Languages and scripts have specific numbering conventions. Some use decimal digits
(\eg Arabic numerals), some use alphabetic or alphanumerical notation (\eg Roman numbering).
In some cases, different conventions are available (\eg Mashriq or Maghrib numbering
in Arabic script, Arabic or Hebrew [=~alphanumeric] numbering in Hebrew).

If the latter is the case, \pkg{polyglossia} provides language options which allow you to select
or switch to the suitable convention. With the appropriate language option set, \pkg{polyglossia}
will automatically convert the output of internal \LaTeX\ counters to their
localized forms, for instance to display page, chapter and section numbers.

For manual input of numbers, macros are provided. These convert Arabic numeric input to the respective
local decimal digit (see sec.~\ref{sec:decdigit}), alphanumeric representation (see sec.~\ref{abjad})
or whatever is appropriate (see sec.~\ref{sec:localnumber}). The possibilities are described in turn.

\subsection{General localization of numbering}\label{sec:localnumber}

As of 1.45,\new{1.45} \pkg{polyglossia} provides a generic macro \Cmd\localnumeral\ which converts numbers
to the current local form (which might be script-specific decimal digit, an alphabetic numbering or something else).
For instance in an Arabic environment
¦\localnumeral{42}¦ yields \textarabic{\localnumeral{42}}, whereas in an Hebrew environment, it
results in \texthebrew[numerals=hebrew]{\localnumeral{42}} with ¦numerals=hebrew¦, and \texthebrew{\localnumeral{42}}
with ¦numerals=arabic¦. Note that, as opposed to the various ¦digits¦ macros (described in sec.~\ref{sec:decdigit}),
the argument of \cmd\localnumeral\ must consist of numbers only.

For\new{1.45} the conversion of counters, the starred version \Cmd{\localnumeral*} is provided. This takes a counter as argument.
For instance in an Arabic environment ¦\localnumeral*{page}¦ yields \textarabic{\localnumeral*{page}}.

For scripts with alphanumeric numbering, the variants \Cmd{\Localnumeral} and \Cmd{\Localnumeral*} provide the uppercased
versions.\medskip

\noindent All these macros provide the following options:

\begin{itemize}
	\item \DescribeMacro{[lang=]}\xpgchoicekey{lang}{\xpgpresetvalue{local}, \xpgvalue{main},
	or \xpgvalue{\meta{language}}}
		Output number in the local form of the currently active language for
		¦local¦, the main language of the document for ¦main¦, and any (loaded)
		language for \meta{language} (\eg ¦\localnumeral[lang=arabic]{42}}¦).
\end{itemize}

\subsection{Non-Western decimal digits}\label{sec:decdigit}

In addition\new{1.1.1} to the generic macros described above, \pkg{polyglossia} provides language-specific conversion macros
which can be used if the generic ones do not suit the need.\footnote{%
A third method are so-called TECKit fontmappings.
Those can be activated with the \pkg{fontspec} ¦Mapping¦ option,
using ¦arabicdigits¦, ¦farsidigits¦ or ¦thaidigits¦.
For instance if \cmd\arabicfont\ is defined with the option ¦Mapping=arabicdigits¦,
typing \cmd{\textarabic\{2010\}} results in \textarabic{٢٠١٠}. Note that this method has some drawbacks, though,
for instance when the value of a counter has to be written and read from auxiliary files.
So please use this with care.}
The macros have the form \cmd{\⟨script⟩digits}. They convert Arabic numerical input and leave every other input untouched.
In an Arabic context, for instance, ¦\arabicdigits{9182/738543-X}¦ yields \textarabic{\arabicdigits{9182/738543-X}}.

Currently, the following macros are provided:

\begin{itemize}
	\item \Cmd\arabicdigits
	\item \Cmd\bengalidigits
	\item \Cmd\devanagaridigits
	\item \Cmd\farsidigits
	\item \Cmd\gurmukhidigits
	\item \Cmd\kannadadigits
	\item \Cmd\khmerdigits
	\item \Cmd\laodigits
	\item \Cmd\nkodigits
	\item \Cmd\thaidigits
	\item \Cmd\tibetandigits
\end{itemize}


\subsection{Non-Latin alphabetic numbering}\label{abjad}

For languages which use special (non-Latin) alphanumerical notation\footnote{%
	For instance, see \url{http://en.wikipedia.org/wiki/Greek_numerals},
	\url{http://en.wikipedia.org/wiki/Abjad_numerals},
	\url{http://en.wikipedia.org/wiki/Hebrew_numerals},
	and \url{http://en.wikipedia.org/wiki/Syriac_alphabet}.}, dedicated macros are provided.

They work in a similar way than the \cmd{\⟨script⟩digits} macros described above: They take Arabic numerical input
and output the respective value in the local alphabetic numbering scheme (most of these macros are equivalent
to \cmd\localnumeral\ and \cmd\Localnumeral\ in the respective context).

The following macros are provided:

\begin{itemize}
    \item \Cmd\abjad outputs Arabic \textit{abjad} numbers according to the Mashriq varieties.
			Example: ¦\abjad{1863}¦ yields \textarabic{\abjad{1863}}.
			
	\item \Cmd\abjadmaghribi outputs Arabic \textit{abjad} numbers according to the Maghrib varieties.
			Example: ¦\abjadmaghribi{1863}¦ yields \textarabic{\abjadmaghribi{1863}}.
			
	\item \Cmd\abjadsyriac outputs Syriac abjad numerals.\footnote{%
				A fine guide to numerals in Syriac can be found at \link{http://www.garzo.co.uk/documents/syriac-numerals.pdf}.}\\
			Example: ¦\abjadsyriac{463}¦ yields \textsyriac{\abjadsyriac{463}}.
	
	\item \Cmd\armeniannumeral produces Armenian alphabetic numbering.
	          Example: ¦\armeniannumeral{1863}¦ yields \textarmenian{\armeniannumeral{1863}}.
	          
	\item \Cmd\belarusiannumeral produces Belarusian numbering, with uppercased variant (for alphanumerical variant) via \Cmd\Belarusiannumeral.
	          Depending on the ¦numerals¦ option in the Belarusian language selection, this is either Arabic digit or Cyrillic
	          alphanumercial output.\\
	          Example: With ¦numerals=latin¦ ¦\belarusiannumeral{19}¦ yields \textrussian{\russiannumeral{19}},
	          with ¦numerals=cyrillic-trad¦ ¦\belarusiannumeral{19}¦ results in \textrussian[numerals=cyrillic-trad]{\russiannumeral{19}},\\
	          with ¦numerals=cyrillic-alph¦ ¦\belarusiannumeral{19}¦ results in \textrussian[numerals=cyrillic-alph]{\russiannumeral{19}}.

	\item \Cmd\chinesenumeral produces Chinese numbering which, depending on the ¦numerals¦ option in the Chinese language selection,
			  produces is either Arabic digit or Chinese ideographic output.\\
	          Example: With ¦numerals=arabic¦ ¦\chinesenumeral{753}¦ yields \textchinese[numerals=arabic]{\chinesenumeral{753}},
	          with ¦numerals=chinese¦ ¦\chinesenumeral{753}¦ results in \textchinese[numerals=chinese]{\chinesenumeral{753}}.

	\item \Cmd\georgiannumeral produces Georgian alphabetic numbering.\\
              Example: ¦\georgiannumeral{1863}¦ yields \textgeorgian{\georgiannumeral{1863}}.

	\item \Cmd\greeknumeral produces Greek alphabetic numbering, \Cmd\Greeknumeral outputs uppercased variants.
			Example: ¦\greeknumeral{1863}¦ yields \textgreek{\greeknumeral{1863}},
			¦\Greeknumeral{1863}¦ results in \textgreek{\Greeknumeral{1863}}.

	\item \Cmd\hebrewnumeral, \Cmd\Hebrewnumeral and \Cmd\Hebrewnumeralfinal produce variants of Hebrew alphanumeric numerals.
			The commands behave identical as in \pkg{babel}: \cmd\hebrewnumeral\ outputs the numbers without any decoration at all,
			\cmd\Hebrewnumeral\ adds \textit{gereshayim} before the last letter, and \cmd\Hebrewnumeralfinal\ uses in addition the final
			forms of Hebrew letters.
			Examples:
			¦\hebrewnumeral{1750}¦ yields \texthebrew{\hebrewnumeral{1750}},
			¦\Hebrewnumeral{1750}¦ yields \texthebrew{\Hebrewnumeral{1750}},
			and ¦\Hebrewnumeralfinal{1750}¦ yields \texthebrew{\Hebrewnumeralfinal{1750}}.
			
	\item \Cmd\mongoliannumeral produces Mongolian numbering, with uppercased variant (for alphanumerical variant) via \Cmd\Mongoliannumeral.
            Depending on the ¦numerals¦ option in the Mongolian language selection, this is either Arabic digit or Cyrillic
            alphanumercial output.\\
            Example: With ¦numerals=latin¦ ¦\mongoliannumeral{19}¦ yields \textrussian{\russiannumeral{19}},
            with ¦numerals=cyrillic-trad¦ ¦\mongoliannumeral{19}¦ results in \textrussian[numerals=cyrillic-trad]{\russiannumeral{19}},\\
            with ¦numerals=cyrillic-alph¦ ¦\mongoliannumeral{19}¦ results in \textrussian[numerals=cyrillic-alph]{\russiannumeral{19}}.
    
    \item \Cmd\punjabinumeral produces Punjabi numbering, depending on the setting of the ¦numerals¦ option in the Punjabi language
           selection, this is either Gurmukhi or Arabic (Western).
	
	\item \Cmd\russiannumeral produces Russian numbering, with uppercased variant (for alphanumerical variant) via \Cmd\Russiannumeral.
	        Depending on the ¦numerals¦ option in the Russian language selection, this is either Arabic digit or Cyrillic
	        alphanumercial output.\\
	        Example: With ¦numerals=latin¦ ¦\russiannumeral{19}¦ yields \textrussian{\russiannumeral{19}},
	        with ¦numerals=cyrillic-trad¦ ¦\russiannumeral{19}¦ results in \textrussian[numerals=cyrillic-trad]{\russiannumeral{19}},\\
	        with ¦numerals=cyrillic-alph¦ ¦\russiannumeral{19}¦ results in \textrussian[numerals=cyrillic-alph]{\russiannumeral{19}}.
			
	\item \Cmd\serbiannumeral produces Serbian numbering, with uppercased variant (for alphanumerical variant) via \Cmd\Serbiannumeral.
		    Depending on the ¦numerals¦ option in the Serbian language selection, this is either Arabic digit or Cyrillic
		    alphanumercial output.\\
		    Example: With ¦numerals=latin¦ ¦\serbiannumeral{19}¦ yields \textrussian{\serbiannumeral{19}},
			with ¦numerals=cyrillic-trad¦ ¦\serbiannumeral{19}¦ results in \textrussian[numerals=cyrillic-trad]{\serbiannumeral{19}},\\
			with ¦numerals=cyrillic-alph¦ ¦\serbiannumeral{19}¦ results in \textrussian[numerals=cyrillic-alph]{\serbiannumeral{19}}.
			
	\item \Cmd\ukrainiannumeral produces Ukrainian numbering, with uppercased variant (for alphanumerical variant) via \Cmd\Ukrainiannumeral.
			Depending on the ¦numerals¦ option in the Ukrainian language selection, this is either Arabic digit or Cyrillic
			alphanumercial output.\\
			Example: With ¦numerals=latin¦ ¦\ukrainiannumeral{19}¦ yields \textrussian{\russiannumeral{19}},
			with ¦numerals=cyrillic-trad¦ ¦\ukrainiannumeral{19}¦ results in \textrussian[numerals=cyrillic-trad]{\russiannumeral{19}},\\
			with ¦numerals=cyrillic-alph¦ ¦\ukrainiannumeral{19}¦ results in \textrussian[numerals=cyrillic-alph]{\russiannumeral{19}}.
\end{itemize}


\section{Footnotes in right-to-left context}

With languages that use right-to-left scripts, footnote apparatuses are usually placed at the right side of the page bottom.
Consequently, the footnote rule also is to be placed right. Things get more tricky, though, if right-to-left and left-to-right
scripts are mixed. Then you might want to put the footnotes on some pages left, on some right, or even mix positions on a page.
Thus, footnote handling in right-to-left context sometimes needs manual intervention. This is described in what follows.

\subsection{Horizontal footnote position}

When right-to-left languages are used, the \cmd\footnote\ command becomes sensitive to the text directionality. The footnote is
always placed on the side that is currently the origin of direction: on the left side of the page in LTR paragraphs and
on the right in RTL paragraphs.

For cases where this is not desired, two additional footnote commands are provided: \Cmd\RTLfootnote and \Cmd\LTRfootnote.
\cmd\LTRfootnote\ always places the footnote on the left side, notwithstanding the current
directionality. Likewise, \cmd\RTLfootnote\ always places it on the right side. Like \cmd\footnote, \cmd\RTLfootnote\
and \cmd\LTRfootnote\ provide an optional argument to customize the number.


\subsection{Footnote rule length and position}

The default placement of the footnote rule differs in \XeTeX\ and \LuaTeX\ output (this is due to differences in the \textsf{bidi}
and \textsf{luabidi} packages). With \XeTeX, footnote rules are always placed left, which is often wrong in RTL context.
With \LuaTeX, by contrast, the rule is placed always right if the main language is a right-to-left language, and always left
if the main language is a left-to-right language, which is the right thing in many cases.

In both cases, you can change the default behavior as follows:
\begin{itemize}
	\item Put \Cmd\leftfootnoterule in the preamble to have all rules left-aligned.
	\item Put \Cmd\rightfootnoterule in the preamble to have all rules right-aligned.
	\item Put \Cmd\autofootnoterule in the preamble to have automatic placement depending on the context (see below for elaboration).
	\item Put \Cmd\textwidthfootnoterule in the preamble to have a rule that spans the whole text width.
\end{itemize}
With \cmd\autofootnoterule, the first footnote on the current page determines the placement. Note that this automatic can fail with
footnotes at page boundaries that differ in directionality from the first footnote on the page. You can work around such cases by switching to \cmd\rightfootnoterule\ or \cmd\leftfootnoterule\ on these pages.

Note also that the rule switches might interfere in bad ways with packages or classes that redefine footnotes themselves. This is also the reason
why \cmd\autofootnoterule\ is not used by default.


\section{Calendars}

\subsection{Hebrew calendar (hebrewcal.sty)}
The package \file{hebrewcal.sty} is almost a verbatim copy of \file{hebcal.sty}
that comes with \pkg{babel}.
The command \Cmd\Hebrewtoday\ formats the current date in the Hebrew calendar
(depending of the current writing direction this will automatically set either
in Hebrew script or in roman transliteration).

\subsection{Islamic calendar (hijrical.sty)}\label{sec:hijrical}
This package computes dates in the lunar Islamic (Hijra) calendar.\footnote{ %
	It makes use of the arithmetical algorithm in chapter 6 of
	Reingold \& Gershowitz, \textit{Calendrical calculation: the Millenium edition}
	(Cambridge University Press, 2001).\label{reingold}}
It provides two macros for the end-user.
The command
	\displaycmd{\HijriFromGregorian\marg{year}\marg{month}\marg{day}}{\HijriFromGregorian}
sets the counters ¦Hijriday¦, ¦Hijrimonth¦ and ¦Hijriyear¦.
\Cmd\Hijritoday\ formats the Hijri date for the current day.
This command is now locale-aware\new{1.1.1}: its output will differ depending on the
currently active language. Presently \pkg{polyglossia}’s language definition files
for Arabic, Farsi, Urdu, Turkish and Malay provide a localized version of \cmd\Hijritoday.
If the formatting macro for the current language is undefined, the Hijri date will be formatted
in Arabic or in roman transliteration, depending of the current writing direction.
You can define a new format or redefine one with the command
  \displaycmd{\DefineHijriDateFormat\marg{lang}\marg{code}.}{\DefineHijriDateFormat}

The command \cmd\Hijritoday\ also accepts an optional argument to add or subtract a correction
(in days) to the date computed by the arithmetical algorithm.\footnote{ %
	The Islamic calendar is indeed a purely lunar calendar based on the observation
	of the first visibility of the lunar crescent at the beginning of the lunar month,
	so there can be differences between different localities, as well as between
	civil and religious authorities.}
For instance if \cmd\Hijritoday\ yields the date “7 Rajab 1429” (which is the date that was
displayed on the front page of \href{http://www.aljazeera.net}{aljazeera.net} on
11th July 2008), ¦\Hijritoday[1]¦ would rather print “8 Rajab 1429” (the date
indicated the same day on the site \href{http://www.gulfnews.com}{gulfnews.com}).

\subsection{Farsi (jalālī) calendar (farsical.sty)}
This package is an almost verbatim copy of ¦Arabiftoday.sty¦ (in the \pkg{Arabi} package),
itself a slight modification of ¦ftoday.sty¦ in Farsi\TeX.\footnote{ %
	One day we may rewrite \pkg{farsical} from scratch using the algorithm in
	Reingold \& Gershowitz (ref.~n.\,\ref{reingold}).}
Here we have renamed the command \cmd\ftoday\ to
\Cmd\Jalalitoday.
Example: today is \Jalalitoday.


\section{Auxiliary commands}

The macro\new{1.47} \displaycmd{\charifavailable\marg{char code}\marg{substitution}}{\charifavailable} checks whether
the character with the specified \meta{char code} (\ie unicode utf-16 code without preceding ¦0x¦) exists in
the current font. If so, the character is printed, if not, the \meta{substitution} is printed.

Example: ¦\charifavailable{1E9E}{SS}¦ prints the capital version of the German letter ⟨ß⟩ if available
(\ie \charifavailable{1E9E}{SS}), else it prints the substitution digraph SS.


\section{Accessing language information}\label{sec:langinfo}

The following is specifically relevant to package authors who need information about the languages in use
and the details of the respective locale.
In order to get such information, the recommended way is to use the \Cmd{\BCPdata\marg{type}}\new{1.61}
command to get (parts of) the BCP-47 language tag. This command is used by the \LaTeX\ kernel as of 6/2023,
and the syntax is standardized at least between \pkg{babel} and \pkg{polyglossia}.

\cmd\BCPdata\ supports the following \marg{type} arguments:
\begin{itemize}
	\item \xpgoption{language} The language subtag (\eg \texttt{de})
	\item \xpgoption{region} The region subtag (\eg \texttt{AT})
	\item \xpgoption{script} The script subtag (\eg \texttt{Latn})
	\item \xpgoption{variant} The variant subtag (\eg \texttt{1901} for German old spelling)
	\item \xpgoption{extension.t} The subtag indicating transformation between languages or scripts
	      (\eg \texttt{ja} as in \texttt{en-t-ja}: English transformed to Japanese)
	\item \xpgoption{extension.u} The subtag indicating additional locale information (\eg \texttt{nu-latn} as in \texttt{ar-u-nu-latn}:
	      Arabic with Latin Numbering)
	\item \xpgoption{extension.x} The private usespace subtag (\eg \texttt{classic} as in \texttt{la-x-classic} for classic Latin)
	\item \xpgoption{casing} is a special semantic value which returns whatever is considered suitable
	           for casing commands such as \cmd\MakeUppercase. In many cases, this is identical to \texttt{language} but
	           it could also be something more specific such as \texttt{el-x-iota} or \texttt{ckb-Latn} if this is
	           needed for proper casing. Note that this information is retrieved by the \LaTeX\ kernel for casing as
	           of 06/2023.
	\item \xpgoption{tag} returns the registered full tag (identical to \cmd\languageid\ below). This however omits
	       certain subtags if these are considered obvious (\eg for German, only \texttt{de-DE} is returned, although
	       script [\texttt{Latn}] and variant [\texttt{1996}] are defined as well).
\end{itemize}
If a subtag is not defined for a given language, an empty string is returned.

The values above return information for the language currently in use. If you need to access information for the 
main language of the document, prepend \xpgoption{main.} to the respective argument (\eg \xpgoption{main.language}). This works
for all supported arguments.

Next to \cmd\BCPdata, \pkg{polyglossia} provides the following legacy macros which are partly unique to \pkg{polyglossia},
partly re-implementations of legacy \pkg{babel} commands:

\begin{itemize}
	\item \Cmd\languagename\ stores the currently active (polyglossia) language name.
	\item \Cmd\mainlanguagename\ stores the (polyglossia) language name of the main document language.
	\item \Cmd\languagevariant\ stores the language variant if set. The macro is empty if no variant has been set.
	\item \Cmd\mainlanguagevariant\ stores the language variant of the main document language if set.
	       The macro is empty if no variant has been set.
	\item \Cmd\babelname\ stores the corresponding name of the currently active language (variant) in \pkg{babel}.
	      This might not only be useful if you want to support both \pkg{babel} and \pkg{polyglossia}, but also
	      since this name is unique for a given language variety (\eg ngerman, german, swissgerman etc.).
	      Note that this macro is also defined for languages that are not supported in \pkg{babel}. In that
	      case, they are equal to the polyglossia language name.
	\item \Cmd\mainbabelname\ analogously stores the name of document's main language (variant) in \pkg{babel}.
	\item \Cmd{\languageid\marg{type}}\new{1.47} stores the identifier tag of the current language. Currently supported \meta{types}:
	      \begin{itemize}
	      	\item ¦bcp-47¦ (alias ¦bcp47¦): IETF BCP-47 language identifier
	      \end{itemize}
	 \item \Cmd{\mainlanguageid\marg{type}}\new{1.47} stores identifier tag of the main language. Currently supported \meta{types}:
	      see \cmd\languageid.
\end{itemize}
\bigskip

\noindent If you want to have a full list of loaded languages/variants, use the following macros:
\begin{itemize}
	\item \Cmd{\xpg@loaded}\ stores a comma-separated list of all loaded languages (polyglossia name)
	\item \Cmd{\xpg@vloaded}\ stores a comma-separated list of all loaded variants
	\item \Cmd{\xpg@bloaded}\ stores a comma-separated list of \pkg{babel} names of all language variants
	\item \Cmd{\xpg@bcp@loaded}\new{1.47}\ stores a comma-separated list of the BCP-47 IDs of all language variants
\end{itemize}
\bigskip

\noindent Whether a language is loaded can be tested by
\displaycmd{\iflanguageloaded\marg{lang}\marg{true}\marg{false}}{\iflanguageloaded}
where \texttt{\meta{lang}} is a \pkg{polyglossia} language name, by
\displaycmd{\ifbabellanguageloaded\marg{lang}\marg{true}\marg{false}}{\ifbabellanguageloaded}
where \texttt{\meta{lang}} is a \pkg{babel} language name (see table~\ref{tab:bbllang} on p.~\pageref{tab:bbllang}), or by\new{1.47}
\displaycmd{\iflanguageidloaded\marg{type}\marg{id}\marg{true}\marg{false}}{\iflanguageidloaded}
where \meta{type} is a supported language id type (such as ¦bcp-47¦) and \meta{id} is a language id
(such as ¦en-US¦; see table~\ref{tab:BCP47-polyglossia} on p.~\pageref{tab:BCP47-polyglossia}).
\bigskip

\noindent Finally, if you want to know whether a specific language option has been set, you can use\new{1.47}
\displaycmd{\iflanguageoption\marg{lang}\marg{key}\marg{value}\marg{true}\marg{false}}{\iflanguageoption}

\section{Revision history}

\bgroup\footnotesize
\subsection*{1.63 (forthcoming)}

\subsubsection*{Bug fixes}
\begin{itemize}
	\item Language options are now lowercased before they are handled. This fixes errors when
	      using the options in \cmd{\MakeUppercase} context (\eg in headings).
	\item Fix casing value for medieval Latin (this concerns ⟨u⟩ and ⟨V⟩ casing).
	\item Remove hacks in some gloss files where \cmd\MakeUppercase\ has been redefined to a no-op
		since the respective scripts do not have uppercasing. As \cmd\MakeUppercase\ is now locale-aware,
		these hacks are no longer needed. This concerns Arabic, Divehi, Hebrew, Hindi, Kurdish,
		Persian, Punjabi, Syriac, and Urdu.
	\item Fix Eastern (Mashriq) Arabic numbering direction with \LuaLaTeX\ for Arabic, Kurdish, Persian,
		Syriac, and Urdu (amends previous fix to \TXI{213}).
	\item French \xpgoption{variant=acadian} refers to the Canadian French subvariety as spoken in the Acadian
	     region (as opposed to Quebec French). To differentiate it we now use a subtag (\texttt{fr-CA-u-sd-canb})
	     which can also be used to access the language. Note that Quebec French is currently connected with \texttt{fr-CA},
	     not the more specific \texttt{fr-CA-u-sd-caqc}, although the u specifier is returned if requested via \cmd\BCPdata.
\end{itemize}

\subsubsection*{New features}
\begin{itemize}
	\item New option \xpgoption{capitalyiwn} for Armenian.
	\item New option \xpgoption{capitaleszett} for German.
	\item New option \xpgoption{capitaliota} for Greek.
	\item New option \xpgoption{sectionsep} for Arabic, Kurdish, Persian,
		  Syriac, and Urdu (\TXI{589}).
	\item Add aliases \xpgvalue{eastern} (=~\xpgvalue{mashriq}) and \xpgvalue{western} (=~\xpgvalue{maghrib})
	      to Arabic \xpgoption{numerals}.
\end{itemize}

\subsubsection*{Interface changes}
\begin{itemize}
	\item All language options are now lowercase by default (though the mechanism is case-insensitive anyway).
		  Changes cover Kurdish, Sanskrit, Serbian.
\end{itemize}

\subsection*{1.62 (2023/04/22)}

\subsubsection*{Bug fixes}
\begin{itemize}
	\item \cmd\BCPdata\ was not fully expandable as required. Now it is.
\end{itemize}

\subsection*{1.61 (2023/04/16)}

\subsubsection*{Bug fixes}
\begin{itemize}
	\item Properly reset fonts on change to Latin-script other language (\TXI{580})
	\item Turn ¦^¦ shorthand off in aux file for Latin (\TXI{582})
	\item Fix stacking of nested languages and resulting bug in TOC language assignment (\TXI{585})
	\item Fix ¦\greeknumeral{6}¦ (\TXI{587})
	\item Fix (re)setting of scripts in inline language commands.
\end{itemize}

\subsubsection*{New features}
\begin{itemize}
	\item Introduce \cmd\BCPdata, a standardized way to gather localization information
	(see sec.~\ref{sec:langinfo}).
\end{itemize}

\subsection*{1.60 (2023/02/11)}

\subsubsection*{Bug fixes}
\begin{itemize}
	\item Remove stray \texttt{U+FEFF} glyphs in gloss files (\TXI{574})
	\item Fix vertical whitespace in captions (\TXI{575})
	\item Fix whitespace in dutch with shorthands.
\end{itemize}

\subsection*{1.59 (2022/11/29)}

\subsubsection*{Bug fixes}
\begin{itemize}
	\item Do not error, but warn, after language change due to non-updated aux file (\TXI{565}).
	\item Fix catcode issue for Latin (\TXI{566}).
	\item Improve error message with missing fonts (\TXI{571})
\end{itemize}

\subsubsection*{New features}
\begin{itemize}
\item Add support for (simplified and traditional) Chinese (\TXI{65}, provided by \TA{Takuji Tanaka}).
\item Add support for Punjabi (\TXP{572}), provided by \TA{Arvinder Singh}.
\end{itemize}

\subsection*{1.58 (2022/10/26)}

\subsubsection*{Bug fixes}
\begin{itemize}
    \item The option \xpgoption{forceheadingpunctuation} is introduced
        for correct running headers in Hungarian documents (\TXI{557}).
	\item Fix language setting in list of figure and table if captions float to different
	      language areas (\TXI{542}).
	\item Make Croatian digraphs robust (\TXI{552}).
	\item Fix definition of some Finnish shorthands (\TXI{554}).
	\item Re-fix Hungarian patching of parts with hyperref (\TXI{555}).
	\item Correct running headers in Hungarian documents (\TXI{557}).
	\item Fix option expansion issue (\TXI{559}).
\end{itemize}

\subsection*{1.57 (2022/07/18)}

\subsubsection*{Bug fixes}
\begin{itemize}
	\item Fix passing of macros to \cmd\setmainlanguage\ and \cmd\setotherlanguage\ (\TXI{543}).
	\item Fix ¦otherlanguage*¦ environment with bidi text (\TXP{544}).
	\item Re-fix patching of French part headings with hyperref (\TXI{546}).
	\item Fix Latin prosodic shorthand issues (\TXI{547}, \href{https://github.com/latex3/babel/issues/126}{Babel/\#126},
	      \href{https://github.com/latex3/babel/issues/129}{Babel/\#129}).
\end{itemize}

\subsubsection*{New features}
\begin{itemize}
	\item New option \xpgoption{transliteration} to Hebrew and hebrewcal (\TXI{540}).
	\item New option \xpgoption{fullyear} to Hebrew.
\end{itemize}

\subsection*{1.56 (2022/04/20)}

\subsubsection*{Bug fixes}
\begin{itemize}
	\item Fix English setup via aliases (\TXI{539}).
\end{itemize}


\subsection*{1.55a (2022/04/10)}

\subsubsection*{Bug fixes}
\begin{itemize}
	\item Fix \cmd\foreignlanguage\ command (\TXI{538}).
\end{itemize}

\subsection*{1.55 (2022/04/09)}

\subsubsection*{New features}
\begin{itemize}
	\item New option \xpgoption{splithyphens} and \xpgoption{vlna} for Polish (\TXI{535}).
	\item Add \xpgoption{babelshorthands} to Polish.
\end{itemize}

\subsubsection*{Bug fixes}
\begin{itemize}
	\item Add missing \texttt{gloss-latex.lde} file to release (\TXI{537}).
\end{itemize}

\subsection*{1.54 (2022/03/27)}

\subsubsection*{New features}
\begin{itemize}
	\item New option \xpgoption{splithyphens} for Serbian (\TXI{496}) and Portuguese (\TXI{534}).
	\item Add \xpgoption{babelshorthands} to Portuguese.
	\item Add \xpgoption{schoolhyphens} option to Finnish (\TXI{525}).
\end{itemize}

\subsubsection*{Interface and defaults changes}
\begin{itemize}
	\item Rename \xpgoption{disableligatures} to \xpgoption{disabledigraphs} for Croatian (\TXI{497}, \TXP{500}).
	\item Fix output with \xpgoption{numerals=cyrillic-alph} (part of \TXI{503}).
	\item Standardize February and November in Indonesian according to the Great Dictionary of the Indonesian Language
	      of the Language Center (\emph{Kamus Besar Bahasa Indonesia}) (\TXP{526}).
\end{itemize}

\subsubsection*{Bug fixes}
\begin{itemize}
	\item Fix robustification of font family switches (\TXI{428}).
	\item Preserve font family switches across languages (\TXI{519}).
	\item Fix \TeX\ dash ligatures with \xpgoption{splithyphens} (\TXI{502}).
	\item Prevent \emph{missing hyphenmins value} \LaTeX\ error with unknown languages (\TXI{513}).
	\item Fix global \xpgoption{babelshorthands}, \xpgoption{localmarks} and \xpgoption{verbose}
	      options (\TXI{515}).
	\item Fix Latin shorthands (\TXI{516}).
	\item Remove spurious space in Bosnian date (\TXP{528}).
	\item Fix \cmd\languagevariant\ and \cmd\mainlanguagevariant\ macros (\TXI{530}, \TXI{531}).
	\item Make \cmd\iflanguageloaded\ and friends work in preamble (\TXI{532}).
	\item Fix deactivation of shorthands (\TXI{320}).
	\item Fix deactivation of numerals.
	\item Update deprecated \LaTeX\ hooks (\TXI{523}).
	\item Fix \xpgoption{luatexrenderer} option which was not considered appropriately (\TXI{533}).
	\item Turn warning about \texttt{totalhyphenmin} to info, which is more appropriate (\TXI{111}).
\end{itemize}

\subsubsection*{Documentation improvements}
\begin{itemize}
	\item Document in this manual where to report bugs (\TXI{512}).
\end{itemize}


\subsection*{1.53 (2021/04/12)}

\subsubsection*{New features}
\begin{itemize}
	\item New option \xpgoption{localalph} for Croatian (\TXI{486}).
	\item Use ordinals in part and chapter headings. A new option \xpgoption{chapterformat} has been
	      added to disable this or change format to Roman numbering (\TXI{479}).
	\item New command \cmd\uyghurordinal\ and \cmd\uyghurord\ to produce Uyghur ordinals from 1--100.
\end{itemize}

\subsubsection*{Bug fixes}
\begin{itemize}
	\item Fix typo in Uyghur language definition file (\TXI{479}).
	\item Fix Uyghur date format (\TXI{479}).
	\item Fix Uyghur caption format and add \xpgoption{swapstrings} option (\TXI{479}).
	\item Add overlooked digraphs in Croatian (\TXP{484}).
	\item Fix typo in Serbian ¦\thepart¦ (\TXP{485}).
	\item Fix part modifications in Hungarian with \pkg{hyperref} (\TXI{493}).
\end{itemize}

\subsection*{1.52 (2021/03/16)}

\subsubsection*{New features}
\begin{itemize}
\item Adaptations to \LaTeX\ 2021/05/01 pre-release 2 for Korean (\TXI{477}).
\item Add support for Uyghur (\TXI{475}).
\item New option \xpgoption{mathfunctions} for Russian and Ukrainian allows to disable the definitions
      of math macros that might clash with other packages (\TXI{465}).
\item Support LaTeX's new NFSS hooks (\TXI{471}).
\end{itemize}

\subsubsection*{Bug fixes}
\begin{itemize}
\item Fix French part modifications with \pkg{hyperref} (\TXI{469}).
\item Fix markup of French ¦\see¦ and ¦\alsoname¦ (\TXI{468}).
\end{itemize}

\subsection*{1.51 (2020/12/08)}

\subsubsection*{New features}
\begin{itemize}
\item New option \xpgoption{frenchpart} for French (\TXI{458}).
\item New option \xpgoption{splithyphens} for Croatian (\TXI{454}).
\end{itemize}

\subsubsection*{Bug fixes}
\begin{itemize}
\item Use new LaTeX core hooks rather than \pkg{filehook} package. This fixes
      a recent breakage of \pkg{filehook} with other external packages (\TXI{453}).
\item Remove very old code that pretends \pkg{polyglossia} is \pkg{babel} (\TXI{455}).
\item Fix spelling of Albanian contentsname (\TXI{456}).
\item Fix part heading modification in French (\TXI{458}).
\item Fix extra space in Hebrew (\TXI{459}).
\item Register main polyglossia language earlier (\TXI{461}).
\item Allow for hyphenations in words following opening guillemet in French with XeTeX (\TXI{462}).
\end{itemize}


\subsection*{1.50a (2020/10/15)}

\subsubsection*{Bug fixes}
\begin{itemize}
\item Assure ¦\autodot¦ is defined with ¦\KOMAScript¦ in Russian.
\end{itemize}

\subsection*{1.50 (2020/10/09)}

\subsubsection*{New features}
\begin{itemize}
\item Polyglossia now uses the Harfbuzz renderer by default with LuaTeX
      output. This brings LuaTeX on par with XeTeX for all scripts (\TXI{337}).
      The renderer can be changed via the new global \xpgoption{luatexrenderer} option.
\item The (previously inadvertently working) ¦hyphenrules¦ environment that ceased
      to work after a recent \pkg{babel} update is back and now officially supported.
      The environment now also supports language options and aliases (\TXI{427}).
\item New command ¦\setlanghyphenmins¦ to adapt hyphenation thresholds of languages
      and varieties.
\item New command ¦\abjadalph¦ for Arabic with corresponding option (\TXI{431}).
\item Replace consecutive glues around punctuation by the correct amount of space
      with lualatex for French, ecclesiastic Latin, and Sanskrit (\TXI{437}).
\end{itemize}

\subsubsection*{Bug fixes}
\begin{itemize}
\item Remove warning about missing Brazil patterns (\TXI{404}).
\item Fix incompatibility with recent \pkg{babel} release (\TXI{408}).
\item Fixed some spellings in Marathi (\TXI{409}).
\item Fix spacing of geminating dot in Catalan (\TXI{410}).
\item Fix incompatibility of Marathi with \pkg{beamer}.
\item Correct ¦\partname¦ in Hindi (\TXI{416}).
\item Updates and improvements to Kurdish (\TXI{418}).
\item Only activate shorthand character if \xpgoption{babelshorthands} is \xpgvalue{true} (\TXI{421}). 
\item Fix whitespace issue in Czech and Slovak with \xpgvalue{vlna=true} (\TXI{423}).
\item Fix whitespace issue in Danish (\TXI{424}).
\item Fix catcode conflicts that might occur in language definition files
      f.\,ex.\ when loaded from a LaTeX3 class (\TXI{67}, \TXI{425}).
\item Robustify font family switches (\TXI{428}).
\item Fix whitespace issue in Russian \xpgoption{indentfirst} option (\TXI{433}).
\item In Russian, \xpgoption{indentfirst} is now again default (\TXI{434}).
\item Fix LaTeX error with arabic numbering in Ukrainian (\TXI{440}).
\item Fix directionality after Hebrew decimal numbers (\TXI{441}).
\item Fix ¦babelname¦ of Latin Serbian (\TXI{442}).
\item Fix recording of secondary languages in ¦\xpg@bloaded¦ and ¦\xpg@bcp@loaded¦
      lists (\TXI{443}).
\item Simplify and robustify section heading modification in Russian
      and introduce option \xpgoption{forceheadingpunctuation} (\TXI{444}).
\item Fix Cyrillic dash (via babelshorthand ¦"---¦) when TeX ligatures
      are disabled (\TXI{445}).
\item Fix problem with large character indices in Lua module for punctuation
      spacing
\end{itemize}

\subsubsection*{Interface and defaults changes}
\begin{itemize}
\item Polyglossia now uses the Harfbuzz renderer by default with LuaTeX
      output. See new features section.
\end{itemize}

\subsubsection*{Build fixes}
\begin{itemize}
\item Fix an embarrassing bug in the dtx build script which was the reason for an utterly
       incomplete ¦polyglossia.dtx¦ file (\TXI{420}).
\end{itemize}

\subsubsection*{Documentation improvements}
\begin{itemize}
\item Document how to change ¦\lefthyphenmin¦ and ¦\righthyphenmin¦ for a language
      (\TXI{435}).
\end{itemize}


\subsection*{1.49 (2020/04/08)}

\subsubsection*{New features}
\begin{itemize}
\item Add hook ¦\polyglossia@language@switched¦ to the external package interface (\TXI{398}).
\item Real fix for \TXI{400}, that wasn’t properly taken care of in 1.48.
\end{itemize}

\subsubsection*{Bug fixes}
\begin{itemize}
\item Fix compilation error with some \xpgoption{swapstring} options in Hungarian (\TXI{373}).
\item Fix whitespace problem in Greek language.
\end{itemize}

\subsubsection*{Interface and defaults changes}
\begin{itemize}
\item Changed Finnish caption for ``Table of Contents'' to ``Sisällys'' (\TXI{403}).
\end{itemize}


\subsection*{1.48 (2020/03/25)}

\begin{itemize}
\item No new features
\end{itemize}

\subsubsection*{Bug fixes}
\begin{itemize}
\item Fix use of Hebrew with LuaLaTeX (\TXI{389}).
\item Do not overwrite footnote redefinitions of other packages
      with Latin and French (\TXI{391}).
\item Fix Serbian cyrillic numerals code (\TXI{392}).
\item Fix \xpgoption{[no]localmarks} option, whose logic was swapped (part of \TXI{395}).
\item Protect \xpgoption{localmarks} function against uppercased language names (part of \TXI{395}).
\item Fix buggy redefinition of ¦\@markright¦ with option localmarks (\TXI{396}).
\item Fix incompatibility between Latin and unicode-math (\TXI{394}).
\item Make (undocumented) ¦\defineshorthand¦ command (imported from babel) work.
\item Fix usage of \xpgoption{localmarks} option without value.
\item Emergency fixes for bugs caused by updates in \pkg{babel}’s ¦switch.def¦ (\TXI{399} and \TXI{400}).
\end{itemize}

\subsubsection*{Interface and defaults changes}
\begin{itemize}
\item Use private macros in keyval choice keys (\TXI{390}).
\end{itemize}

\subsection*{1.47 (2020/01/29)}

\subsubsection*{New features}
\begin{itemize}
\item IETF BCP-47 compliant language tags can now be used for loading and switching
      languages alternatively to language names (\TXI{226}).
\item New commands ¦\languageid{<type>}¦ and ¦\mainlanguageid{<type>}¦.
\item New test ¦\iflanguageidloaded¦.
\item New list ¦\xpg@bcp@loaded¦.
\item New environment ¦{lang}{<lang>}¦ (this is equivalent to ¦{<lang>}¦,  but also available with
      ¦\setlanguagealias*¦ which does not define dedicated alias environments).
\item New gloss option ¦totalhyphenmin¦ (corresponds to LuaTeX's ¦\hyphenationmin¦) (\TXI{111}).
\item New test ¦\iflanguageoption{<lang>}{<key>}{<val>}¦ (\TXI{364}).
\item Restore simple alphabetic numbering for ¦\asbuk¦ and ¦\Asbuk¦ in Belarusian, Mongolian,
      Russian, Serbian, and Ukrainian (\TXI{377}).
\item New command ¦\AsbukTrad¦ and ¦\asbukTrad¦ for Belarusian, Mongolian, Russian, Serbian,
      and Ukrainian which uses traditional alphanumerical numbering.
\item New numerals option ¦cyrillic-trad¦ and ¦cyrillic-alph¦ to differentiate simple
      alphabetic and traditional alphanumerical Cyrillic numbering.
\item ¦\selectbackgroundlanguage¦ and ¦\resetdefaultlanguage¦ now also support language
      aliases.
\item New macro ¦\charifavailable{<char code>}{<substitution>}¦.
\item Add French language variant ¦swiss¦.
\item Implement \xpgoption{babelshorthands} for Croatian.
\item Implement ¦\localnumeral¦ for Japanese.
\end{itemize}

\subsubsection*{Bug fixes}
\begin{itemize}
\item Fix font family issue in headers (\TXI{355}).
\item Fix whitespace issues in ¦\text<lang>¦ (\TXI{356}).
\item Fix option-less ¦\babelname¦ in multi-variant languages (\TXI{357}).
\item Fix some spacing inconsistencies with French, Latin, and Sanskrit (\TXI{358}).
\item Fix issues with \xpgoption{babelshorthands} and \pkg{graphics} package (\TXI{368}).
\item Fix some captions and improve numbering in Marathi (\TXI{370}).
\item Fix Hungarian \xpgoption{swapstrings} feature (\TXI{373}).
\item Fix lua punctuation code problem (\TXI{374}).
\item Fix Bengali changecounternumbering option (\TXI{381}).
\item Fix whitespace issue in Japanese (\TXI{387}).
\item Fix ¦\text<lang>¦ command with multiple paragraphs.
\item Actually implement documented german spelling variant ¦1996¦ (= ¦new¦).
\item Fix Slovenian \xpgoption{localalph} option.
\item Fix Czech and Slovak \xpgoption{splithyphens} with typewriter fonts.
\item ¦farsical.sty¦: fix spacing issue with some month names.
\item Fix directionalty of numbers in Hebrew with XeTeX.
\item Improve interoperatability with \pkg{biblatex} (some language variants did not work yet).
\end{itemize}

\subsubsection*{Interface and defaults changes}
\begin{itemize}
\item Some boolean options had ¦false¦ value by default, which meant if you passed
      them without value, the logic was reversed. This has been changed, leading to
      change of behavior should you have used one of these options without value (\TXI{363}).
      Concerned are the following options:
      \begin{itemize}
	   \item \xpgoption{babelshorthands} in language Belarusian, Mongolian, Ukrainian, and Russian
             (now \xpgoption{babelshorthands} equals \xpgoption{babelshorthand=true}, no longer \xpgoption{babelshorthands=false}).
	   \item \xpgoption{localalph} in language Slovenian (\xpgoption{localalph} now equals \xpgoption{localalph=true}).
	   \item \xpgoption{fullyear} in package \pkg{hebrewcal} (\xpgoption{fullyear} now equals \xpgoption{fullyear=true}).
      \end{itemize}
\item The option \xpgoption{latesthyphen} in language German is deprecated. \XeTeX\ and \LuaTeX\ nowadays
      always use the latest German hyphens.
\item The command ¦\setlanguagealias*¦ (introduced in v1.46) does no longer define
      dedicated alias environments.
\item The babelnames for Latin variants are now changed to ¦classiclatin¦, ¦ecclesiasticlatin¦
      and ¦medievallatin¦. This is how the hyphenation patterns and \pkg{babel} ¦\extras¦ are named, even
      though the variants can currently be selected in \pkg{babel} only via appended ``dot modifier''.
\item In accordance with the respective \pkg{l3kernel} change, ¦\str_lower_case:n¦  has been renamed to
      ¦\str_lowercase:n¦ where used in ¦polyglossia.sty¦. Thus \pkg{polyglossia} 1.47 requires \pkg{l3kernel}
      2020-01-12 at least.
\end{itemize}

\subsection*{1.46 (2019/11/15)}

\subsubsection*{New features}
\begin{itemize}
\item Add option \xpgoption{indentfirst} to Russian (\TXI{78}).
\item Add options to set and customize French-style itemize item labels to French (\TXI{89}).
\item \pkg{Polyglossia} now decodes all supported \pkg{babel} language names in ¦\setdefaultlanguage¦,
      ¦\setotherlanguage¦ and the language switching commands (\TXI{112}, \TXI{132}).
\item Add optional localized math operators to Spanish (\TXI{123}).
\item Swap section headings in Hungarian (\TXI{344}). New option \xpgoption{swapstrings} provides control
      over this.
\item Introduce macro ¦\setlanguagealias¦ and ¦\setlanguagealias*¦.
\item Introduce language switching command ¦\textlang{lang}{...}¦ (these are equivalent to ¦\text<lang>¦,
      but also available with ¦\setlanguagealias*¦ which does not define ¦\text<alias>¦).
\item Add support for Afrikaans.
\item Add support for Belarusian.
\item Add support for Bosnian.
\item Add support for Georgian.
\item Add Spanish variant ¦mexican¦.
\item Add babelshorthands as well as options \xpgoption{splithyphens} and \xpgoption{vlna} to Slovak.
\item Add Latin language variant ¦ecclesiastic¦.
\item Add Latin language options ¦capitalizemonth¦, ¦ecclesiasticfootnotes¦, ¦hyphenation¦,
      ¦prosodicshorthands¦, and ¦usej¦.
\item Add Latin shorthands for «, », æ, Æ, œ, and Œ.
\item Add French language option \xpgoption{thincolonspace}.
\end{itemize}

\subsubsection*{Bug fixes}
\begin{itemize}
\item Fix problems with fragile font settings (\TXI{24}).
\item Fix clash of French punctuation spacing with the \pkg{soul} package (\TXI{52}).
\item Re-enable the possibility to pass a macro as main argument to ¦\setmainlanguage¦ and
      ¦\setotherlanguage¦ (\TXI{331}).
\item Fix detection of default ¦\languagevariant¦ (\TXI{332}).
\item Fix LaTeX error with undefined hyphenation pattern (\TXI{346}).
\item Fix some babel shorthand issues by updating the shorthand code from recent \pkg{babel}.
\item Fix some problems with French and Latin auto-spacing (\TXI{345}, \TXI{352}).
\item Fix an \pkg{expl3} declaration (\TXI{348}).
\end{itemize}

\subsubsection*{Interface and defaults changes}
\begin{itemize}
\item The sub-package ¦cyrillicnumbers.sty¦ has been renamed to
      ¦xpg-cyrillicnumbers.sty¦ (per TeXLive request).
\item In Russian, all paragraphs are now indented by default, as common in Russian typography.
      The behavior can be opted out by \xpgoption{indentfirst=false}.
\item In Czech, \xpgoption{splithyphens} and \xpgoption{lvna} are enabled by default.
      Also, the option does now work as well with LuaTeX.
\item Changed option name ¦fraktur¦ to ¦blackletter¦ in German (the former is still available
      as an alias).
\item In French, high punctuation characters and guillemets are spaced by half an interword
      space now instead of a ¦\thinspace¦ (cf. \TXI{345}).
\end{itemize}

\subsection*{1.45 (2019/10/27)}

\subsubsection*{New features}
\begin{itemize}
\item Introduce a framework for external packages to access language variants. This fixes,
      among other things, long-standing problems in the interaction of \pkg{biblatex} and \pkg{polyglossia}.
\item Add new macros ¦\languagevariant¦ and ¦\mainlanguagevariant¦ as well as ¦\babelname¦ and ¦\mainbabelname¦
      for package authors to access language information.
\item Add new test ¦\iflanguageloaded{<language>}{<true>}{<false>}¦ where <language>
      can be a \pkg{polyglossia} or \pkg{babel} language name.
\item Add new macros ¦\localnumeral,¦ ¦\localnumeral*,¦ ¦\Localnumeral¦ and
      ¦\Localnumeral*¦ that convert Arabic digitals to the local number scheme.
\item Add new macro ¦\pghyphenation¦ to add language-specific hyphenation exceptions (\TXI{18}).
\item Add support form (Khalkha \& Cyrillic) Mongolian in line with \pkg{babel-mongolian} (\TXI{23}).
\item Add option \xpgoption{splithyphens} and \xpgoption{vlna} to Czech (XeTeX only; for LuaTeX, use the package
      \pkg{luavlna} to get these features) (\TXI{32}).
\item Add support for Kurdish, both Kurmanji and Sorani (\TXI{277}).
\item Implement proper Cyrillic (alphanumeric) numbering (\TXI{285}).
\item Add new language ¦friulian¦.
      This deprecates ¦friulan¦ (which is still supported for backwards compatibility).
\item Add new language ¦malay¦ with variants ¦indonesian¦ and ¦malaysian¦.
      This deprecates ¦bahasai¦ and ¦bahasam¦ (which are still supported for
      backwards compatibility).
\item Add new language ¦gaelic¦ with variants ¦irish¦ and ¦scottish¦.
      This deprecates ¦irish¦ and ¦scottish¦ as own \pkg{polyglossia} languages (which
      are still supported for backwards compatibility).
\item Add new language ¦hungarian¦.
      This deprecates ¦magyar¦ (which is still supported for backwards compatibility).
\item Add new language ¦sorbian¦ with variants ¦lower¦ and ¦upper¦.
      This deprecates ¦lsorbian¦ and ¦usorbian¦ (which are still supported for
      backwards compatibility).
\item Add new language ¦portuguese¦ with variants ¦portuguese¦ and ¦brazilian¦.
      This deprecates ¦brazil¦ and ¦portuges¦ (which are still supported for
      backwards compatibility).
\item Add new language ¦norwegian¦ with variants ¦nynorsk¦ and ¦bokmal¦.
      This deprecates ¦nynorsk¦ and ¦norsk¦ (which are still supported for
      backwards compatibility).
\item Add new language ¦persian¦.
      This deprecates ¦farsi¦ (which is still supported for backwards compatibility).
\item Add new language ¦sami¦. Currently only Northern Sami is supported.
      This deprecates ¦samin¦ (which is still supported for backwards compatibility).
\item ¦gloss-serbian¦: add \xpgoption{numerals=cyrillic} option. Add ¦\asbuk¦ and ¦\Asbuk¦ (\TXI{285}).
\item Implement basic support for (French) ¦canadien¦ and (English) ¦canadian¦ (\TXI{22}).
\item Improve support for Armenian (\TXI{79}): Add captions, Eastern month names
      (accessible via variant=eastern) and Armenian alphabetic numbering
      (via \xpgoption{numerals=armenian} and ¦\armenicnumeral¦).
\item Add french option \xpgoption{autospacing} and commands ¦\AutoSpacing,¦ ¦\NoAutoSpacing¦
      This allows to switch off autospacing globally or locally (\TXI{113}).
\item Fixup ¦\normalfont¦ (\TXI{203}).
\item Fix directionality issues in mixed RTL/LTR paragraphs (\TXI{204}).
\item Implement \xpgoption{babelshorthands} for Finnish (\TXI{212}) and Czech.
\item Implement access to current language via Lua (\TXI{243}).
\item Introduce french option option \xpgoption{autospacetypewriter}
      alias \xpgoption{OriginalTypewriter}.
\item Support ¦\aemph¦ with lualatex
\item Rename \xpgoption{automaticspacesaroundguillemets} to \xpgoption{autospaceguillemets} 
      The old option is kept for backwards compatibility.
\end{itemize}

\subsubsection*{Bug fixes}
\begin{itemize}
\item Fix equation number in Arabic and Farsi (\TXI{7}).
\item Simplify and document Hebrew \xpgoption{marcheshvan} option (\TXI{16}).
\item Fix hyphenation of Greek with LuaTeX (\TXI{55}).
\item Fix N'ko date format (\TXI{63}).
\item Disable the extras of a language when a nested language starts (\TXI{66}, \TXI{169}).
\item Properly implement Bengali numbers (\TXI{69}, \TXI{184}).
\item Fix conflicts with other packages caused by premature shorthand activation
      in preamble (\TXI{81}, \TXI{200}).
\item Fix kerning in math with French (\TXI{92}).
\item Fix expansion issue in Hebrew (\TXI{93}).
\item Fix numbering expansion issue in Greek (\TXI{110}).
\item Postpone ¦\disablehyphenation¦ in preamble until after setting of
      document language (\TXI{125}).
\item Postpone the assignment of defaultfamily to ¦\AtBeginDocument¦,
      thus do not overwrite ¦\familydefault¦ redefinitions in the preamble (\TXI{127}).
\item Reset number settings when switching language (\TXI{133}).
\item Hebrew: Properly store ¦\MakeUppercase¦ for later restoration (\TXI{152}).
\item Fix whitespace issue in ¦\datewelsh¦ (\TXI{158}).
\item When switching language, set the language/script specific font families (\TXI{164}).
\item Correct some Bengali captions (\TXI{165}).
\item Fix documentation of Serbian (\TXI{168}).
\item Reset ucl codes in Latin only if the respective variant is used (\TXI{172}).
\item Fix ¦\disablehyphenation¦ with LuaTeX (\TXI{187}).
\item Fix typos in Hindi captions (\TXI{202}).
\item Pass language options to the aux files (\TXI{205}).
\item Rewrite and fix English variant handling (\TXI{208}).
\item Define magyar caption formats in ¦\blockextras¦ and undef them in ¦\noextras¦ (\TXI{209}).
\item Ensure proper direction with arabic digits in Arabic and Farsi (\TXI{213}).
\item Fix ¦\linespread¦ with Korean (\TXI{218}).
\item Define Russian caption before key allocation (\TXI{219}).
\item Register current language in \pkg{polyglossia} lua module after selection (\TXI{234}).
\item Fix \pkg{babel} language switching commands (\TXI{239}):
      ¦\foreignlanguage¦ and the starred ¦otherlanguage*¦ environment are not
      supposed to change dates.
\item Fix French spacing leaking beyond French (\TXI{270}).
\item Redefine font families for French only if language is loaded (\TXI{270}).
\item ¦gloss-russian¦: 
     \begin{itemize}
	 \item Check whether command exist before redefining (\TXI{280}).
	 \item Fix some whitespace issues.
     \end{itemize}
\item Fix and simplify ¦\frenchfootnote¦ definition (\TXI{294}).
\item Fix footnote numbering in Farsi.
\item Fix Latin footnotes in Arabic documents.
\item Set the correct main direction with \pkg{luabidi}.
\item Fix \xpgoption{autospaceguillemets} option in French.
\item Fix grouping in ¦gloss-danish.ldf¦.
\item Properly store ¦\MakeUppercase¦ and ¦\@arabic¦ for later restoration.
\end{itemize}

\subsubsection*{Documentation}
\begin{itemize}
\item Add documentation about footnotes in RTL context
\item Document Tibetan numerals option (\TXI{109}).
\item Improve ¦\frenchfootnote¦ documentation.
\item Mention Japanese support in the docs.
\end{itemize}

\subsection*{1.44 (2019/04/04)}
\begin{itemize}
\item Correction to Russian language file, by \TA{Maksim Zholudev} (commit d2f383e).
\item Added Macedonian language file, by \TA{Stefan Zlatinov} (commit cd379e1).
\end{itemize}

\subsection*{1.43 (2019/03/05)}
\begin{itemize}
\item Correction to Hindi language file, by \TA{Zdenĕk Wagner}.
\end{itemize}

\subsection*{1.42.5 (2017/04/13)}
\begin{itemize}
\item Many changes to the French language file, by \TA{Maïeul Rouquette}.
\end{itemize}

\subsection*{1.42.4 (February, March 2016)}
\begin{itemize}
\item Remedial actions for the \pkg{babel} changes.
\item Fixed side effect of pull request \TXI{122} (see commit d2a34ff).
\item Added automatic Josa selection, variant, and captions options to Korean, by \TA{Dohyun Kim} (pull request \TXI{128}).
\item Updated ¦gloss-occitan¦ from CTAN.
\end{itemize}

\subsection*{18-01-2016}
\begin{itemize}
	\item Fixed issue \TXI{124} (minor typo in ¦polyglossia-frpt.lua¦)
	\item Merged pull request \TXI{117} for more French guillemet spacing
	\item Merged pull request \TXI{121} to add ¦\bbl@loaded¦; fixes issue \TXI{120}
	\item Merged pull request \TXI{122} that build on \TXI{121}
	\item Merged pull request \TXI{116} for French (spacing around guillemets)
	\item Fixed issue \TXI{115} (spurious spaces in Arabic)
\end{itemize}
	
\subsection*{19-08-2015}
\begin{itemize}
	\item Fixed issue \TXI{107} for Marathi (labels and month names)
\end{itemize}
		
\subsection*{1.42.0 (2015/08/06)}
\begin{itemize}
	\item Add Bengali digits package, and option to reset all numbering functions.
	\item Add ¦long¦ option for Welsh date.
	\item Add local alphabet in Slovenian, for enumerations and such.
	\item Fix long-standing bug with Welsh: date should use ordinals.
	\item Fix for Latin with LuaTeX: all variants had same problems as Classic.
	\item Fixed error with British variant of English and LuaTeX (issue \TXI{86}).
\end{itemize}
			
\subsection*{1.41.0 (2015/07/16)}
\begin{itemize}
	\item Added support for Khmer, by \TA{Say Ol} (private email)
\end{itemize}
			
\subsection*{1.40.1 (2015/07/14)}
\begin{itemize}
	\item Bugfix for Korean, by \TA{Dohyun Kim} (pull request \TXI{103})
\end{itemize}
			
\subsection*{1.40.0 (2015/07/07)}
\begin{itemize}
	\item ¦gloss-korean.ldf¦ contributed by \TA{Dohyun Kim} (pull request \TXI{102})
\end{itemize}
			
\subsection*{1.33.7 (2015/07/04)}
\begin{itemize}
	\item Release to CTAN, no code change
	\item Fixed extraneous space in code for Swiss German (pull request \TXI{101})
	\item Fixed a typo in Ukrainian alphabet, for ¦\Asbuk¦ (pull request \TXI{99})
	\item Fix for Classic Latin: load patterns for LuaTeX
	\item Made ¦\rmfamily¦, ¦\sffamily¦ and ¦\ttfamily¦ robust again
	\item Merged fix for Hebrew date format, by \TA{Guy Rutenberg} (pull request \TXI{94})
	\item Merged fix for spurious space, by \TA{Caleb McKennan} (pull request \TXI{91})
	\item Merged pull request \TXI{84} by \TA{Élie Roux} for Tibetan
	\item Added support for Swiss German (pull request \TXI{75})
	\item Added commands ¦\Asbuk¦ and ¦\asbuk¦ for Ukrainian (pull request \TXI{76}), similar to Russian
	\item Documented changes to Latin from last year.
	\item Be friendlier to right-to-left languages with LuaTeX
	\item Enhanced Latin support by \TA{Claudio Beccari}
\end{itemize}

\subsection*{1.33.6 (2015/05/15)}
\begin{itemize}
	\item Introduce a ¦classical¦ and ¦medieval¦ variant of Latin
	\item Add ¦\asbuk¦ and ¦\Asbuk¦ for Ukrainian (after their Russian counterpart)
	\item Fix a number of bugs
\end{itemize}
					
\subsection*{1.33.5 (2014/05/21)}
\begin{itemize}
	\item Option to disable hyphenation entirely, by \TA{Élie Roux}
	\item Fix spurious spaces in gloss-russian.ldf, by \TA{Oleg Domanov}
	\item Support for the Austrian variant of German, by \TA{Jürgen Spitzmüller}
	\item Changes to the Croatian translations, by \TA{Ivan Kokan}
	\item Correction to the Lithuanian translations, by \TA{Ignas Anikevičius}
\end{itemize}
					
\subsection*{1.33.4 (2013/06/27)}
\begin{itemize}
	\item Emergency release for a bug introduced in ¦babelsh.def¦
\end{itemize}
					
\subsection*{1.33.3 (2013/05/28)}
\begin{itemize}
	\item Changed formatting of some error messages (emergency fixes for TeX Live 2013)
\end{itemize}
					
\subsection*{1.33.2 (2013/05/26)}
\begin{itemize}
	\item Added ¦\disablehyphenation¦ and ¦\enablehyphenation¦, contributed by
	      \TA{Élie Roux}.
	\item Fixed bug related to package inclusion. \pkg{Polyglossia} would break if
	      we loaded ¦breqn.sty¦, and then called ¦\setmainlanguage{english}¦, this
	      is no longer the case.
	\item Removed spurious space introduced by ¦\captionswedish¦.
\end{itemize}
					
\subsection*{1.33.1 (2013/05/23)}
\begin{itemize}
	\item Editorial changes to the documentation
	\item Hunted and documented bugs
\end{itemize}
					
\subsection*{1.33.0 (2013/05/20)}
\begin{itemize}
	\item Added support for N’Ko.
	\item Bugfixes for LuaTeX
	\item More work in progress on Bidi in LuaTeX.
\end{itemize}
					
\subsection*{1.32.0 (2013/05/15)}
\begin{itemize}
	\item Transitional version to support right-to-left languages with LuaTeX.
\end{itemize}

\subsection*{1.31 (2013/05/10) / 1.3 (2013/05/11)}
\begin{itemize}
	\item Several bugfixes.
	\item Sync with \pkg{babel} 3.9.
	\item Consolidated support for LuaTeX for all languages but the ones using
	      South and South-East Asian scripts, and languages written from right
	      to left.  Many thanks to \TA{Élie Roux} for his help.
	\item Added support for Tibetan, contributed by \TA{Élie Roux} (end of lines are experimental).
\end{itemize}
					
\subsection*{1.30 (2012/08/06)}
\begin{itemize}
	\item Added support for LuaTeX.  Many languages don’t work yet.  Please be patient.
\end{itemize}
					
\subsection*{1.2.0e (2012/04/28)}
\begin{itemize}
	\item Fixed a number of outstanding and not very interesting bugs.
	\item Added gloss files for Romansh and Friulan, contributed by \TA{Claudio Beccari}.
\end{itemize}
					
\subsection*{1.2.0d (2012/01/12)}
\begin{itemize}
	\item Removed ¦\makeatletter¦ and ¦\makeother¦ from gloss files entirely.
\end{itemize}
					
\subsection*{1.2.0c (2011/10/12) [First update by Arthur Reutenauer]}
\begin{itemize}
	\item Update to ¦gloss-italian.ldf¦ by \TA{Claudio Beccari}, incorporating changes
	      by \TA{Enrico Gregorio}.
	\item Conclude every gloss file with ¦\makeatother¦ to match the initial
	      ¦\makeatletter¦.  (Not necessary from a technical point of vue, because of one 
	      of the changes by Enrico reported below, but I like it better that way :-)
	\item Conclude ¦polyglossia.sty¦ with ¦\ExplSyntaxOff¦ to match the initial
	      ¦\ExplSyntaxOn¦.
	\item Added gloss file for Kannada, contributed by \TA{Aravinda VK} and others.
	\item Corrections to the gloss-dutch.ldf thanks to \TA{Wouter Bolsterlee}.
	\item Several patches by \TA{Enrico Gregorio}, fixing long-standing bugs.
	      From the git log:
			\begin{itemize}
				\item Deleted setup for right-to-left writing direction, see \url{http://tug.org/pipermail/xetex/2011-April/020319.html}
				\item Changed three appearances of ¦\newcommand¦ to ¦\newrobustcmd,¦ as the commands needs to be protected.
				      Bug reported by \TA{kamensky}.
				\item Corrected ¦\datepolish¦ as suggested by \TA{Piotr Kempa}
				\item Changed ¦\lccode"¦ into ¦\lccode\string",¦ because it might come into action at wrong times when ¦"¦ is active
				\item Changed definition of key ¦\xpg@setup¦, as ¦\@tmpfirst¦ and ¦\@tmpsecond¦ were not expanded, causing dependence
				      of ¦\lefthyphenmin¦ and   ¦\righthyphenmin¦ to the last loaded language.
				      Raised by \TA{Vadim Rodionov} on the XeTeX mailing list.
				\item Deleted ¦\bgroup¦ and ¦\egroup¦ tokens from the definition of ¦otherlanguage*¦; they serve no purpose
				      (we are already inside an environment) and conflict with \pkg{csquotes}. Noticed by \TA{P. Lehman}.
				\item Changed the calls of ¦\input¦ to ¦\xpg@input,¦ which inputs the required file and resets the catcode
				      of ¦@¦ to the same value as it had before the input. Since each ¦.ldf¦ file starts with ¦\makeatletter¦,
				      the old behaviour would leave a category 11 @, which is wrong.
				\item Added ¦\csuse{date#2}¦ to the definition of ¦otherlanguage*¦.
			\end{itemize}
\end{itemize}				
						
\subsection*{1.2.0b (2011/10/03) [Update by Philipp Stephani]}
\begin{itemize}
	\item Load \pkg{xkeyval} package explicitly since newer versions
	      of \pkg{fontspec} don't load it any more.
\end{itemize}
							
\subsection*{1.2.0a (2010/07/27) [Last update by François Charette]}
\begin{itemize}
	\item Initialize ¦\fontfamily¦ acc to value of ¦\familydefault¦
	      (Fixes a bug when using \pkg{polyglossia} with beamer)
	\item Remove spurious space in def of ¦\dateenglish¦
	\item Add missing English variant ¦american¦
	\item Serbian: fix date format and captions (Latin+Cyrillic)
	\item Fix ¦\atticnumeral¦ in ¦gloss-greek¦
	\item Small improvements and fixes in documentation
\end{itemize}
								
								
\subsection*{1.2.0 (2010/07/15)}
\begin{itemize}
	\item Adapted for \pkg{fontspec} 2.0 (will not work with earlier versions!)
	\item Implementation of a ¦\PolyglossiaSetup¦ interface
	      that simplifies writing ¦gloss-*.ldf¦ files
	\item Many internal enhancements and refactoring
	      (including a patch by \TA{Dirk Ulrich})
	\item Improved automatic font setup when ¦\<lang>font¦ is not defined
	\item New environment \texttt{otherlanguage*} (equivalent to ¦\foreignlanguage¦
	      (\TA{Enrico Gregorio})
	\item Bugfix to prevent bogus expansion of ¦\{rm,sf,tt}family¦ even in aux files (\TA{Enrico Gregorio})
	\item New gloss files for Armenian, Bengali, Occitan, Bengali, Lao,
	      Malayalam, Marathi, Tamil, Telugu, and Turkmen.
	\item New auxiliary packages ¦devanagaridigits¦ and ¦bengalidigits¦
	\item \pkg{hijrical} no longer loads \pkg{bidi} and checks for incompatible \pkg{l3calc}
	\item Add \pkg{babel} shorthands for Russian (based on a patch by \TA{Vladimir Lomov})
	\item Fix ¦\fnum@{table,figure}¦ for Lithuanian
	\item Various improvements in ¦gloss-russian¦ (provided by \TA{Vladimir Lomov} and
	      others)
	\item Corrected captions for Bahasai, Lithuanian, Russian, Croatian
	\item Add option \xpgoption{indentfirst=true} for Spanish, Croatian and other languages
	      (\xpgoption{indentfirst} was previously named \xpgoption{frenchindent})
	\item New option \xpgoption{script} for German: Setting \xpgoption{script=fraktur} modifies the
	      captions for typesetting in Fraktur.
	\item New command ¦\aemph¦ for Arabic, Farsi, Urdu, etc. to mark emphasis through
	      overlining.
	\item Package option \xpgoption{nolocalmarks} is now true by default: to activate it the
	      option \xpgoption{localmarks} must be passed instead.
	\item Loading languages à la \pkg{babel} as package options is no longer possible (this
	      feature was deprecated since v1.1.0).
\end{itemize}
							
\subsection*{1.1.1 (2010/03/23)}
\begin{itemize}
	\item Bugfix for French: explicit spaces before/after double punctuation
	      signs (Par exemple : les grands « espaces » du Canada !) are
	      now replaced by the appropriate non-breaking spaces, as in \pkg{babel}.
	\item Bugfix for font switching mechanism within Latin script
	      (pending a complete re-implemen\-tation in v1.2):
	      the font shape and series are no longer reset when switching language.
	\item New macros for non-Western decimal digits
	      (instead of fontmappings)
	\item New gloss files for Asturian, Lithuanian and Urdu
	\item ¦hijrical.sty¦ is now locale-aware: ¦\hijritoday¦ is
	      formatted differently in Arabic, Farsi, Urdu, Turkish
	      and Bahasa Indonesia.
	\item Enable \xpgoption{babelshorthands} for Dutch
	\item Add missing macro ¦\allowhyphens¦
	\item Add global option \xpgoption{babelshorthands}
	\item Support Catalan geminated l
	\item Bugfix when declaring more than one pkg option
	\item Protect ¦\reset@font¦
	\item Add missing requirement \pkg{makecmds}
	\item Bugfix for smallcaps in captions
	\item Typo for ¦ccname¦ in Hebrew
	\item Add option \xpgoption{numerals} to ¦gloss-russian¦
	\item Provide ¦\newXeTeXintercharclass¦ when undefined
	\item Bugfix for Russian ¦\alph¦
	\item Remove superfluous level of ¦{}¦ in definition of ¦\markright¦
	\item Bugfix for ¦\datecatalan¦
	\item Change ¦hyphenmins¦ for Sanskrit
\end{itemize}
							
\subsection*{1.1.0b (2009/11/22)}
\begin{itemize}
	\item Modify ¦\hyphenmins¦ for Sanskrit (\TA{Yves Codet})
	\item Bugfixes for Serbian and Bulgarian (\TA{Enrico Gregorio})
\end{itemize}

\subsection*{1.1.0a (2009/11/22)}
\begin{itemize}
	\item Bugfix for interchar tokens
\end{itemize}

\subsection*{1.1.0 (2009/11/20)}
\begin{itemize}
	\item Use ¦\newXeTeXintercharclass¦ (thanks to \TA{Enrico Gregorio})
	\item Fixed implementation of shorthands for German (\pkg{babel} code in file ¦babelsh.def¦)
	\item Arabic (\TA{Khaled Hosny}):
	\begin{itemize}
		\item Fix abjad form for 3 and 5 and add option ¦\abjadjimnotail¦
		\item bugfix for ¦\arabicnumber¦
		\item make Gregorian calendar the default
		\item fixed typos in the sample text
	\end{itemize}
	\item Turkish (\TA{S. Ö. Yıldız}):
	\begin{itemize}
		\item fix white-space before : and !
		\item also check if the font specified TRK for language
		\item added missing Turkish translation of ``Glossary''
	\end{itemize}
	\item Suppress ¦nopattern¦ warning for non-hyphenated scripts
	\item Changed U+0163 to U+021B for Romanian (\TA{Elie Roux})
	\item Stylistic fixes and use macro ¦\xpg@option¦ for package options (\TA{E. Gregorio})
	\item Fix month names in Dutch (\TA{A. Ledda})
	\item Add Brazilian translation for ``glossary''
	\item Remove spurious space generated by ¦gloss-spanish¦
	\item Fix ¦ldf¦ file for brazilian
	\item Various improvements in the code communicated by \TA{E. Gregorio}:
	\begin{itemize}
		\item remove superfluous ¦\protect\language¦
		\item change default language from ¦0¦ to ¦\l@nohyphenation=255¦
		\item localize ¦lccode¦ handling of apostrophe in French; add it to Italian
	\end{itemize}
	\item Fix ¦frenchspacing¦ for Vietnamese
	\item Other minor bugfixes
\end{itemize}
							
\subsection*{1.0.2 (2009/01/27)}
\begin{itemize}
	\item Captions corrected in Hebrew, Russian and Spanish
	\item Removed all ¦\text<lang>¦ wrappers within caption definitions
	\item Improved compatibility with \pkg{babel}
	\item New option \xpgoption{babelshorthands} for German
	\item New option \xpgoption{Script} for Sanskrit
\end{itemize}
							
\subsection*{1.0.1 (2008/07/31)}
\begin{itemize}
	\item Improved documentation (added sections on font setup and numeration mappings)
	\item Improvements and bug fixes for English and German
	\item Bugfix in ¦gloss-syriac.ldf¦ (spurious space after ¦\textsyriac{...}¦)
	\item Extended the scope of ¦\syriacabjad¦
	\item Added ¦gloss-amharic.ldf¦ (ported from ¦ethiop.ldf¦ in the package \pkg{ethiop})
\end{itemize}
							
\subsection*{1.0 (2008/07/13)}
\begin{itemize}
	\item Initial release on CTAN.
\end{itemize}
\egroup

\section{Acknowledgements (by François Charette)}
\pkg{Polyglossia} is notable for being a recycle box of previous contributions
by other people. I take this opportunity to thank the following individuals,
whose splendid work has made my task almost trivial in comparision: \TA{Johannes
Braams} and the numerous contributors to the \pkg{babel} package (in particular
\TA{Boris Lavva} and others for its Hebrew support), \TA{Alexej Kryukov} (\pkg{antomega}),
\TA{Will Robertson} (\pkg{fontspec}), \TA{Apostolos Syropoulos} (\pkg{xgreek}), \TA{Youssef Jabri}
(\pkg{arabi}), and \TA{Vafa Khalighi} (\pkg{xepersian} and \pkg{bidi}).
The work of \TA{Mojca Miklavec} and \TA{Arthur Reutenauer} on hyphenation patterns with their package
\pkg{hyph-utf8} is of course invaluable. I should also thank other
individuals for their assistance in supporting specific languages: \TA{Yves Codet}
(Sanskrit), \TA{Zdenĕk Wagner} (Hindi), \TA{Mikhal Oren} (Hebrew), \TA{Sergey Astanin} (Russian),
\TA{Khaled Hosny} (Arabic), \TA{Sertaç Ö. Yıldız} (Turkish), \TA{Kamal Abdali} (Urdu),
and several other members of the \XeTeX\ user community, notably \TA{Enrico Gregorio}, who
has sent me many useful suggestions and corrections and contributed the \cmd\newXeTeXintercharclass\
mechanism in xelatex.ini which is now used by polyglossia.
More recently, \TA{Kevin Godby} of the \href{http://ubuntu-manual.org}{Ubuntu Manual} project has
contributed very useful feedback, bug hunting and, with the help of translators,
new language definition files for Asturian, Lithuanian, Occitan, Bengali, Malayalam, Marathi, Tamil, and Telugu.
It is particularly heartening to realize that this package is used to typeset a widely-read
document in dozens of different languages!
Support for Lao was also added thanks to \TA{Brian Wilson}.
I also thank \TA{Alan Munn} for kindly proof-reading the penultimate version of this documentation.
And of course my gratitude also goes to \TA{Jonathan Kew}, the formidable author of \XeTeX!

\section{More acknowledgements (by the current development team)}
Many thanks to all the people who have contributed bugfixes and new features to \pkg{polyglossia}
since we took over.
In alphabetical order: \TA{Ignas Anikevicius}, \TA{Sina Ahmadi}, \TA{Wouter Bolsterlee}, \TA{Christian Buhtz},
\TA{Zgarbul Andrey}, \TA{Oleg Domanov}, user \TA{fakhriaunur}, \TA{Philipp Gesang}, \TA{Kevin Godby}, \TA{Enrico Gregorio},
\TA{Khaled Hosny}, \TA{Najib Idrissi}, user \TA{julroy67}, \TA{Dohyun Kim}, \TA{Phil Kime}, \TA{Mike Kroutikov},
\TA{Ivan Kokan}, \TA{Caleb Maclennan}, \TA{Gyula Magyarkuti}, \TA{José Mancera}, \TA{Miquel Ortega}, \TA{Yevgen Pogribnyi},
\TA{Will Robertson}, \TA{Maïeul Rouquette}, \TA{Elie Roux}, \TA{Hugo Roy},  \TA{Guy Rutenberg}, \TA{Marcin Serwin},
\TA{Arvinder Singh}, \TA{Philipp Stephani}, \TA{Niranjan Tambe}, \TA{Takuji Tanaka}, \TA{Osman Tursun}, \TA{Keno Wehr},
\TA{Dominik Wujastyk}, \TA{Sertaç Ö. Yıldız}, \TA{Maksim Zholudev}, \TA{Yan Zhou}, and \TA{Stefan Zlatinov}.
Their respective contributions can be identified from the contributor statistics on
\href{https://github.com/reutenauer/polyglossia/graphs/contributors}{GitHub}.

Among the ones who sent contributions directly to us we would like to especially thank
\TA{Claudio Beccari}, the indefatigable champion of Romance languages, and beyond! Furthermore,
kudos go to \TA{Moritz Wemheuer} (of \pkg{biblatex}) who has helped a lot to improve \pkg{polyglossia} interaction
with \pkg{biblatex} and \pkg{csquotes}.

Not at least, we are very grateful for all bug reports and feature enhancement requests we received from
the numerous users we cannot list all here (but again, you can find all names on \href{https://github.com/reutenauer/polyglossia/issues?utf8=%E2%9C%93&q=is%3Aissue}{GitHub}).
Please go on with that, you are keeping \pkg{polyglossia} running!

\end{document}

