\documentclass{article}
\usepackage{fontspec}
\usepackage{polyglossia}
\setdefaultlanguage{chinese}
\setotherlanguage{english}
% Package noto, notoCJKsc
\setmainfont{Noto Serif CJK SC}[Script=CJK]
\newfontfamily\chinesefont{Noto Serif CJK SC}[Script=CJK]
% Package arphic-ttf
%\setmainfont{bkai00mp.ttf}[Script=CJK]% ZenKai-Medium
%\setmainfont{bsmi00lp.ttf}[Script=CJK]% ShanHeiSun-Light
%\setmainfont{gbsn00lp.ttf}[Script=CJK]% BousungEG-Light-GB
%\setmainfont{gkai00mp.ttf}[Script=CJK]% GBZenKai-Medium
\begin{document}
\XeTeXlinebreaklocale "zh"
\XeTeXlinebreakskip=0em plus 0.1em minus 0.01em
\parindent0em

\begin{center}
	\abstractname
\end{center}
\begin{english}
All human beings are born free and equal in dignity and rights.
They are endowed with reason and conscience and should act towards one another in a spirit of brotherhood.
\footnote{%
	This is a footnote.}

This is today: \today
\end{english}

\section{第一条 (简体中文, Simplified Chinese)}

人人生而自由、在尊严和权利上一律平等。他们赋有理性和良心、并应以兄弟关系的精神相对待。

\today

localnumeral: \localnumeral{1863}, chinesenumeral: \chinesenumeral{1863}

\renewfontfamily\chinesefont{Noto Serif CJK TC}[Script=CJK]
\begin{chinese}[variant=traditional,numerals=chinese]

\section{第一條 (繁體中文, Traditional Chinese)}

人人生而自由、在尊嚴和權利上一律平等。他們賦有理性和良心、並應以兄弟關係的精神相對待。

\today

localnumeral: \localnumeral{1863}, chinesenumeral: \chinesenumeral{1863}

\end{chinese}

\section{numeral}
\begin{english}

numerals=arabic: \textchinese[numerals=arabic]{\localnumeral{1863}, \chinesenumeral{1863}}

numerals=chinese: \textchinese[numerals=chinese]{\localnumeral{1863}, \chinesenumeral{1863}}

\end{english}

\end{document}
